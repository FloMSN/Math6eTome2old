\begin{enigme}[Les œufs \small{(d’après le GVJM)}]

Deux œufs d’autruche permettent de faire une omelette qu’on pourrait faire avec 45 œufs de poule. Avec 9 œufs de poule, on fait une omelette pour 5 personnes. \\[0.5em]
Combien faudrait-il d’œufs d’autruche pour faire une omelette pour 100 personnes ?

\end{enigme} 

%%%%%%%%%%%%%%%%%%%%%%%%%%%%%%%%%%%%%%%%%%%%%%%%%%%%%%%%%%%%%%%%%%%%%%%%%%

\begin{enigme}[Des billes \small{(source : www.educalire.net)}]

Paul a 20 ans ; il décide de donner ses 738 billes à ses 3 frères, âgés de 11, 14 et 16 ans. Il veut les partager proportionnellement à l’âge de chacun. \\[0.5em]
Combien chaque frère recevra-t-il de billes ?
           
\end{enigme} 

%%%%%%%%%%%%%%%%%%%%%%%%%%%%%%%%%%%%%%%%%%%%%%%%%%%%%%%%%%%%%%%%%%%%%%%%%%

\begin{enigme}[Les pommes \small{(d’après le GVJM)}]

Deux paniers, $A$ et $B$, contiennent des pommes. Il y a 2 fois plus de pommes dans le panier $A$ que dans le panier $B$. Un voleur prend 18 pommes et pourtant il reste encore 2 fois plus de pommes dans le panier $A$ que dans le panier $B$. \\[0.5em]
Combien de pommes ont été volées dans le panier $A$ ?

\end{enigme} 

%%%%%%%%%%%%%%%%%%%%%%%%%%%%%%%%%%%%%%%%%%%%%%%%%%%%%%%%%%%%%%%%%%%%%%%%%%

\begin{enigme}[Les bougies \small{(d’après le GVJM)}]

Fonfon Labricole s’est aperçu que les bougies ne se consument jamais complètement. Avec 7 restes de bougies, il fabrique une grande bougie. \\[0.5em]
Quel est le maximum de grandes bougies qu’il peut allumer avec 49 restes de bougies et un briquet ?

\end{enigme} 

%%%%%%%%%%%%%%%%%%%%%%%%%%%%%%%%%%%%%%%%%%%%%%%%%%%%%%%%%%%%%%%%%%%%%%%%%%