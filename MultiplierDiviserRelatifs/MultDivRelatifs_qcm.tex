\begin{acquis}
\begin{itemize}
\item BlaBla1
\item BlaBla2
\item BlaBla3
\item BlaBla4
\item BlaBla5
\item BlaBla6
\end{itemize}
\end{acquis}

\QCMautoevaluation{Pour chaque question, plusieurs réponses sont
  proposées.  Déterminer celles qui sont correctes.}

\begin{QCM}
  \begin{GroupeQCM}
    \begin{exercice}
      $(- 10) + (+ 15) = \ldots$
      \begin{ChoixQCM}{4}
      \item $(- 5)$
      \item $(- 150)$
      \item $(+ 5)$
      \item $(- 25)$
      \end{ChoixQCM}
\begin{corrige}
     \reponseQCM{a} %j'ai mis "a" partout
   \end{corrige}
    \end{exercice}
    
    
    \begin{exercice}
      (+ 8) + \ldots = (- 5)
      \begin{ChoixQCM}{4}
      \item $(+ 3)$
      \item impossible
      \item $(- 13)$
      \item $(- 3)$
      \end{ChoixQCM}
\begin{corrige}
     \reponseQCM{a}
   \end{corrige}
    \end{exercice}
    
    
    \begin{exercice}
      $(+ 2,1) + (- 3,9) = \ldots$
      \begin{ChoixQCM}{4}
      \item 6
      \item $- 6$
      \item $- 1,8$
      \item 1,8
      \end{ChoixQCM}
\begin{corrige}
     \reponseQCM{a}
   \end{corrige}
    \end{exercice}
    
    
    \begin{exercice}
      $(+ 7) - (- 3) = \ldots$
      \begin{ChoixQCM}{4}
      \item 4
      \item 10
      \item $- 4$
      \item $- 10$
      \end{ChoixQCM}
\begin{corrige}
     \reponseQCM{a}
   \end{corrige}
    \end{exercice}
    
    
    \begin{exercice}
      $(- 2) - \ldots = (- 5)$
      \begin{ChoixQCM}{4}
      \item $(+ 3)$
      \item $(- 7)$
      \item $(+ 7)$
      \item $(- 3)$
      \end{ChoixQCM}
\begin{corrige}
     \reponseQCM{a}
   \end{corrige}
    \end{exercice}
    
    
    \begin{exercice}
      $1,3 - (- 2,4) = \ldots$
      \begin{ChoixQCM}{4}
      \item $- 1,1$
      \item $1,1$
      \item $3,7$
      \item $- 3,7$
      \end{ChoixQCM}
\begin{corrige}
     \reponseQCM{a}
   \end{corrige}
    \end{exercice}
    
    
    \begin{exercice}
      $- 7 \cdot (- 3) = \ldots$
      \begin{ChoixQCM}{4}
      \item $- 10$
      \item $- 21$
      \item 10
      \item 21
      \end{ChoixQCM}
\begin{corrige}
     \reponseQCM{a}
   \end{corrige}
    \end{exercice}
    
    
    \begin{exercice}
      $4 \cdot (- 3) = \ldots$
      \begin{ChoixQCM}{4}
      \item 1
      \item $- 12$
      \item $- 7$
      \item 12
      \end{ChoixQCM}
\begin{corrige}
     \reponseQCM{a}
   \end{corrige}
    \end{exercice}
    
    
    \begin{exercice}
      $- 15 : (- 5) = \ldots$
      \begin{ChoixQCM}{4}
      \item $(- 15) : (- 5)$
      \item $- 3$
      \item $15 : 5$
      \item $3$
      \end{ChoixQCM}
\begin{corrige}
     \reponseQCM{a}
   \end{corrige}
    \end{exercice}
    
    
    \begin{exercice}
      $4 \cdot (- 4) = \ldots$
      \begin{ChoixQCM}{4}
      \item 0
      \item $- 8$
      \item 16
      \item $- 16$
      \end{ChoixQCM}
\begin{corrige}
     \reponseQCM{a}
   \end{corrige}
    \end{exercice}

    
    \begin{exercice}
      Le produit de l'opposé de $- 6$ par l'opposé de 7 vaut \ldots
      \begin{ChoixQCM}{4}
      \item 42
      \item $- 42$
      \item $- 1$
      \item $6 : (- 7)$
      \end{ChoixQCM}
\begin{corrige}
     \reponseQCM{a}
   \end{corrige}
    \end{exercice}
    
    
    \begin{exercice}
      $- 6 + 6 \cdot (- 10) = \ldots$
      \begin{ChoixQCM}{4}
      \item 0
      \item 120
      \item 66
      \item $- 66$
      \end{ChoixQCM}
\begin{corrige}
     \reponseQCM{a}
   \end{corrige}
    \end{exercice}


    \begin{exercice}
      Le produit de 108 facteurs égaux à $- 1$ est égal à \ldots
      \begin{ChoixQCM}{4}
      \item $- 108$
      \item 0
      \item 1
      \item $- 1$
      \end{ChoixQCM}
\begin{corrige}
     \reponseQCM{a}
   \end{corrige}
    \end{exercice}
    
\end{GroupeQCM}
\end{QCM}

  