
\serie{Produits de relatifs}

\begin{exercice}
Complète : 
\begin{enumerate}
 \item $A = (- 4) + (- 4) + (- 4) + (- 4) + (- 4)$
 
$A = (- 4) \cdot \ldots \ldots$

$A = \ldots \ldots$
 \item $B = (- 8,2) + (- 8,2) + (- 8,2)$
 
$B = (- 8,2) \cdot \ldots \ldots$

$B = \ldots \ldots$
 \item $C = (- 1,7) + (- 1,7) + (- 1,7) + (- 1,7)$

$C = (- 1,7) \cdot \ldots \ldots$

$C = \ldots \ldots$
 \end{enumerate}
\end{exercice}


\begin{exercice}
Sans les calculer, donne le signe de chacun des produits suivants :
\begin{colenumerate}{2}
 \item $(- 12) \cdot (+ 2)$ ;
 \item $(+ 34) \cdot (- 28)$ ;
 \item $(- 10,3) \cdot (- 46)$ ;
 \item $(+ 12,5) \cdot (+ 3,1)$.
 \end{colenumerate}
\end{exercice}


\begin{exercice}
Sans les calculer, donne le signe de chacun des produits suivants :
\begin{colenumerate}{2}
 \item $- 36 \cdot (- 1)$ ;
 \item $(- 2) \cdot (+ 24)$ ; 
 \item $2,3 \cdot (- 2,3)$ ;
 \item $- 9,1 \cdot 6$.
 \end{colenumerate}
\end{exercice}


\begin{exercice}
Quel est le signe du résultat quand on \ldots
\begin{enumerate}
 \item \ldots multiplie un nombre négatif par un nombre positif ?
 \item \ldots multiplie quatre nombres négatifs entre eux ?
 \item \ldots multiplie un nombre positif et deux nombres négatifs ?
 \item \ldots multiplie un nombre relatif par lui-même ?
 \item \ldots multiplie trois nombres négatifs entre eux ?
 \end{enumerate}
\end{exercice}


\begin{exercice}
Effectue :
\begin{colenumerate}{2}
 \item $(+ 5) \cdot (- 4)$ ;
 \item $(- 5) \cdot (- 3)$ ;
 \item $(- 3) \cdot (+ 4)$ ;
 \item $(+ 4) \cdot (+ 4)$ ;
 \item $(- 4) \cdot (- 3)$ ;
 \item $(- 5) \cdot (- 4)$ ;
 \item $(- 5) \cdot (+ 3)$ ;
 \item $(- 4) \cdot (+ 4)$.
 \end{colenumerate}
\end{exercice}


\begin{exercice}
Effectue :
\begin{colenumerate}{2}
 \item $(- 8) \cdot (+ 2)$ ;
 \item $(- 2) \cdot (+ 5) $ ;
 \item $(- 4) \cdot (- 8)$ ;
 \item $(+ 9) \cdot (+ 10)$ ; 
 \item $(+ 191) \cdot (+ 0,1)$ ; 
 \item $(- 1,5) \cdot (+ 20)$ ;
 \item $(- 0,25) \cdot (- 4)$ ;
 \item $(+ 0,8) \cdot (- 3)$ ;
 \item $(- 3,2) \cdot (+ 4)$ ;
 \item \phantom{.} $(- 1) \cdot (- 17)$.
 \end{colenumerate}
\end{exercice}


\begin{exercice}
Calcule, sachant que $11,2 \cdot 2,5 = 28$ :
\begin{colenumerate}{2}
 \item $11,2 \cdot (- 2,5)$ ;
 \item $- 11,2 \cdot (- 2,5)$.
 \end{colenumerate}
\end{exercice}


\begin{exercice}
\emph{Un produit peut en cacher un autre \ldots}
\begin{enumerate}
 \item Calcule le produit $7,5 \cdot 0,2$ ;
 \item Effectue alors les calculs suivants :
 \begin{colitemize}{2}
  \item $A = 7,5 \cdot (- 0,2)$ ;
  \item $B = (- 0,2) \cdot (- 7,5)$ ;
  \item $C = (- 75) \cdot (+ 0,2)$ ;
  \item $D = (- 7,5) \cdot (- 20)$.
  \end{colitemize}
 \end{enumerate}
\end{exercice}


\begin{exercice}
Relie les expressions dont les produits sont égaux :
\begin{center}
 \begin{tabularx}{\linewidth}{|c|cXc|c|}
  \cline{1-1}\cline{5-5}
  \cellcolor{F3} $(+ 5) \cdot (- 12)$ & \cellcolor{F2} $\cdot$ & & \cellcolor{F2} $\cdot$ & \cellcolor{F3} $(- 1) \cdot (+ 20)$ \\  \cline{1-1}\cline{5-5}
  \cellcolor{F3} $(- 8) \cdot (- 3)$ & \cellcolor{F2} $\cdot$ & & \cellcolor{F2} $\cdot$ & \cellcolor{F3} $(+ 12) \cdot (+ 5)$ \\ \cline{1-1}\cline{5-5}
  \cellcolor{F3} $(+ 4) \cdot (- 6)$ & \cellcolor{F2} $\cdot$ & & \cellcolor{F2} $\cdot$ & \cellcolor{F3} $(+ 2) \cdot (+ 12)$ \\ \cline{1-1}\cline{5-5}
  \cellcolor{F3} $(+ 5) \cdot (- 4)$ & \cellcolor{F2} $\cdot$ & & \cellcolor{F2} $\cdot$ & \cellcolor{F3} $(+ 5) \cdot (+ 4)$ \\ \cline{1-1}\cline{5-5}
  \cellcolor{F3} $(+ 2) \cdot (+ 10)$ & \cellcolor{F2} $\cdot$ & & \cellcolor{F2} $\cdot$ & \cellcolor{F3} $(- 3) \cdot (+ 20)$ \\ \cline{1-1}\cline{5-5}
  \cellcolor{F3} $(- 2) \cdot (- 30)$ & \cellcolor{F2} $\cdot$ & & \cellcolor{F2} $\cdot$ & \cellcolor{F3} $(- 12) \cdot (+ 2)$ \\ \cline{1-1}\cline{5-5}
  \end{tabularx}
\end{center}
\end{exercice}


\begin{exercice}
Recopie et complète cette table de multiplication :
\begin{center}
 \renewcommand*\tabularxcolumn[1]{>{\centering\arraybackslash}m{#1}}
 \begin{ttableau}{\linewidth}{6}
  \hline
  \rowcolor{A2} $\cdot$ & $- 3$ & $+ 5$ & $- 9$ & $+ 6$ & $- 8$ \\\hline
  \cellcolor{A2} $- 1$ & \cellcolor{A3} & \cellcolor{A3} & \cellcolor{A3} & \cellcolor{A3} & \cellcolor{A3} \\\hline
  \cellcolor{A2} $+ 4$ & \cellcolor{A3} & \cellcolor{A3} & \cellcolor{A3} & \cellcolor{A3} & \cellcolor{A3} \\\hline
  \cellcolor{A2} $- 7$ & \cellcolor{A3} & \cellcolor{A3} & \cellcolor{A3} & \cellcolor{A3} & \cellcolor{A3} \\\hline
  \cellcolor{A2} $0$ & \cellcolor{A3} & \cellcolor{A3} & \cellcolor{A3} & \cellcolor{A3} & \cellcolor{A3} \\\hline
  \end{ttableau}
\end{center}
\end{exercice}


\begin{exercice}
Recopie et complète les « pyramides » suivantes sachant que le nombre contenu dans une case est le produit des nombres contenus dans les deux cases situées en dessous de lui : \\[0.5em]
\begin{minipage}[c]{0.48\linewidth}
\begin{center} \fcolorbox{Noir}{C1}{\phantom{hello}} \end{center}
\vspace{-0.69cm}
\begin{center} \fcolorbox{Noir}{C4}{\phantom{hello}} \negthinspace \fcolorbox{Noir}{C4}{\phantom{hello}} \end{center}
\vspace{-0.71cm}
\begin{center} \fcolorbox{Noir}{C2}{\phantom{hello}} \negthinspace \fcolorbox{Noir}{C2}{\phantom{hello}} \negthinspace  \fcolorbox{Noir}{C2}{\phantom{hello}} \end{center}
\vspace{-0.69cm}
\begin{center} \negthinspace \fcolorbox{Noir}{C3}{\phantom{!}$- 2$\phantom{!}} \negthinspace \fcolorbox{Noir}{C3}{\phantom{!}$+ 2$\phantom{!}} \negthinspace \fcolorbox{Noir}{C3}{\phantom{!}$- 2$\phantom{!}} \negthinspace \fcolorbox{Noir}{C3}{\phantom{!}$- 2$\phantom{!}} \end{center}
 \end{minipage} \hfill%
 \begin{minipage}[c]{0.48\linewidth}
\begin{center} \fcolorbox{Noir}{A1}{\phantom{hello}} \end{center}
\vspace{-0.69cm}
\begin{center} \fcolorbox{Noir}{A4}{\phantom{hello}} \negthinspace \fcolorbox{Noir}{A4}{\phantom{hello}} \end{center}
\vspace{-0.71cm}
\begin{center} \fcolorbox{Noir}{A2}{\phantom{hello}} \negthinspace \fcolorbox{Noir}{A2}{\phantom{hello}} \negthinspace \fcolorbox{Noir}{A2}{\phantom{hello}} \end{center}
\vspace{-0.71cm}
\begin{center} \fcolorbox{Noir}{A3}{\phantom{!}$- 1$\phantom{!}} \negthinspace \fcolorbox{Noir}{A3}{\phantom{!}$+ 1$\phantom{!}} \negthinspace \fcolorbox{Noir}{A3}{\phantom{!}$+ 1$\phantom{!}} \negthinspace \fcolorbox{Noir}{A3}{\phantom{!}$- 1$\phantom{!}} \end{center}
  \end{minipage} \\
\end{exercice}


\begin{exercice}
Donne le signe de chacun des produits suivants :
\begin{enumerate}
 \item $5,4 \cdot (- 3,2) \cdot (+ 4) \cdot (- 5,1)$ ;
 \item $(- 0,5) \cdot (- 9) \cdot 0 \cdot 7 \cdot (- 1,4) \cdot (- 1)$ ;
 \item $- 6 \cdot (- 10) \cdot 4 \cdot (- 9) \cdot (- 3) \cdot (- 4,1)$.
 \end{enumerate}
\end{exercice}


\begin{exercice}
Effectue les calculs suivants :
\begin{enumerate}
 \item $(- 2) \cdot (- 3) \cdot (+ 5)$ ;
 \item $(- 3) \cdot (- 2) \cdot (- 4)$ ;
 \item $(+ 6) \cdot (- 1) \cdot (+ 3)$.
 \end{enumerate}
\end{exercice}


\begin{exercice}
Effectue les calculs suivants :
\begin{enumerate}
 \item $(- 3,2) \cdot (- 10) \cdot (+ 2) \cdot (- 0,5)$ ;
 \item $(- 75) \cdot (- 0,25) \cdot (+ 4) \cdot (+ 2)$ ;
 \item $(- 3) \cdot (- 0,1) \cdot (+ 5) \cdot (+ 4)$ ;
 \item $(- 1,5) \cdot (+ 4) \cdot (- 1) \cdot (+ 0,8) \cdot (- 3)$ ;
 \item $(+ 2) \cdot (- 10) \cdot (+ 3) \cdot (- 1) \cdot (- 1)$.
 \end{enumerate}
\end{exercice}


\begin{exercice}
Calcule astucieusement :
\begin{enumerate}
 \item $(- 2) \cdot (- 1,25) \cdot (- 2,5) \cdot (- 8)$ ;
 \item $(- 75) \cdot (- 0,25) \cdot (+ 2) \cdot (+ 4)$ ;
 \item $(+ 0,01) \cdot (- 25) \cdot (- 13,2) \cdot 4 \cdot (- 3)$.
 \end{enumerate}
\end{exercice}


\begin{exercice}
Complète par le nombre qui convient :
\begin{colenumerate}{2}
 \item $(- 4) \cdot \text{§} = 20$ ;
 \item $(- 13) \cdot \text{§} = - 39$ ;  
 \item $\text{§} \cdot 7 = - 42$ ;
 \item $\text{§} \cdot (- 11) = 121$.
 \end{colenumerate}
\end{exercice}


\begin{exercice}
Complète par le nombre qui convient :
\begin{colenumerate}{2}
 \item $(+ 4) \cdot \text{§} = - 100$ ;
 \item $(- 2,9) \cdot \text{§} = 29$ ;  
 \item $\text{§} \cdot 17 = - 17$ ;
 \item $\text{§} \cdot (- 3) = - 99$.
 \end{colenumerate}
\end{exercice}


\begin{exercice}[Suite logique de nombres]
Donne le signe de chacun des produits suivants :
\begin{enumerate}
 \item $(- 1) \cdot 2 \cdot (- 3) \cdot 4 \cdot \ldots \cdot (- 9)$ ;
 \item $(- 1) \cdot (- 2) \cdot (- 3) \cdot (- 4) \cdot \ldots \cdot (- 12)$ ;
 \item $(- 4) \cdot (- 3) \cdot (- 2) \cdot \ldots \cdot 3 \cdot 4 \cdot 5$ ;
 \item $5 \cdot (- 10) \cdot 15 \cdot (- 20) \cdot \ldots \cdot (- 100)$ ;
 \item $1 \cdot (- 2) \cdot 4 \cdot (- 8) \cdot \ldots \cdot 1\,024$.
 \end{enumerate}
\end{exercice}


\begin{exercice}[Températures]
Il fait $0^\circ$C et la température chute de deux degrés toutes les heures. 
\begin{enumerate}
 \item Combien de temps faudra-t-il pour que la température atteigne $- 10^\circ$C ?
 \item Quelle sera la température dans huit heures ?
 \end{enumerate}
\end{exercice}


\begin{exercice}
Calcule dans chaque cas le produit $x \cdot y$ :
\begin{colenumerate}{2}
 \item $x = 5$ et $y = - 3$ ;
 \item $x = + 4$ et $y = - 11$ ;
 \item $x = - 2$ et $y = - 5$ ;
 \item $x = - 0,5$ et $y = - 5,2$.
 \end{colenumerate}
\end{exercice}


\begin{exercice}
Recopie et complète le tableau suivant :
\begin{center}
\begin{tabular}{|c|c|c|c|c|c|c|}
\hline
\cellcolor{H2} $a$ & \cellcolor{H2} $b$ & \cellcolor{H2} $c$ & \cellcolor{A2} $ab$ &  \cellcolor{A2} ($-$ $a$) $\cdot$ $c$ &  \cellcolor{A2} $-$ ($a$ $\cdot$ $c$) &  \cellcolor{A2} $a$ $\cdot$ $b$ $\cdot$ $c$ \\\hline 
\cellcolor{H3} $- 5$ & \cellcolor{H3} $+ 6$ & \cellcolor{H3} $- 4$ & \cellcolor{A3} & \cellcolor{A3} & \cellcolor{A3} & \cellcolor{A3} \\\hline
\cellcolor{H3} $- 1$ & \cellcolor{H3} $- 2$ & \cellcolor{H3} $- 3$ & \cellcolor{A3} & \cellcolor{A3} & \cellcolor{A3} & \cellcolor{A3} \\\hline
\cellcolor{H3} $- 2,1$ & \cellcolor{H3} $- 4$ & \cellcolor{H3} $+ 3$ & \cellcolor{A3} & \cellcolor{A3} & \cellcolor{A3} & \cellcolor{A3} \\\hline
 \end{tabular}
 \end{center}
\end{exercice}


\begin{exercice}[Décompositions \ldots]
\begin{enumerate}
 \item Trouve toutes les façons de décomposer le nombre – 20 en produit de deux nombres entiers relatifs.
 \item Trouve toutes les façons de décomposer le nombre 24 en produit de trois nombres entiers relatifs.
 \end{enumerate}
\end{exercice}


\begin{exercice}
Sans calculer, donne le signe de chaque résultat :
\begin{colenumerate}{3}
 \item $(- 6)^{4}$ ;
 \item $6^{8}$ ;
 \item $- 132^{51}$ ;
 \item $(- 12)^{15}$ ;
 \item $(- 3)^{7}$ ;
 \item $(- 6)^{100}$ ;
 \item $- (- 35)^{7}$ ;
 \item $- 87^{4}$ ;
 \item $- (- 13^{8})$.
 \end{colenumerate}
\end{exercice}


\begin{exercice}[Puissance de 1 ou de $- 1$]
Calcule :
\begin{colenumerate}{4}
 \item $1^{12}$ ;
 \item $1^{0}$ ;
 \item $(- 1)^{8}$ ;
 \item $(- 1)^{0}$ ;
 \item $- 1^{7}$ ;
 \item $- 1^{6}$ ;
 \item $(- 1)^{9}$ ;
 \item $- 1^{0}$.
 \end{colenumerate}
\end{exercice}

%%%%%%%%%%%%%%%%%%%%%%%%%%%%%%%%%%%%%%%%%%%%%%%%%%%%%%%%%%%%%%%%%%%%%%%%%%%%%

\serie{Quotients de relatifs}

\begin{exercice}
Complète chaque égalité et écris chaque facteur manquant \textcolor{PartieGeometrie}{§} sous la forme d'un quotient :
\begin{enumerate}
 \item $(+ 6) \cdot \text{\textcolor{PartieGeometrie}{§}} = + 18$ donc $\text{\textcolor{PartieGeometrie}{§}} = \ldots$ ;
 \item $(+ 5) \cdot \text{\textcolor{PartieGeometrie}{§}} = - 20$ donc $\text{\textcolor{PartieGeometrie}{§}} = \ldots$ ;
 \item $\text{\textcolor{PartieGeometrie}{§}} \cdot (- 7) = + 14$ donc  $\text{\textcolor{PartieGeometrie}{§}} = \ldots$ ;
 \item $(- 2) \cdot \text{\textcolor{PartieGeometrie}{§}} = + 12$ donc  $\text{\textcolor{PartieGeometrie}{§}} = \ldots$ ;
 \item $\text{\textcolor{PartieGeometrie}{§}} \cdot (- 10) = - 130$ donc  $\text{\textcolor{PartieGeometrie}{§}} = \ldots$.
 \end{enumerate}
\end{exercice}


\begin{exercice}
Sans les calculer, donne le signe de chacun des quotients suivants :
\begin{colenumerate}{2}
 \item $(- 3) \div (- 8)$ ;
 \item $(+ 1) \div (- 2)$ ;
 \item $(- 4) \div (- 5)$ ;
 \item $(- 3,7) \div (+ 5,1)$.
 \end{colenumerate}
\end{exercice}


\begin{exercice}
Calcule mentalement :
\begin{colenumerate}{2}
 \item $64 \div (- 8)$ ;
 \item $42 \div (- 6)$ ;
 \item $- 24 \div (- 3)$ ;
 \item $81 \div (+ 9)$ ;
 \item $- 17 \div (- 1)$ ;
 \item $- 35 \div 7$ ;
 \item $(- 54) \div (- 6)$ ;
 \item $25 \div (- 5)$ ;
 \item $(- 4) \div (+ 4)$ ;
 \item $(- 29) \div (+ 1)$.
 \end{colenumerate}
\end{exercice}


\begin{exercice}
Calcule mentalement :
\begin{colenumerate}{2}
 \item $(- 100) \div (+ 25)$ ;
 \item $(- 42) \div (- 4)$ ;
 \item $(+ 54) \div (- 3)$ ;
 \item $(+ 55) \div (+ 5)$ ;
 \item $(- 24) \div (- 5)$ ;
 \item $(- 13)  \div (- 10)$.
 \end{colenumerate}
\end{exercice}


\begin{exercice}
Calcule le quotient de $x$ par $y$ :
\begin{colenumerate}{2}
 \item $x = - 15$ et $y = - 3$ ;
 \item $x = + 64$ et $y = - 8$ ;
 \item $x = - 36$ et $y = 12$ ;
 \item $x = - 2,4$ et $y = 1,2$ ;
 \item $x = y = - 2,3$ ;
 \item $x = 0$ et $y = - 5$.
 \end{colenumerate}
\end{exercice}


\begin{exercice}
Recopie et complète le tableau suivant et donne le résultat sous forme décimale :
\begin{center}
\begin{tabular}{|c|c|c|c|c|c|}
\hline
\cellcolor{H2} $a$ & \cellcolor{H2} $b$ & \cellcolor{H2} $c$ & \cellcolor{A2} $a$ $:$ $b$ & \cellcolor{A2} ($-$ $b$) $:$ $c$ & \cellcolor{A2} $c$ $:$ ($-$ $a$) \\\hline 
\cellcolor{H3} $- 5$ & \cellcolor{H3} $+ 4$ & \cellcolor{H3} $- 4$ & \cellcolor{A3} & \cellcolor{A3} & \cellcolor{A3} \\\hline
\cellcolor{H3} $- 2,5$ & \cellcolor{H3} $- 1$ & \cellcolor{H3} $+ 20$ & \cellcolor{A3} & \cellcolor{A3} & \cellcolor{A3} \\\hline
\cellcolor{H3} $+ 8$ & \cellcolor{H3} $- 4$ & \cellcolor{H3} $- 0,5$ & \cellcolor{A3} & \cellcolor{A3} & \cellcolor{A3} \\\hline
\cellcolor{H3} $- 2,4$ & \cellcolor{H3} $- 1,2$ & \cellcolor{H3} $- 24$ & \cellcolor{A3} & \cellcolor{A3} & \cellcolor{A3} \\\hline
 \end{tabular}
 \end{center}
\end{exercice}


\begin{exercice}
Donne, à l'aide de ta calculatrice, l'arrondi à l'unité de chacun des nombres suivants, comme dans l'exemple : \\[0.5em]
Exemple : $A = \dfrac{- 153}{23}$. \\[0.5em]
La calculatrice donne $A \approx - 6,652173913$. \\[0.5em]
On a donc : $- 7 < A < - 6$. \\[0.5em]
L'arrondi à l'unité de $A$ est $- 7$ car $A$ est plus proche de $- 7$ que de $- 6$.
\begin{colitemize}{3}
 \item $B = \dfrac{39}{- 9}$ ;
 \item $C = \dfrac{- 17}{- 7}$ ;
 \item $D = \dfrac{- 28}{51}$.
 \end{colitemize}
\end{exercice}

%%%%%%%%%%%%%%%%%%%%%%%%%%%%%%%%%%%%%%%%%%%%%%%%%%%%%%%%%%%%%%%%%%%%%%%%%%%%%

\serie{Calculs variés}

\begin{exercice}
Pour chacun des calculs suivants, indique s'il s'agit d'une somme ou d'un produit, puis donne le résultat :
\begin{colitemize}{2}
 \item $- 4 \cdot (+ 9)$ ;
 \item $- 3 - (+ 8)$ ;
 \item $- 7 + (- 5)$ ;
 \item $3 \cdot (- 7)$ ;
 \item $- 8 + (+ 6)$ ;
 \item $+ 9 \cdot (+ 3)$ ;
 \item $- 5 - (- 16)$ ;
 \item $- 11 \cdot (- 4)$.
 \end{colitemize}
\end{exercice}


\begin{exercice}
Sans calculer, donne le signe de chaque résultat :
\begin{colitemize}{2}
 \item $(- 4) \cdot (- 12)$ ;
 \item $(+ 15) + (- 22)$ ;
 \item $(- 45) - (- 51)$ ;
 \item $(- 37) \cdot (+ 51)$ ;
 \item $(+ 7) \cdot (+ 8)$ ;
 \item $(- 7) + (+ 8)$ ;
 \item $(- 3,12) \cdot (- 2,5)$ ;
 \item $(- 3,17) - (+ 3,7)$.
 \end{colitemize}
\end{exercice}


\begin{exercice}
Calcule mentalement :
\begin{colitemize}{2}
 \item $8 \cdot (- 8)$ ;
 \item $- 22 + (- 6)$ ;
 \item $- 14 \cdot 3$ ;
 \item $- 5 - (+ 17)$ ;
 \item $(- 34) + (- 19)$ ;
 \item $- 15 \cdot (- 5)$.
 \end{colitemize}
\end{exercice}


\begin{exercice}
Calcule mentalement :
\begin{colitemize}{2}
 \item $(- 4) \cdot (- 2,5)$ ;
 \item $(+ 3,5) + (- 2,2)$ ;
 \item $(- 3,9) + (- 5,4)$ ;
 \item $(- 3) \cdot (+ 4,2)$ ;
 \item $(+ 2,6) \cdot (- 3)$ ;
 \item $(- 7,15) - (- 2,2)$ ;
 \item $(- 3,12) \cdot (- 10)$ ;
 \item $(- 0,7) - (+ 1,17)$.
 \end{colitemize}
\end{exercice}


\begin{exercice}
Recopie et remplace le symbole \textcolor{PartieGeometrie}{§} par le signe opératoire qui convient :
\begin{enumerate}
 \item $(- 3) \text{\textcolor{PartieGeometrie}{§}} (- 2) = - 5$ ;
 \item $(- 3) \text{\textcolor{PartieGeometrie}{§}} (- 2) = + 6$ ;
 \item $(- 2) \text{\textcolor{PartieGeometrie}{§}} (- 2) = + 4$ ;
 \item $(- 2) \text{\textcolor{PartieGeometrie}{§}} (- 2) = - 4$ ;
 \item $(- 5) \text{\textcolor{PartieGeometrie}{§}} (+ 4) = (- 12) \text{\textcolor{PartieGeometrie}{§}} (+ 8)$.
 \end{enumerate}
\end{exercice}


\begin{exercice}[Logique !]
Complète chaque suite de nombres :
\begin{enumerate}
 \item 3 ; 1 ; $- 1$ ; \ldots ; \ldots ; \ldots ;
 \item 1 ; $- 2$ ; $+ 4$ ; \ldots ; \ldots ; \ldots ;
 \item $- 16$ ; 8 ; $- 4$ ; \ldots ; \ldots ; \ldots ;
 \item 0,5 ; $- 5$ ; 50 ; \ldots ; \ldots ; \ldots .
 \end{enumerate}
\end{exercice}


\begin{exercice}
Recopie et complète les « pyramides » suivantes sachant que le nombre contenu dans une case est le produit des nombres contenus dans les deux cases situées en dessous de lui : \\[0.5em]
\begin{minipage}[c]{0.49\linewidth}
\begin{center} \fcolorbox{Noir}{C1}{\phantom{hello}} \end{center}
\vspace{-0.69cm}
\begin{center} \fcolorbox{Noir}{C4}{\phantom{!}$- 3$\phantom{!}} \negthinspace \fcolorbox{Noir}{C4}{\phantom{hello}} \end{center}
\vspace{-0.71cm}
\begin{center} \fcolorbox{Noir}{C2}{\phantom{hello}} \negthinspace \fcolorbox{Noir}{C2}{\phantom{!}$+ 3$\phantom{!}} \negthinspace  \fcolorbox{Noir}{C2}{\phantom{hello}} \end{center}
\vspace{-0.69cm}
\begin{center} \negthinspace \fcolorbox{Noir}{C3}{\phantom{hello}} \negthinspace \fcolorbox{Noir}{C3}{\phantom{!}$- 1$\phantom{!}} \negthinspace \fcolorbox{Noir}{C3}{\phantom{hello}} \negthinspace \fcolorbox{Noir}{C3}{\phantom{!}$+ 2$\phantom{!}} \end{center}
 \end{minipage} \hfill%
 \begin{minipage}[c]{0.49\linewidth}
\begin{center} \fcolorbox{Noir}{A1}{$- 2\,160$} \end{center}
\vspace{-0.69cm}
\begin{center} \fcolorbox{Noir}{A4}{\phantom{!}$- 24$\phantom{!}} \negthinspace \fcolorbox{Noir}{A4}{\phantom{hello}} \end{center}
\vspace{-0.71cm}
\begin{center} \fcolorbox{Noir}{A2}{\phantom{hello}} \negthinspace \fcolorbox{Noir}{A2}{\phantom{!}$- 6$\phantom{!}} \negthinspace \fcolorbox{Noir}{A2}{\phantom{hello}} \end{center}
\vspace{-0.71cm}
\begin{center} \fcolorbox{Noir}{A3}{\phantom{hello}} \negthinspace \fcolorbox{Noir}{A3}{\phantom{hello}} \negthinspace \fcolorbox{Noir}{A3}{\phantom{hello}} \negthinspace \fcolorbox{Noir}{A3}{\phantom{!}$+ 5$\phantom{!}} \end{center}
  \end{minipage} \\
\end{exercice}


\begin{exercice}
Effectue les calculs suivants en détail :
\begin{enumerate}
 \item $7 + (- 6) \cdot (- 6)$ ;
 \item $13 - (+ 3) \cdot (- 4) - 8$ ;
 \item $- 30 : (- 9 + 15)$ ;
 \item $- 3 - 9 • (- 3)$ ;
 \item $- 3 \cdot 6 \cdot (- 2 + 8)$.
 \end{enumerate}
\end{exercice}


\begin{exercice}
Effectue les calculs suivants en détail :
\begin{enumerate}
 \item $- 22 + (13 - 5) \cdot (- 5)$ ;
 \item $(- 2) \cdot (- 8) + 2 \cdot (- 20) : 4$ ;
 \item $- 28 + (5 - 2) \cdot (- 4)$ ;
 \item $7 \cdot (- 7) + 3 \cdot (- 25) : (- 5)$ ;
 \item $- 3,2 \cdot (- 6) + (- 2,3 - 7,7)$ ;
 \item $150 : (- 1,2 - 9 \cdot 3,2)$.
 \end{enumerate}
\end{exercice}


\begin{exercice}[Vocabulaire]
\begin{enumerate}
 \item Traduis les phrases suivantes par un calcul :
 \begin{itemize}
  \item \textcolor{A1}{La somme du produit de 4 par $- 5$ et de $- 6$ ;}
  \item \textcolor{H1}{Le produit de la somme de 7 et de $- 8$ par la somme de 8 et de $- 2$.}
  \end{itemize}  
 \item Effectue ces calculs.
 \end{enumerate}
\end{exercice}


\begin{exercice}[Vocabulaire (bis)]
Traduis les expressions mathématiques suivantes par des phrases. \\[0.5em]
Exemple : $(- 2) \cdot 3 + 1$ se traduit par \\[0.5em]
« \textcolor{A1}{La somme du produit de $(- 2)$ par 3 et de 1.} »\\[0.5em]
\begin{itemize}
 \item $A = 5 \cdot (- 7) + 3$ ;
 \item $B = 3 + 2 : (- 4)$ ;
 \item $C = 7 - 4 \cdot (- 10)$ ;
 \item $D = (2 - 3) \cdot (- 1 - 2)$ ;
 \item $E = (1 - 7) : (2 + 5)$ ;
 \item $F = - 2 +(- 6) \cdot (- 6) - 9$.
 \end{itemize}
\end{exercice}


\begin{exercice}
Recopie et complète le tableau suivant :
\begin{center}
\begin{tabular}{|c|c|c|c|c|c|c|}
\hline
\cellcolor{H2} $a$ & \cellcolor{H2} $b$ & \cellcolor{H2} $c$ & \cellcolor{A2} $a$ $\cdot$ $b$ &  \cellcolor{A2} ($-$ $a$) $\cdot$ $c$ &  \cellcolor{A2} $-$ ($a$ $\cdot$ $c$) &  \cellcolor{A2} $a$ $\cdot$ $b$ $\cdot$ $c$ \\\hline 
\cellcolor{H3} $- 5$ & \cellcolor{H3} & \cellcolor{H3} $+ 4$ & \cellcolor{A3} & \cellcolor{A3} & \cellcolor{A3} & \cellcolor{A3} \\\hline
\cellcolor{H3} & \cellcolor{H3} & \cellcolor{H3} $+ 2$ & \cellcolor{A3} & \cellcolor{A3} & \cellcolor{A3} $- 12$ & \cellcolor{A3} $- 36$ \\\hline
 \end{tabular}
 \end{center}
\end{exercice}