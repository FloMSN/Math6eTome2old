%\section{Une section}

% remarque : pour qu'un mot se retrouve dans le lexique : \MotDefinition{asymptote horizontale}{} 

\begin{aconnaitre}
Pour multiplier deux nombres relatifs, on multiplie les valeurs absolues et on applique la \MotDefinition{règle des signes}{} :
\begin{itemize}
 \item Le produit de deux nombres relatifs de \textbf{même signe} est \textbf{positif} ;
 \item Le produit de deux nombres relatifs de \textbf{signes opposés} est \textbf{négatif}.
 \end{itemize}
\end{aconnaitre}

\begin{methode*1}[Multiplier deux nombres relatifs]

 \begin{exemple*1}
Effectue la multiplication : $E = (-4) \cdot (-2,5)$. \\[0.5em]
Le résultat est positif car c'est le produit de deux nombres négatifs :

$E = 4 \cdot 2,5$,

$E = 10$.
 \end{exemple*1}

 \begin{exemple*1}
Effectue la multiplication : $F = 0,2 \cdot (-14)$. \\[0.5em]
Le résultat est négatif car c'est le produit d'un nombre positif par un nombre négatif :

$F = -(0,2 \cdot 14)$,

$F = -2,8$.
 \end{exemple*1}

 \exercice  
Effectue les multiplications suivantes :
\begin{colenumerate}{3}
 \item $(-7) \cdot (-8)$ ;
 \item $-5 \cdot (-11)$ ;
 \item $(-9) \cdot 6$ ;
 \item $-8 \cdot 0,5$ ;
 \item $10 \cdot (-0,8)$ ;
 \item $(-7) \cdot 0$.
 \end{colenumerate}
%\correction

 \end{methode*1}
 
 %%%%%%%%%%%%%%%%%%%%%%%%%%%%%%%%%%%%%%%%%%%%%%%%%%%%%%%%%%%%%%%%%%%%%%%%
 
 \begin{aconnaitre}
 \begin{itemize}
  \item Le produit de plusieurs nombres relatifs est \textbf{positif} s'il comporte un nombre \textbf{pair} de \textbf{facteurs négatifs} ;
  \item Le produit de plusieurs nombres relatifs est \textbf{négatif} s'il comporte un nombre \textbf{impair} de \textbf{facteurs négatifs}.
  \end{itemize}
\end{aconnaitre}

\begin{methode*1}[Multiplier plusieurs nombres relatifs]

 \begin{exemple*1}
Quel est le signe du produit : $A = -6 \cdot 7 \cdot (-8) \cdot (-9)$ ? \\[0.5em]
Le produit comporte trois facteurs négatifs. Or 3 est impair donc $A$ est négatif.
 \end{exemple*1}
 
  \begin{exemple*1}
Calcule le produit : $B = 2 \cdot (-4) \cdot (-5) \cdot (-2,5) \cdot (-0,8)$. \\[0.5em]
Le produit comporte quatre facteurs négatifs. Or 4 est pair donc $B$ est positif :

$B = 2 \cdot 4 \cdot 5 \cdot 2,5 \cdot 0,8$,
       
$B = (2 \cdot 5) \cdot (4 \cdot 2,5) \cdot 0,8$,
       
$B = 10 \cdot 10 \cdot 0,8$,
       
$B = 80$.
 \end{exemple*1}
 
 \exercice  
Quel est le signe du produit $C = 9 \cdot (-9) \cdot (-9) \cdot 9 \cdot (-9) \cdot (-9) \cdot (-9)$ ?
%\correction
     
 \exercice  
Calcule :
\begin{colenumerate}{2}
 \item $-25 \cdot (-9) \cdot (-4)$ ;
 \item $0,5 \cdot 6 \cdot (-20) \cdot 8$.
 \end{colenumerate}
%\correction

 \end{methode*1}
 
 %%%%%%%%%%%%%%%%%%%%%%%%%%%%%%%%%%%%%%%%%%%%%%%%%%%%%%%%%%%%%%%%%%%%%%%%
 
 \begin{aconnaitre}
Pour diviser deux nombres relatifs non nuls, on divise les valeurs absolues et on applique la \MotDefinition{règle des signes}{} :
\begin{itemize}
 \item Le quotient de deux nombres relatifs de \textbf{même signe} est \textbf{positif} ;
 \item Le quotient de deux nombres relatifs de \textbf{signes opposés} est \textbf{négatif}.
 \end{itemize}
\end{aconnaitre}

\begin{methode*1}[Diviser deux nombres relatifs]

 \begin{exemple*1}
Effectue la division suivante : $A = 65 : (-5)$. \\[0.5em]
Le résultat est négatif car c'est le quotient de deux nombres de signes opposés :

$65 : 5 = 13$ donc $A = -13$.
 \end{exemple*1}
 
 
 \begin{exemple*1}
Effectue la division: $B = (-30) : (-4)$. \\[0.5em]
Le résultat est positif car c'est le quotient de deux nombres négatifs :

$B = 30 : 4$,

$B = 7,5$.
 \end{exemple*1}
 
 \exercice  
Quel est le signe des quotients suivants ?
\begin{colenumerate}{4}
 \item $56 : (-74)$ ;
 \item $(-6) : (-5)$ ;
 \item $9 : (-13)$ ;
 \item $-7 : (-45)$.
 \end{colenumerate}
%\correction

 \exercice  
Calcule de tête :
\begin{colenumerate}{4}
 \item $45 : (-5)$;
 \item $(-56) : (- 8)$ ;
 \item $-59 : (-10)$ ;
 \item $-14 : 4$.
 \end{colenumerate}
%\correction

 \end{methode*1}
 
 %%%%%%%%%%%%%%%%%%%%%%%%%%%%%%%%%%%%%%%%%%%%%%%%%%%%%%%%%%%%%%%%%%%%%%%%