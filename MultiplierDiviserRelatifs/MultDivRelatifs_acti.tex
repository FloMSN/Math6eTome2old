\begin{activite}[Produit d'un nombre négatif par un nombre positif] \label{MultDivRelatifs_acti1}

On considère l'expression $A = (- 2) + (- 2) + (- 2) + (- 2)$.

\begin{partie}
Quelle est la valeur de $A$ ? \\[0.5em]
On va revenir sur le sens de la multiplication : $20 + 20 + 20$ est la somme de trois termes tous égaux. On peut donc écrire cette somme sous la forme du produit $20 \cdot 3$ qui se lit « 20 multiplié par 3 ».
\end{partie}

\begin{partie}
Écris $A$ sous la forme d'un produit.
 \end{partie}
 
\begin{partie} 
Écris les expressions suivantes sous la forme d'une somme et calcule-les :
 \begin{colenumerate}{4}
  \item $(- 6) \cdot 3$ ;
  \item $(- 22) \cdot 5$ ;
  \item $(- 7) \cdot 7$ ;
  \item $(- 1,5) \cdot 6$.
  \end{colenumerate}
 \end{partie}
 
\begin{partie} \label{MultDivRelatifs_acti2}
Trouve une règle permettant de calculer le produit d'un nombre négatif par un nombre positif.
 \end{partie}

\end{activite}

%%%%%%%%%%%%%%%%%%%%%%%%%%%%%%%%%%%%%%%%%%%%%%%%%%%%%%%%%%%%%%%%%%%%%%%%%

\begin{activite}[À propos des produits]

\begin{minipage}[c]{0.48\linewidth}
\begin{partie}
Voici une table de multiplication :
\begin{enumerate}
 \item Recopie-la sur ton cahier et complète la partie qui concerne le produit de deux nombres positifs (en bas à droite).
 \item D'après le résultat de la partie \ref{MultDivRelatifs_acti2} de l'activité \ref{MultDivRelatifs_acti1}, complète la partie qui concerne le produit d'un nombre négatif par un nombre positif (en haut à droite).
 \item Observe les résultats dans cette table de multiplication et complète-la entièrement, en expliquant tes choix.
 \item À l'aide d'un tableur, crée cette table de multiplication et vérifie que les résultats obtenus sont les mêmes que les tiens.
 \end{enumerate}
\end{partie}
\end{minipage} \hfill%
 \begin{minipage}[c]{0.48\linewidth}
  \includegraphics[width=7.6cm]{grille_produits}
  \end{minipage} \\

\begin{partie}
Application sur quelques exemples :
\begin{enumerate}
 \item En t'aidant de la table, donne le résultat pour chaque calcul suivant :
 \begin{colitemize}{4}
  \item $A = (- 5) \cdot 4$ ;
  \item $B = 3 \cdot (- 2)$ ;
  \item $C = 5 \cdot (- 4)$ ;
  \item $D = (- 1) \cdot (- 3)$.
  \end{colitemize}
 \item En t'inspirant de ce qui précède, propose un résultat pour les calculs suivants :
 \begin{colitemize}{4}
  \item $E = (- 9,2) \cdot 2$ ;
  \item $F = 1,5 \cdot (- 8)$ ;
  \item $G = (- 3,14) \cdot 0$ ;
  \item $H = (- 1,2) \cdot (- 0,1)$.
  \end{colitemize}
 \item Vérifie ces résultats à la calculatrice.
 \end{enumerate}
\end{partie}

\begin{partie}
Propose une règle qui permet, dans tous les cas, de calculer le produit de deux nombres relatifs.
\end{partie}

\end{activite}

%%%%%%%%%%%%%%%%%%%%%%%%%%%%%%%%%%%%%%%%%%%%%%%%%%%%%%%%%%%%%%%%%%%%%%%%%

\begin{activite}[Produit de plusieurs nombres relatifs]

\begin{partie}
Calcule ces expressions et déduis-en une règle pour trouver rapidement chaque résultat :
\begin{itemize}
 \item $A = (- 1) \cdot (- 1)$ ;
 \item $B = (- 1) \cdot (- 1) \cdot (- 1)$ ;
 \item $C = (- 1) \cdot (- 1) \cdot (- 1) \cdot (- 1)$ ;
 \item $D = (- 1) \cdot (- 1) \cdot (- 1) \cdot (- 1) \cdot (- 1)$ ;
 \item $E = (- 1) \cdot (- 1) \cdot (- 1) \cdot (- 1) \cdot (- 1) \cdot (- 1)$.
 \end{itemize}
\end{partie}

\begin{partie}
On sait que $(- 4) = (- 1) \cdot 4$ et $(- 2) = (- 1) \cdot 2$.
\begin{enumerate}
 \item Complète alors le calcul suivant :
\begin{center} $(- 4) \cdot (- 2) \cdot (- 5) = (- 1) \cdot \ldots \cdot (- 1) \cdot \ldots \cdot (- 1) \cdot \ldots$ \end{center}
\begin{center} $\phantom{(- 4) \cdot (- 2) \cdot (- 5) }= (- 1) \cdot (- 1) \cdot (- 1) \cdot \ldots \cdot \ldots \cdot \ldots$ \end{center}
 \item Déduis-en une méthode pour trouver le résultat de $(- 4) \cdot (- 2) \cdot (- 5)$.
 \end{enumerate}
\end{partie}

\begin{partie}
Inspire-toi de la question précédente pour effectuer le calcul suivant :
\begin{itemize}
 \item $F = (- 2) \cdot (- 3) \cdot 5 \cdot (- 4) \cdot 6 \cdot (- 5)$.
 \end{itemize}
\end{partie}

\begin{partie}
Propose une méthode pour multiplier plusieurs nombres relatifs.
\end{partie}

\end{activite}

%%%%%%%%%%%%%%%%%%%%%%%%%%%%%%%%%%%%%%%%%%%%%%%%%%%%%%%%%%%%%%%%%%%%%%%%%

\begin{activite}[Quotient de nombres relatifs]

Revenons sur le sens de la division : \\[0.5em]
Écrire $3 \cdot 5 = 15$ revient à écrire $3 = 15 : 5$ ou $5 = 15 : 3$.
       
\begin{partie}
Recopie et complète les trous par les nombres manquants pour que les égalités soient correctes :
\begin{colenumerate}{4}
 \item $4 \cdot  \ldots = 12$ ;
 \item $(- 5) \cdot  \ldots = 130$ ;
 \item $8 \cdot  \ldots = (- 16)$ ;
 \item $ \ldots \cdot (- 3) = (- 27)$.
 \end{colenumerate}
\end{partie}

\begin{partie}
Écris ces nombres manquants sous forme de quotients.
\end{partie}

\begin{partie}
Que dire du quotient de deux nombres relatifs ?
\end{partie}

\end{activite}

%%%%%%%%%%%%%%%%%%%%%%%%%%%%%%%%%%%%%%%%%%%%%%%%%%%%%%%%%%%%%%%%%%%%%%%%%
