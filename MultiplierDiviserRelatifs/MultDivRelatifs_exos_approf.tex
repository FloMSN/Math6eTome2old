\begin{exercice}[Températures]
Pour mesurer la température, il existe plusieurs unités. Celle que nous utilisons en Suisse est le degré Celsius ($^\circ$C). Cette unité est faite de façon à ce que la température à laquelle l'eau se transforme en glace est $0^\circ$C et celle à laquelle l'eau se transforme en vapeur est $100^\circ$C. Dans cette échelle, il existe des températures négatives. \\[0.5em]
Il existe une autre unité, le Kelvin (K), dans laquelle les températures négatives n'existent pas. Pour passer de l'une à l'autre, on utilise la formule :

\begin{center} $T_{Kelvin} = \emph{$\textrm{T}_\textrm{degrésCelsius}$} + 273,15$ \end{center}
       
Ainsi, $10^\circ$C correspondent à 283,15 K.
\begin{enumerate}
 \item Convertis en Kelvin les températures suivantes : $24^\circ$C ; $- 3^\circ$C et $- 22,7^\circ$C.
 \item Convertis en degré Celsius les températures suivantes : 127,7 K ; 276,83 K ; 204 K et 500 K.
 \item Quelle est en Kelvin la plus petite température possible ? À quelle température en degré Celsius correspond-elle ? Cette température est appelée le zéro absolu.
 \end{enumerate}
\end{exercice}


\begin{exercice}[Sur un axe gradué]
\begin{enumerate}
 \item Soit $A$ le point d'abscisse 4. Quelle peut-être l'abscisse du point $B$ sachant que la longueur du segment $[AB] = 8$ ?
 \item Soit $C$ le point d'abscisse $- 3$. Quelle peut-être l'abscisse du point $D$ sachant que la longueur du segment $[CD] = 2$ ?
 \item Soit $E$ le point d'abscisse $- 5$. Détermine l'abscisse de $F$ sachant que la longueur du segment $[EF] = 9$ et que l'abscisse de $F$ est inférieure à celle de $E$.
 \end{enumerate}
\end{exercice}


\begin{exercice}[Signes mystères]
Recopie en remplaçant les $§$ par le signe $-$ ou le signe $+$ de sorte que les égalités soient vraies :
\begin{enumerate}
 \item $\text{§} \, 7 \, \text{§} \, 3 = - 4$ ;
 \item $\text{§} \, 13 \, \text{§} \, 8 = - 21$ ;
 \item $\text{§} \, 3,7 \, \text{§} \, 8,4 = 4,7$ ;
 \item $\text{§} \, 45 \, \text{§} \, 72 = - 27$ ;
 \item $\text{§} \, 2 \, \text{§} \, 7 \, \text{§} \, 13 = - 8$ ;
 \item $\text{§} \, 1,5 \, \text{§} \, 2,3 \, \text{§} \, 4,9 = - 5,7$ ;
 \item $\text{§} \, 8 \, \text{§} \, 5 \, \text{§} \, 12 \, \text{§} \, 2 = 13$ ;
 \item $\text{§} \, 7 \, \text{§} \, 14 \, \text{§} \, 18 \, \text{§} \, 3 = - 22$.
 \end{enumerate}
\end{exercice}


\begin{exercice}[Carré magique]
Recopie et complète ce carré magique sachant qu'il contient tous les entiers de $- 12$ à 12 et que les sommes des nombres de chaque ligne, de chaque colonne et de chaque diagonale sont toutes nulles :
\begin{center}
\begin{tabular}{|*5{@{}>{\vrule width0pt height\dimexpr1cm/2+1ex-.2pt\relax depth\dimexpr1cm/2-1ex-.2pt\relax\centering\arraybackslash}p{\dimexpr1cm-.4pt\relax}@{}|}}\hline
    & & 0 & 8 & \\\hline
    & & & $- 11$ & 2 \\\hline
   $- 9$ & $- 1$ & 12 & & 3 \\\hline
   $- 3$ & & $- 12$ & & 9 \\\hline
   $- 2$ & 11 & $- 6$ & 7 & \\\hline
\end{tabular}
 \end{center}
\end{exercice}


\begin{exercice}[Triangle magique]
La somme des nombres de chaque côté du triangle est 2. Remplis les cases vides avec les nombres relatifs $(- 2)$ ; $(- 1)$ ; 1 ; 2 et 3, qui doivent tous être utilisés.
\begin{center} \includegraphics[width=3.5cm]{triangle_magique} \end{center}
\end{exercice}


\begin{exercice}[Coup de froid]
Chaque matin de la 1\up{re} semaine du mois de Février, Julie a relevé la température extérieure puis a construit le tableau suivant :
\begin{center}
\begin{tabularx}{1.07\linewidth}{|c|*{7}{>{\centering\arraybackslash}X|}} 
\hline
\cellcolor{H2} Jour & \cellcolor{H3} Lu & \cellcolor{H3} Ma & \cellcolor{H3} Me & \cellcolor{H3} Je & \cellcolor{H3} Ve & \cellcolor{H3} Sa & \cellcolor{H3} Di \\\hline
\cellcolor{A2} \small{Température (en $^\circ$C)} & \cellcolor{A3} $- 4$ & \cellcolor{A3} $- 2$ & \cellcolor{A3} $- 1$ & \cellcolor{A3} $+ 1$ & \cellcolor{A3} 0 & \cellcolor{A3} $+ 2$ & \cellcolor{A3} $- 3$ \\\hline
 \end{tabularx}
\end{center}
Calcule la moyenne des températures relevées par Julie.
\end{exercice}


\begin{exercice}
Recopie et complète les carrés magiques suivants :
\begin{enumerate}
 \item Pour l'addition :
 \begin{center}
\begin{tabular}{|*3{@{}>{\vrule width0pt height\dimexpr1cm/2+1ex-.2pt\relax depth\dimexpr1cm/2-1ex-.2pt\relax\centering\arraybackslash}p{\dimexpr1cm-.4pt\relax}@{}|}}\hline
   \cellcolor{U1} & \cellcolor{U2} $- 9$ & $- 2$ \cellcolor{U1} \\\hline
   \cellcolor{U2} & \cellcolor{U1} $- 4$ & \cellcolor{U2}  \\\hline
   \cellcolor{U1} $- 6$ & \cellcolor{U2} & \cellcolor{U1} \\\hline
\end{tabular}
 \end{center}
 \item Pour l'addition :
  \begin{center}
\begin{tabular}{|*3{@{}>{\vrule width0pt height\dimexpr1.4cm/2+1ex-.2pt\relax depth\dimexpr1.4cm/2-1ex-.2pt\relax\centering\arraybackslash}p{\dimexpr1.4cm-.4pt\relax}@{}|}}\hline
\cellcolor{F2} 1,6 & \cellcolor{F3}  & \cellcolor{F2}  \\
\cellcolor{F3} & \cellcolor{F2} $- 5,4$ & \cellcolor{F3} \\
\cellcolor{F2} $- 4,4$ & \cellcolor{F3} & \cellcolor{F2} $- 12,4$\\
\end{tabular}
\end{center}
 \item Pour la multiplication :
  \begin{center}
\begin{tabular}{|*3{@{}>{\vrule width0pt height\dimexpr1cm/2+1ex-.2pt\relax depth\dimexpr1cm/2-1ex-.2pt\relax\centering\arraybackslash}p{\dimexpr1cm-.4pt\relax}@{}|}}\hline
   \cellcolor{H1} & \cellcolor{H2} 36 & $- 3$ \cellcolor{H1} \\\hline
   \cellcolor{H2} & \cellcolor{H1} 6 & \cellcolor{H2}  \\\hline
   \cellcolor{H1} $- 12$ & \cellcolor{H2} & \cellcolor{H1} \\\hline
\end{tabular}
 \end{center}
 \end{enumerate}
\end{exercice}


\begin{exercice}
La différence $a - b$ est égale à 12.

On augmente $a$ de 3 et on diminue $b$ de 4.

Combien vaut la différence entre ces deux nouveaux nombres? 
\end{exercice}


\begin{exercice}[Le nombre $- 21$ \ldots]
\begin{enumerate}
 \item Écris le nombre $- 21$ comme somme de deux nombres entiers relatifs consécutifs ;
 \item Écris le nombre $- 21$ comme différence de deux carrés.
 \end{enumerate}
\end{exercice}


\begin{exercice}
Recopie et complète les phrases suivantes :
\begin{enumerate}
 \item $- 21$ est la moitié de \ldots \ldots ;
 \item $- 21$ est le triple de \ldots \ldots ;
 \item $- 21$ est l'opposé de \ldots \ldots.
 \end{enumerate}
\end{exercice}


\begin{exercice}[Choisir deux nombres]
\begin{enumerate}
 \item Trouve deux nombres relatifs dont le produit est positif et la somme est négative ;
 \item Trouve deux nombres relatifs dont le produit est négatif et la somme est positive ;
 \item Trouve deux nombres relatifs dont le produit et la somme sont positifs ;
 \item Trouve deux nombres relatifs dont le produit et la somme sont négatifs.
 \end{enumerate}
\end{exercice}


\begin{exercice}[Énigme]
Sachant que le produit deux nombres $A$ et $B$ est positif et que leur somme est négative, quels sont les signes de $A$ et de $B$ ?
\end{exercice}


\begin{exercice}[Calculatrice]
Effectue à la calculatrice les calculs suivants :
\begin{colenumerate}{2}
 \item $13\,857 \cdot (- 253)$ ;
 \item $\dfrac{- 44\,980}{8\,996 - 10\,380}$ ;
 \item $312 - 123 \cdot (- 734)$ ;
 \item $\dfrac{- 34 \cdot (- 713)}{- 68}$.
 \end{colenumerate}
\end{exercice}


\begin{exercice}[Signe]
$A$ est le produit de 24 nombres (non nuls) comportant 23 facteurs négatifs.\\[0.5em]
$B$ est le produit de 13 nombres (non nuls) comportant 11 facteurs négatifs.\\[0.5em] 
Donne, si c'est possible, le signe de :
\begin{colenumerate}{3}
 \item $A \cdot B$ ;
 \item $A : B$ ;
 \item $A - B$ ; 
 \item $A^{2}$ ;
 \item $A + B$.
 \end{colenumerate}           
\end{exercice}




