\begin{activite}[Il faut régler l'addition !]

À la fête foraine, Mamadou a choisi un jeu comportant deux manches à l'issue desquelles il peut gagner ou perdre de l'argent. Un gain de 3 CHF est noté $+ 3$ ou 3 tandis qu'une perte de 7 CHF est notée $- 7$.
\begin{partie}
Donne le bilan de chacune des parties suivantes :
\begin{itemize}
\item Partie 1 : Mamadou a gagné 3 CHF puis a gagné 7 CHF.
\item Partie 2 : Mamadou a gagné 8 CHF puis a perdu 5 CHF.
\item Partie 3 : Mamadou a perdu 4 CHF puis a perdu 6 CHF.
\item Partie 4 : Mamadou a perdu 9 CHF puis a gagné 2 CHF.
\end{itemize}
\end{partie}

\begin{partie}[Dans un tableau]
\begin{enumerate}
 \item Recopie le tableau ci-dessous qui représente les gains et les pertes des deux manches de plusieurs parties :
 
 \begin{center}
 \begin{tabularx}{0.95\linewidth}{|c|*{6}{>{\centering\arraybackslash}X|}}
\hline
\rowcolor{Gris2} & A & B & C & D \\ \hline
\cellcolor{Gris2} 1 & \cellcolor{J2} Partie n\up{$\circ$} & \cellcolor{J2} 1\up{ère} manche & \cellcolor{J2} 2\up{ème} manche & \cellcolor{J2} Bilan de la partie \\ \hline
\cellcolor{Gris2} 2 & \cellcolor{J2}1 & \cellcolor{J2} $+ 3$ & \cellcolor{J2} $+ 7$ & \cellcolor{H1} \\ \hline
\cellcolor{Gris2} 3 & \cellcolor{J2} 2 & \cellcolor{J2} $+ 8$ & \cellcolor{J2} $- 5$ & \cellcolor{J2} \\ \hline
\cellcolor{Gris2} 4 & \cellcolor{J2} 3 & \cellcolor{J2} $- 4$ & \cellcolor{J2} $- 6$ & \cellcolor{J2} \\ \hline
\cellcolor{Gris2} 5 & \cellcolor{J2} 4 & \cellcolor{J2} $- 9$ & \cellcolor{J2} $+ 2$ & \cellcolor{J2} \\ \hline
\cellcolor{Gris2} 6 & \cellcolor{J2} 5 & \cellcolor{J2} $- 7$ & \cellcolor{J2} $+ 10$ & \cellcolor{J2} \\ \hline
\cellcolor{Gris2} 7 & \cellcolor{J2} 6 & \cellcolor{J2} $- 3$ & \cellcolor{J2} $- 9$ & \cellcolor{J2} \\ \hline
\cellcolor{Gris2} 8 & \cellcolor{J2} 7 & \cellcolor{J2} $+ 8$ & \cellcolor{J2} $+ 2$ & \cellcolor{J2} \\ \hline
\cellcolor{Gris2} 9 & \cellcolor{J2} 8 & \cellcolor{J2} $+ 4$ & \cellcolor{J2} $- 2$ & \cellcolor{J2} \\ \hline
\cellcolor{Gris2} 10 & \cellcolor{J2} 9 & \cellcolor{J2} $+ 5$ & \cellcolor{J2} $- 7$ & \cellcolor{J2} \\ \hline
\cellcolor{Gris2} 11 & \cellcolor{J2} 10 & \cellcolor{J2} $+ 10$ & \cellcolor{J2} $+ 12$ & \cellcolor{J2} \\ \hline
\end{tabularx}
 \end{center}

 \item Effectue les calculs des cases $D2$ à $D11$.
 \end{enumerate}
\end{partie}

\begin{partie}[Addition de deux nombres relatifs]
\begin{enumerate}
 \item Sur le tableau, colorie en rouge les parties où Mamadou a gagné ou perdu de l'argent à chacune des deux manches :
 \item Pour chaque cas, quelle opération fais-tu pour trouver la valeur absolue du bilan ? 
 \item Dans quels cas le bilan est-il positif ? négatif ?
 \item Déduis-en une règle pour additionner deux nombres relatifs de même signe.
 \item Que représentent les cas qui ne sont pas coloriés en rouge ? Dans ces cas :
 \item Quelle opération fais-tu pour trouver la valeur absolue du bilan ? 
 \item Comment détermines-tu le signe du bilan ? 
 \item Déduis-en une règle pour additionner deux nombres relatifs de signes différents. 
 \end{enumerate}
\end{partie}

\begin{partie}
Recopie et complète :
\begin{colenumerate}{3}
 \item $(+ 8) + (+ 2) = \ldots$ ;
 \item $(- 7) + (+ 5) = \ldots$ ;
 \item $(- 4) + (- 6) = \ldots$ ;
 \item $(- 4) + (+ 7) = \ldots$ ;
 \item $(- 5) + (- 9) = \ldots$ ;
 \item $(+ 1) + (- 4) = \ldots$.
 \end{colenumerate}
\end{partie}

\end{activite}

%%%%%%%%%%%%%%%%%%%%%%%%%%%%%%%%%%%%%%%%%%%%%%%%%%%%%%%%%%%%%%%%%%%%%%

\begin{activite}[Quelles différences \ldots]

Voici un tableau qui donne les températures en degrés Celsius durant une semaine à Caprino lors d'un hiver très rigoureux :

 \begin{center}
 \begin{tabularx}{1.05\linewidth}{|c|*{8}{>{\centering\arraybackslash}X|}}
\hline
\rowcolor{J2} Jour & lundi & mardi & mercredi & jeudi & vendredi & samedi & dimanche \\ \hline
\rowcolor{J2} Température & $+ 2$ & $+ 6$ & $+ 3$ & $- 5$ & $- 7$ & $- 3$ & $+ 1$ \\ \hline
\end{tabularx}
 \end{center}
 \vspace{-0.12cm}
 \qquad \qquad \begin{tabularx}{0.9\linewidth}{|c|*{7}{>{\centering\arraybackslash}X|}}
\hline
\rowcolor{J2} Variation & $+ 4$ & $- 3$ & & & & \\ \hline
\end{tabularx} \\[0.5em]
La variation indique la différence de température remarquée entre deux jours consécutifs.

\begin{partie}
Reproduis et complète ce tableau.

La différence de température entre le lundi et le mardi est de $+ 4^{\circ}$C. On peut écrire : $(+ 6) - (+ 2) = (+ 4)$.
\end{partie}

\begin{partie} \label{OpererRelatifs_actiA}
En utilisant les réponses du tableau précédent, complète de la même manière les différences suivantes :
\begin{colenumerate}{3}
 \item $(+ 6) - (+ 2) = (+ 4)$ ;
 \item $(+ 3) - (+ 6) = \ldots$ ;
 \item $(- 5) - (+ 3) = \ldots$ ;
 \item $(- 7) - (- 5) = \ldots$ ;
 \item $(- 3) - (- 7) = \ldots$ ;
 \item $(+ 1) - (- 3) = \ldots$ .
 \end{colenumerate}
\end{partie}

\begin{partie}
Calcule les sommes suivantes : \label{OpererRelatifs_actiB}
\begin{colenumerate}{3}
 \item $(+ 6) + (- 2) = (+ 4)$ ;
 \item $(+ 3) + (- 6) = \ldots$ ;
 \item $(- 5) + (- 3) = \ldots$ ;
 \item $(- 7) + (+ 5) = \ldots$ ;
 \item $(- 3) + (+ 7) = \ldots$ ;
 \item $(+ 1) + (+ 3) = \ldots$ .
 \end{colenumerate}
\end{partie}

\begin{partie}
Compare les calculs et les résultats des parties \ref{OpererRelatifs_actiA} et \ref{OpererRelatifs_actiB}. Que remarques-tu ?

Recopie et complète la phrase : « Soustraire un nombre relatif revient à \ldots son \ldots . ».
\end{partie}

\begin{partie}
Effectue les soustractions suivantes en transformant d'abord chaque soustraction en addition :
\begin{colenumerate}{3}
 \item $(+ 7) - (+ 11)$ ;
 \item $(+ 29) - (- 15)$ ;
 \item $(- 73) - (- 52)$.
 \end{colenumerate}
\end{partie}

\end{activite}


%%%%%%%%%%%%%%%%%%%%%%%%%%%%%%%%%%%%%%%%%%%%%%%%%%%%%%%%%%%%%%%%%%%%%

\begin{activite}[Pour tout simplifier]

\begin{partie}[Simplification, 1\up{er} acte]
\begin{enumerate}
 \item Effectue les calculs $(+ 6) + (- 4)$ et $6 - 4$. Que remarques-tu ?
 \item Simplifie de même l'écriture de $(+ 7) + (- 1)$ puis effectue le calcul.
 \end{enumerate}
\end{partie}

\begin{partie}[Simplification, 2\up{ème} acte]
\begin{enumerate}
 \item Effectue les calculs $(+ 7) - (+ 5)$ et $7 -  5$. Que remarques-tu ?
 \item Simplifie de même l'écriture de $(+ 12) - (+ 7)$ puis calcule.
 \end{enumerate}
\end{partie}

\begin{partie}[Simplification, 3\up{ème} acte]
\begin{enumerate}
 \item Effectue $(- 10) + (+ 1)$.
 \item Pour soustraire 9 à un nombre, il est souvent plus rapide de soustraire 10 puis d'ajouter 1, ce qu'on peut noter : $- 10 + 1 = \ldots$ . Qu'en déduis-tu ?
 \end{enumerate}
\end{partie}

\begin{partie}[Simplification, dernier acte]
\begin{enumerate}
 \item Effectue les calculs $(- 9) - (- 2)$ et $- 9 + 2$. Que remarques-tu ?
 \item Simplifie alors l'écriture de $(+ 8) - (- 7)$ puis calcule.
 \end{enumerate}
\end{partie}

\begin{partie}
En observant bien les questions précédentes, essaie de supprimer les parenthèses et les signes inutiles dans l'expression :  $A = (- 5) + (- 9) - (+ 3)$ puis effectue le calcul.
\end{partie}

\end{activite}