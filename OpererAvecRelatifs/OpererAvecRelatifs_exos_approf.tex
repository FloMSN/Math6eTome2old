\begin{exercice}[Sur un axe gradué]
\begin{enumerate}
 \item Soit $A$ le point d'abscisse 4. Quelle peut-être l'abscisse du point $B$ sachant que la longueur du segment $[AB] = 8$ ?
 \item Soit $C$ le point d'abscisse $- 3$. Quelle peut-être l'abscisse du point $D$ sachant que la longueur du segment $[CD] = 2$ ?
 \item Soit $E$ le point d'abscisse $- 5$. Détermine l'abscisse de $F$ sachant que la longueur du segment $[EF] = 9$ et que l'abscisse de $F$ est inférieure à celle de $E$.
 \end{enumerate}
\end{exercice}


\begin{exercice}
Recopie en remplaçant les $\S$ par le signe $-$ ou le signe $+$ de sorte que les égalités soient vraies :
\begin{enumerate}
 \item $\S \quad 7 \quad \S \quad 3 = - 4$ ;
 \item $\S \quad 13 \quad \S \quad 8 = - 21$ ;
 \item $\S \quad 3,7 \quad \S \quad 8,4 = 4,7$ ;
 \item $\S \quad 45 \quad \S \quad 72 = - 27$ ;
 \item $\S \quad 2 \quad \S \quad 7 \quad \S \quad 13 = - 8$ ;
 \item $\S \quad 1,5 \quad \S \quad 2,3 \quad \S \quad 4,9 = - 5,7$ ;
 \item $\S \quad 8 \quad \S \quad 5 \quad \S \quad 12 \quad \S \quad 2 = 13$ ;
 \item $\S \quad 7 \quad \S \quad 14 \quad \S \quad 18 \quad \S \quad 3 = - 22$.
 \end{enumerate}
\end{exercice}


\begin{exercice}[Carré magique]
Recopie et complète ce carré magique sachant qu'il contient tous les entiers de $- 12$ à $12$ et que les sommes des nombres de chaque ligne, de chaque colonne et de chaque diagonale sont toutes nulles :
\begin{center}
\begin{tabular}{|*5{@{}>{\vrule width0pt height\dimexpr1cm/2+1ex-.2pt\relax depth\dimexpr1cm/2-1ex-.2pt\relax\centering\arraybackslash}p{\dimexpr1cm-.4pt\relax}@{}|}}\hline
    & & 0 & 8 & \\\hline
    & & & $- 11$ & 2 \\\hline
   $- 9$ & $- 1$ & 12 & & 3 \\\hline
   $- 3$ & & $- 12$ & & 9 \\\hline
   $- 2$ & 11 & $- 6$ & 7 & \\\hline
\end{tabular}
 \end{center}
\end{exercice}


\begin{exercice}[Triangle magique]
La somme des nombres de chaque côté du triangle est 2. Remplis les cases vides avec les nombres relatifs $(- 2)$ ; $(- 1)$ ; 1 ; 2 et 3, qui doivent tous être utilisés.
\end{exercice}


\begin{exercice}[Coup de froid]
Chaque matin de la 1\up{re} semaine du mois de Février, Julie a relevé la température extérieure puis a construit le tableau suivant :
\begin{center}
\begin{tabularx}{1.03\linewidth}{|c|*{8}{>{\centering \arraybackslash}X|}}
\hline \cellcolor{H1} Jour & \cellcolor{H2} \small{Lu} & \cellcolor{H2} \small{Ma} & \cellcolor{H2} \small{Me} & \cellcolor{H2} \small{Je} & \cellcolor{H2} \small{Ve} & \cellcolor{H2} \small{Sa} & \cellcolor{H2} \small{Di} \\
\hline \cellcolor{U1} \small{Température (en $^{\circ}$C)} & \cellcolor{U2} \small{$- 4$} & \cellcolor{U2} \small{$- 2$} & \cellcolor{U2} \small{$- 1$} & \cellcolor{U2} \small{$+ 1$} & \cellcolor{U2} \small{0} & \cellcolor{U2} \small{$+ 2$} & \cellcolor{U2} \small{$- 3$} \\
\hline
\end{tabularx} \\
\end{center}
Calcule la moyenne des températures relevées par Julie.
\end{exercice}


\begin{exercice}
Recopie et complète les carrés magiques suivants :
\begin{enumerate}
 \item Pour l'addition : \\[0.5em]
\begin{center}
\begin{tabular}{|*3{@{}>{\vrule width0pt height\dimexpr1cm/2+1ex-.2pt\relax depth\dimexpr1cm/2-1ex-.2pt\relax\centering\arraybackslash}p{\dimexpr1cm-.4pt\relax}@{}|}}\hline
\cellcolor{U1} & \cellcolor{U2} $- 9$ & \cellcolor{U1} $- 2$ \\
\cellcolor{U2} & \cellcolor{U1} $- 4$ & \cellcolor{U2} \\
\cellcolor{U1} $- 6$ & \cellcolor{U2} & \cellcolor{U1} \\
\end{tabular}
\end{center}
\vspace{0.5cm}
 \item Pour l'addition : \\[0.5em]
 \begin{center}
\begin{tabular}{|*3{@{}>{\vrule width0pt height\dimexpr1.4cm/2+1ex-.2pt\relax depth\dimexpr1.4cm/2-1ex-.2pt\relax\centering\arraybackslash}p{\dimexpr1.4cm-.4pt\relax}@{}|}}\hline
\cellcolor{F2} 1,6 & \cellcolor{F3}  & \cellcolor{F2}  \\
\cellcolor{F3} & \cellcolor{F2} $- 5,4$ & \cellcolor{F3} \\
\cellcolor{F2} $- 4,4$ & \cellcolor{F3} & \cellcolor{F2} $- 12,4$\\
\end{tabular}
\end{center}
 \end{enumerate}
\end{exercice}


\begin{exercice}
La différence a $- b$ est égale à 12.

On augmente $a$ de 3 et on diminue $b$ de 4.

Combien vaut la différence entre ces deux nouveaux nombres? 
\end{exercice}


\begin{exercice}[Le nombre $- 21$ \ldots]
\begin{enumerate}
 \item Écris le nombre $- 21$ comme somme de deux nombres entiers relatifs consécutifs ;
 \item Écris le nombre $- 21$ comme différence de deux carrés.
 \end{enumerate}
\end{exercice}


