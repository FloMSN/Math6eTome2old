
\serie{Sommes de relatifs}

\begin{exercice}
Recopie dans ton cahier, effectue les additions puis relie chaque calcul à son résultat :
\begin{center}
 \begin{tabularx}{0.95\linewidth}{|cc|X|cc|}
  \cline{1-2}\cline{4-5}
  $(- 12) + (- 4)$ & $\cdot$ & & $\cdot$ & $+ 4$ \\ \cline{1-2}\cline{4-5}
  $(+ 12) + (- 4)$ & $\cdot$ & & $\cdot$ & $- 20$ \\ \cline{1-2}\cline{4-5}
  $(- 12) + (- 8)$ & $\cdot$ & & $\cdot$ & $- 16$ \\ \cline{1-2}\cline{4-5}
  $(- 8) + (+ 12)$ & $\cdot$ & & $\cdot$ & $+ 12$ \\ \cline{1-2}\cline{4-5}
  $(+ 8) + (+ 4)$ & $\cdot$ & & $\cdot$ & $+ 8$ \\ \cline{1-2}\cline{4-5}
  \end{tabularx}
  \end{center}
\end{exercice}

\begin{exercice}
Effectue les additions suivantes :
\begin{colenumerate}{2}
 \item $(+ 2) + (+ 7)$ ;
 \item $(- 4) + (+ 5)$ ;
 \item $(- 8) + (- 14)$ ;
 \item $(+ 9) + (- 9)$ ;
 \item $(- 20) + (- 12)$ ;
 \item $(+ 40) + (- 60)$ ;
 \item $(- 36) + (+ 18)$ ;
 \item $(- 25) + (+ 0)$.
 \end{colenumerate}
\end{exercice}


\begin{exercice}
Effectue les additions suivantes :
\begin{colenumerate}{2}
 \item $(- 8) + (- 16)$ ;
 \item $(+ 24) + (- 4)$ ;
 \item $(- 14) + (- 3)$ ;
 \item $(- 7) + (+ 7)$ ;
 \item $(+ 14) + (+ 8)$ ;
 \item $(+ 11) + (+ 33)$ ;
 \item $(+ 30) + (- 47)$ ;
 \item $(+ 19) + (+ 1)$ ;
 \item $(- 11) + (- 13)$ ;
 \item $(+ 63) + (- 63)$.
 \end{colenumerate}
\end{exercice}


\begin{exercice}
Effectue les additions suivantes :
\begin{colenumerate}{2}
 \item $(- 2,3) + (- 4,7)$ ;
 \item $(+ 6,8) + (- 9,9)$ ;
 \item $(- 3,5) + (+ 1,8)$ ;
 \item $(- 2,51) + (- 0)$ ;
 \item $(- 7,8) + (- 2,1)$ ;
 \item $(+ 13,4) + (- 20,7)$ ;
 \item $(- 10,8) + (+ 11,2)$ ;
 \item $(+ 17) + (+ 5,47)$.
 \end{colenumerate}
\end{exercice}


\begin{exercice}[La pyramide]
Recopie puis complète les pyramides suivantes sachant que le nombre contenu dans une case est la somme des nombres contenus dans les deux cases situées en dessous de lui :

\begin{minipage}[c]{0.48\linewidth}
\begin{center} \boxed{\phantom{hello}} \end{center}
\vspace{-0.69cm}
\begin{center} \boxed{\phantom{hello}} \negthinspace \boxed{\phantom{hello}} \end{center}
\vspace{-0.71cm}
\begin{center} \boxed{\phantom{hello}} \negthinspace \boxed{\phantom{hello}} \negthinspace  \boxed{\phantom{hello}} \end{center}
\vspace{-0.71cm}
\begin{center} \negthinspace \boxed{\phantom{!}$- 21$\phantom{!}} \negthinspace \boxed{$+ 12$} \negthinspace \boxed{\phantom{!}$- 4$\phantom{!}} \negthinspace \boxed{$- 9$\phantom{.}} \end{center}
 \end{minipage} \hfill%
 \begin{minipage}[c]{0.48\linewidth}
\begin{center} \boxed{\phantom{hello}} \end{center}
\vspace{-0.69cm}
\begin{center} \boxed{\phantom{hello}} \negthinspace \boxed{\phantom{hello}} \end{center}
\vspace{-0.71cm}
\begin{center} \boxed{\phantom{hello}} \negthinspace \boxed{\phantom{hello}} \negthinspace \boxed{\phantom{hello}} \end{center}
\vspace{-0.71cm}
\begin{center} \boxed{$- 1,2$} \negthinspace \boxed{$+ 3,3$} \negthinspace \boxed{$+ 4,1$} \negthinspace \boxed{$- 9,3$} \end{center}
  \end{minipage} \\
\end{exercice}


\begin{exercice}[La pyramide (bis)]
\begin{minipage}[c]{0.48\linewidth}
\begin{center} \boxed{\phantom{hello}} \end{center}
\vspace{-0.69cm}
\begin{center} \boxed{\phantom{.}$- 14$\phantom{.}} \negthinspace \boxed{\phantom{hello}} \end{center}
\vspace{-0.71cm}
\begin{center} \boxed{\phantom{hello}} \negthinspace \boxed{\phantom{.}$+ 2$\phantom{.}} \negthinspace  \boxed{\phantom{hello}} \end{center}
\vspace{-0.71cm}
\begin{center} \negthinspace \boxed{\phantom{hello}} \negthinspace \boxed{\phantom{..}$- 7$\phantom{..}} \negthinspace \boxed{\phantom{hello}} \negthinspace \boxed{\phantom{.}$+ 3$\phantom{.}} \end{center}
 \end{minipage} \hfill%
 \begin{minipage}[c]{0.48\linewidth}
\begin{center} \boxed{$- 1,7$} \end{center}
\vspace{-0.69cm}
\begin{center} \boxed{$- 4,5$} \negthinspace \boxed{\phantom{hello}} \end{center}
\vspace{-0.71cm}
\begin{center} \boxed{\phantom{hello}} \negthinspace \boxed{$+ 2,1$} \negthinspace \boxed{\phantom{hello}} \end{center}
\vspace{-0.71cm}
\begin{center} \boxed{\phantom{hello}} \negthinspace \boxed{\phantom{hello}} \negthinspace \boxed{\phantom{hello}} \negthinspace \boxed{$+ 1,2$} \end{center}
  \end{minipage} \\
\end{exercice}


\begin{exercice}
Effectue les additions suivantes en détail :
\begin{enumerate}
 \item $(+ 3) + (- 7) + (- 8) + (+ 2)$ ;
 \item $(- 9) + (- 14) + (+ 25) + (- 3)$ ;
 \item $(- 2,3) + (- 12,7) + (+ 24,7) + (- 1,01)$ ;
 \item $(+ 7,8) + (+ 2,35) + (- 9,55) + (+ 4)$.
 \end{enumerate}
\end{exercice}


\begin{exercice}
Calcule les sommes suivantes en détail :
\begin{enumerate}
 \item $(+ 17) + (- 5) + (+ 4) + (+ 5) + (- 3)$ ;
 \item $(- 12) + (- 4) + (+ 7) + (+ 8) + (- 6)$ ;
 \item $(- 3) + (+ 5) + (- 4) + (+ 6) + (- 1)$ ;
 \item $(+ 1,2) + (- 4,2) + (+ 7,1) + (- 6,7)$.
 \end{enumerate}
\end{exercice}


\begin{exercice}[Durées de vie]
Remarque : pour cet exercice, n'oubliez pas que l'an 0 n'existe pas.
\begin{enumerate}
 \item Cicéron est né en l'an $- 23$ et est mort en l'an 38. Combien de temps a-t-il vécu ?
 \item Thalès de Milet est né en l'an $- 625$ et est mort à l'âge de 78 ans. En quelle année est-il mort ?
 \item L'Empire de Césarius a été créé en $- 330$ et s'est terminé en 213. Combien de temps a-t-il duré ?
 \item Ératosthène est mort en l'an $- 194$ à l'âge de 82 ans. En quelle année est-il né ?
 \item Thésée avait 11 ans à la mort de Claudius. Claudius est mort en l'an $- 18$. Thésée est mort en l'an 31. À quel âge est mort Thésée ?
 \end{enumerate}
\end{exercice}

%%%%%%%%%%%%%%%%%%%%%%%%%%%%%%%%%%%%%%%%%%%%%%%%%%%%%%%%%%%%%%%%%%%


\serie{Différences de relatifs}

\begin{exercice}
Recopie puis complète afin de transformer les soustractions suivantes en additions :
\begin{enumerate}
 \item $(+ 2) - (+ 7) = (+ 2) + (\ldots \ldots)$ ;
 \item $(- 4) - (+ 5) = (- 4) + (\ldots \ldots)$ ;
 \item $(- 8) - (- 14) =  (\ldots \ldots) + (\ldots \ldots)$ ;
 \item $(+ 9) - (- 9) =  (\ldots \ldots) + (\ldots \ldots)$.
 \end{enumerate}
\end{exercice}

\begin{exercice}
Transforme les soustractions suivantes en additions puis effectue-les :
\begin{colenumerate}{2}
 \item $(+ 4) - (+ 15)$ ;
 \item $(- 12) - (+ 5)$ ;
 \item $(- 10) - (- 7)$ ;
 \item $(+ 14) - (- 4)$ ;
 \item $(+ 6) - (+ 6)$ ;
 \item $(- 20) - (+ 7)$.
 \end{colenumerate}
\end{exercice}


\begin{exercice}
Effectue les soustractions suivantes :
\begin{colenumerate}{2}
 \item $(- 2,6) - (+ 7,8)$ ;
 \item $(+ 6,4) - (+ 23,4)$ ;
 \item $(+ 4,5) - (- 12,8)$ ;
 \item $(- 2,7) - (- 9,9)$ ;
 \item $(- 12,8) - (+ 9,5)$ ;
 \item $(+ 6,7) - (+ 2,4)$ ;
 \item $(+ 8,1) - (- 13,6)$ ;
 \item $(- 12,7) - (- 9,8)$.
 \end{colenumerate}
\end{exercice}


\begin{exercice}
Pour chaque expression, transforme les soustractions en additions puis effectue les calculs :
\begin{enumerate}
 \item $(+ 4) - (- 2) + (- 8) - (+ 7)$ ;
 \item $(- 27) - (- 35) - (- 20) + (+ 17)$ ;
 \item $(+ 3,1) + (- 3,5) - (+ 7,8) - (+ 1,6)$ ;
 \item $(- 16,1) - (+ 4,25) + (+ 7,85) - (+ 1,66)$.
 \end{enumerate}
\end{exercice}


\begin{exercice}
Jean et Saïd vont à la fête foraine. Ils misent la même somme d'argent au départ. Jean perd 2,30 CHF puis gagne 7,10 CHF. Saïd gagne 6 CHF puis perd 1,30 CHF. Lequel des deux amis a remporté le plus d'argent à la fin du jeu ?
\end{exercice}


\begin{exercice}
Pour chaque expression, transforme les soustractions en additions puis calcule les sommes :
\begin{enumerate}
 \item $(+ 12) - (- 6) + (- 2) + (+ 7) - (+ 8)$ ;
 \item $(- 20) - (+ 14) + (+ 40) + (- 12) - (- 10)$ ;
 \item $(- 2,4) + (- 7,1) - (- 3,2) - (+ 1,5) + (+ 8,4)$ ;
 \item $(+ 1,9) - (- 6,8) + (- 10,4) + (+ 7,7) - (+ 2)$.
 \end{enumerate}
\end{exercice}


\begin{exercice}
Le professeur Sésamatheux donne à ses élèves un questionnaire à choix multiples (Q.C.M) comportant huit questions. Il note de la façon suivante :
\begin{itemize}
 \item Réponse fausse ($F$) : $- 3$
 \item Sans réponse ($S$) : $- 1$
 \item Réponse bonne ($B$) : $+ 4$
 \end{itemize}
 \begin{enumerate}
 \item Calcule la note de Wenda dont les résultats aux questions sont : $F$ ; $B$ ; $S$ ; $F$ ; $F$ ; $B$ ; $B$ ; $S$. 
 \item Quelle est la note la plus basse qu'un élève peut obtenir ? Et la plus haute ?
 \item Quels sont les résultats possibles pour Emeline qui a obtenu une note $+ 4$ ?
 \end{enumerate}
\end{exercice}


\begin{exercice}
Calcule astucieusement les expressions suivantes :
\begin{enumerate}
 \item $(+ 14) + (- 45) + (- 14) + (+ 15)$ ;
 \item $(- 1,4) + (- 1,2) + (+ 1,6) - (+ 1,6)$ ;
 \item $(+ 1,35) + (- 2,7) - (- 0,65) + (- 1,3)$ ;
 \item $(- 5,45) - (- 0,45) + (+ 1,3) - (- 1) - (+ 1,3)$.
 \end{enumerate}
\end{exercice}


\begin{exercice}
Remplace les pointillés par le nombre qui convient :
\begin{enumerate}
 \item $(- 10) - \ldots \ldots  = 25$ ;
 \item $(+ 16) - \ldots \ldots  = 42$ ;
 \item $(+ 25) - (- 13) + (- 5) + \ldots \ldots = 26$ ;
 \item $(- 63) + (- 8) - \ldots \ldots + (+ 18) = 21$.
 \end{enumerate}
\end{exercice}


\begin{exercice}
Pour chaque cas, calcule en détail $x + y - z$ et $x - (y + z)$ :
\begin{center}
\begin{tabularx}{0.4\linewidth}{|X|c|c|c|}
\hline
 & x & y & z \\ \hline
\textbf{a.} & 10 & $- 3$ & 8 \\ \hline
\textbf{b.} & $- 6$ & $- 5$ & 2 \\ \hline
\textbf{c.} & 3 & $- 8$ & $- 2$ \\ \hline
\textbf{d.} & 7 & $- 2$ & $- 5$ \\ \hline
 \end{tabularx}
 \end{center}
\end{exercice}

%%%%%%%%%%%%%%%%%%%%%%%%%%%%%%%%%%%%%%%%%%%%%%%%%%%%%%%%%%%%%%%%%%%


\serie{Écriture simplifiée}

\begin{exercice}
Relie chaque expression à son écriture simplifiée :
\begin{center}
 \begin{tabularx}{0.8\linewidth}{|cc|X|cc|}
  \cline{1-2}\cline{4-5}
  $(- 8) + (- 16)$ & $\cdot$ & & $\cdot$ & $- 14 - 3$ \\ \cline{1-2}\cline{4-5}
  $(+ 24) - (- 4)$ & $\cdot$ & & $\cdot$ & $- 8 - 16$ \\ \cline{1-2}\cline{4-5}
  $(- 14) + (- 3)$ & $\cdot$ & & $\cdot$ & $14 + 8$ \\ \cline{1-2}\cline{4-5}
  $(- 7) - (+ 7)$ & $\cdot$ & & $\cdot$ & $- 7 - 7$ \\ \cline{1-2}\cline{4-5}
  $(+ 14) + (+ 8)$ & $\cdot$ & & $\cdot$ & $24 + 4$ \\ \cline{1-2}\cline{4-5}
  \end{tabularx}
  \end{center}
\end{exercice}


\begin{exercice}
Recopie et complète le tableau :
\begin{center}
\begin{tabularx}{1.04\linewidth}{|c|X|c|}
\hline
 & Écriture avec parenthèses & Écriture simplifiée \\ \hline
\textbf{a.} & \small{$(- 9) - (+ 13) + (- 15)$} & \\ \hline
\textbf{b.} & \small{$(- 10) + (+ 7) - (- 3) - (- 3)$} & \\ \hline
\textbf{c.} & \small{$(+ 5) - (- 2) + (+ 3) - (+ 2)$} & \\ \hline
\textbf{d.} & & \small{$- 6 - 8 + 5 - 3$} \\ \hline
\textbf{e.} & & \small{$15 - 13 - 8 - 7$} \\ \hline
\textbf{f.} & & \small{$- 13 - 5 - 9 + 1$} \\ \hline
 \end{tabularx}
 \end{center}
\end{exercice}


\begin{exercice}
Donne une écriture simplifiée des expressions suivantes en supprimant les parenthèses et les signes qui ne sont pas nécessaires:
\begin{enumerate}
 \item $(- 5) + (- 3)$ ;
 \item $(- 4) - (+ 6)$ ;
 \item $(+ 9) - (- 3)$ ;
 \item $(+ 4) + (+ 7)$ ;
 \item $(+ 17) - (- 5) + (+ 4) - (+ 5) - (- 3)$ ;
 \item $(- 15) + (+ 3,5) - (- 7,9) + (- 13,6)$.
 \end{enumerate}
\end{exercice}


\begin{exercice}
Effectue les calculs suivants :
\begin{colenumerate}{2}
 \item $5 - 14$ ;
 \item $8 - 13$ ;
 \item $- 6 - 6$ ;
 \item $- 13 + 9$ ;
 \item $- 53 - 48$ ;
 \item $- 2,8 - 4,7$ ;
 \item $- 5,7 + 4,4$ ;
 \item $3,2 - 8,9$.
 \end{colenumerate}
\end{exercice}


\begin{exercice}
Effectue en détail :
\begin{colenumerate}{2}
 \item $24 - 36 + 18$ ;
 \item $- 13 - 28 + 35$ ;
 \item $- 3,8 - 4,4 + 8,2$ ;
 \item $18 - 8 + 4 - 14$ ;
 \item $- 1,3 + 4,4 - 21$ ;
 \item $14 - 23 + 56 - 33$.
 \end{colenumerate}
\end{exercice}


\begin{exercice}
Effectue en détail :
\begin{enumerate}
 \item $5 + 13 - 4 + 3 - 6$ ;
 \item $- 7 + 5 - 4 - 8 + 13$ ;
 \item $3,5 - 4,2 + 6,5 - 3,5 + 5$ ;
 \item $25,2 + 12 - 4,8 + 24 - 3,4$.
 \end{enumerate}
\end{exercice}


\begin{exercice}
Regroupe les termes astucieusement puis effectue en détail :
\begin{enumerate}
 \item $13 + 15 + 7 - 15$ ;
 \item $- 8 + 4 + 18 - 2 + 12 + 6$ ;
 \item $4,3 - 7,4 + 4 - 2,25 + 6,7 + 3,4 - 2,75$ ;
 \item $- 2,5 + 4,8 - 3,6 + 0,2 + 2,5$.
 \end{enumerate}
\end{exercice}


\begin{exercice}
Calcule les expressions suivantes, en détail :
\begin{enumerate}
 \item $(- 3 + 9) - (4 - 11) - (- 5 - 6)$ ;
 \item $- 3 + 12 - (13 - 8) - (3 + 8)$ ;
 \item $- 3 - [4 - (3 - 9)]$.
 \end{enumerate}
\end{exercice}


\begin{exercice}[Températures]
Pour mesurer la température, il existe plusieurs unités. Celle que nous utilisons en Suisse est le degré Celsius ($^{\circ}$C). Cette unité est faite de façon à ce que la température à laquelle l'eau se transforme en glace est $0^{\circ}$C et celle à laquelle l'eau se transforme en vapeur est $100^{\circ}$C. Dans cette échelle, il existe des températures négatives. \\[0.5em]
Il existe une autre unité, le Kelvin (K), dans laquelle les températures négatives n'existent pas. Pour passer de l'une à l'autre, on utilise la formule :
\begin{center} $T_{Kelvin} = T_{degresCelsius} + 273,15$ \end{center}
Ainsi, $10^{\circ}$C correspondent à 283,15 K.
\begin{enumerate}
 \item Convertis en Kelvin les températures suivantes : $24^{\circ}$C ; $- 3^{\circ}$C et $- 22,7^{\circ}$C ;
 \item Convertis en degré Celsius les températures suivantes : 127,7 K ; 276,83 K ; 204 K et 500 K ;
 \item Quelle est en Kelvin la plus petite température possible ? À quelle température en degré Celsius correspond-elle ? Cette température est appelée le zéro absolu.
 \end{enumerate}
\end{exercice}