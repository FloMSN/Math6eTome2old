\documentclass[TS]{sesamanuel}

% modifications dans la classe (fichier sesamanuel.cls) :
% 	1) ligne 2873, la ligne de remise à zéro du compteur de chapitre a été commentée pour "themaG"
% 	2) création d'une boîte à connaître à la ligne 3471
%	3) ligne 2141 : définition de trois couleurs utilisé par les anciennes figures libreoffice
%	4) ligne 179 et 180 ajout de 2 lignes pour utiliser la police pazocal qui modifie les caractères calligraphiés en mode math
%		du coup : \mathcal{C} donne un C fortement calligraphié et \pazocal{C} donne l'habituel C calligraphié de mathcal
%	5) suppression du texte et du logo dans le titre des QCM d'autoévaluation "Ressources dispo sur sesamaths..." :
%		lignes 3117 et 4396 : commande \StringManuel est commentée
%		lignes 3122 et 4401 : je n'ai pas supprimé le \LogoManuel (une @) mais j'ai changé sa couleur de U4 à Blanc (à la ligne 2702)

% modifications dans le fichier commandesTikZ :
% lignes 31 et 32 pour ajouter la commande \circled qui entoure un caractère


% a priori, après l'impression du livre, tout ce fichier devrait être
% fusionné avec la classe principale, dernière version

\AtBeginDocument{\color{Noir}}

\renewcommand*\StringPrerequis{Connaissances
  n\'ecessaires \`a ce chapitre}

\usepackage{textcomp,pgfplots}
% standalone ne gère pas graphicspath... hum...
\newcommand{\standalonepath}[1]{#1}
% Au début de chaque chapitre on redéfinira cette commande en
% indiquant le chemin

\frenchbsetup{og=«,fg=»,}

\renewcommand\PrefixeCorrection{Corrections/}
\DeclareRemLike{intuition}{Idée intuitive}
\DeclareRemLike{exemples}{Exemples}


\newcommand*\StringExemples{Exemples}
\makeatletter
\newcommand*\smc@cartoucheexemples{% au pluriel
  \begin{pspicture}(-\ExempleVRuleWidthFrame,0)
                 (\ExempleWidthFrame,\ExempleHeightFrame)
    \psframe*[linewidth=0pt,linecolor=ExempleEdgeFrameColor]
             (-\ExempleVRuleWidthFrame,-\ExempleHRuleWidthFrame)
             (\ExempleWidthFrame,\ExempleHeightFrame)
    \psframe*[linewidth=0pt,linecolor=ExempleBkgFrameColor]
             (0mm,-0mm)(\ExempleWidthFrame,\ExempleHeightFrame)
    \rput[B](\dimexpr\ExempleWidthFrame/2,0){%
      \ExempleTitleFont
      \textcolor{ExempleTitleColor}{\StringExemples}%
    }
  \end{pspicture}%
}

\newenvironment{exemples*1}[1][]{%
  \par\addvspace{\BeforeExempleVSpace}
  \let\correction\smc@one@exemplecorrection
  \let\itemize\smc@exempleitemize
  \let\enditemize\endsmc@exempleitemize
  \let\colitemize\smc@exemplecolitemize
  \let\endcolitemize\endsmc@exemplecolitemize
  \let\enumerate\smc@exempleenumerate
  \let\endenumerate\endsmc@exempleenumerate
  \let\colenumerate\smc@exemplecolenumerate
  \let\endcolenumerate\endsmc@exemplecolenumerate
  \let\partie\smc@nopartie
  \let\exercice\smc@noexercice
  \let\endexercice\endsmc@noexercice
  \let\corrige\smc@nocorrige
  \let\endcorrige\endsmc@nocorrige
  \def\smc@currpart{Exemple}%
  \hspace*{\dimexpr \SquareWidth*3}%
  \color{ExempleRuleColor}%
  \vrule width \RuleWidth
  \hspace*{\dimexpr \SquareWidth-\RuleWidth}%
  \minipage[t]{\dimexpr\linewidth-\SquareWidth*4-\ExtraMarginRight}
    \smc@cartoucheexemples
    \space
    \color{Noir}%
    \ignorespaces
}
{%
  \endminipage
  \par
}

\DeclareRemLike{consequence}{Conséquence}
\DeclareRemLike{rappel}{Rappel}
\DeclareRemLike{valeurspart}{Valeurs particulières}
%\DeclareRemLike{theoreme}{Théorème}
% \NewThema{SP}
%          {sp}
%          {stat. et probabilités}
%          {Stat. et probabilités}
%          {STATISTIQUES\\ PROBABILITÉS}
%          {PartieStatistique}
%          {PartieStatistique}

\NewThema{A}
         {a}
         {analyse}
         {Analyse}
         {ANALYSE}
         {PartieFonction}
         {A3}


% Attention, il faudra ajouter un renewcommand \ListeMethodesThemes
% pour afficher la liste des méthodes à la fin du livre

% Si ils veulent changer la couleur du numéro des exos des
% auto-évaluations il faudra aller voir
% \colorlet{CorrigeNumExerciceFrameBkg}{J1}
 
\newcommand{\N}{\mathbb{N}} 
\newcommand{\R}{\mathbb{R}}
\newcommand{\Z}{\mathbb{Z}}
\renewcommand{\cfrac}[2]{{\displaystyle\frac{%
  \vrule height10pt depth0pt width0pt #1}{#2}}%
  \kern-\nulldelimiterspace}
%\usepackage{casio-fx,ti83symbols}

\newcommand\renvoimethode[1]{%
  Méthode \ref{#1}, p.~\pageref{#1}%
}

\newcommand*\calculatrice{%
  \psframebox[framesep=1pt,linewidth=\LogoLineWidth,
              linecolor=TiceLineColor, fillstyle=solid,
              fillcolor=TiceBkgColor, framearc=0.6]{%
    \TiceFont
    \textcolor{TiceTextColor}{CALC}%
  }
}
\newcommand\calc{\calculatrice}

% Chapitre G1 et G3
\newcommand{\covec}[2]{\left(\begin{array}{c} #1\\#2\end{array}\right)}

% Chapitre G2
\usepackage{tipa}
\newcommand{\arc}[1]{%
  \setbox9=\hbox{#1}%
  \ooalign{\resizebox{\wd9}{\height}{\texttoptiebar{\phantom{A}}}\cr#1}}

% Chapitre A4


\DeclareMathOperator{\e}{e}
\renewcommand{\cosh}{\operatorname{ch}}
\renewcommand{\sinh}{\operatorname{sh}}
\renewcommand{\tanh}{\operatorname{th}}

\newcommand*{\StringLEMM}{LEMME\footnote{Un \MotDefinition{lemme}{} est un résultat préliminaire ou  intermédiaire qui intervient parfois dans la preuve d'un théorème lorsqu'elle est un peu longue.}}
\newcommand*{\StringLEMME}{LEMME}
\DeclareDefLike{lemme}{\StringLEMME}
\DeclareDefLike{lemm}{\StringLEMM}

\newcommand*{\StringPROPRIETA}{PROPRIÉTÉ (admise)}
\DeclareDefLike{proprieta}{\StringPROPRIETA}

\usepackage{wrapfig}

% Environnement général pour toutes les fiches
\newcommand\AnnexeTICE{%
  \ChangeAnnexe{C2}{A1}{G1}{Blanc}%
  \annexe{}%
}
% Déclaration de l'environnement pour une fiche
\DeclareTPLike{ficheTICE}{Fiche}
              {TPTopColor}
              {TPBottomColor}
              {TPTitleColor} 
% Définition d'une commande \souspartie pour les besoins de la fiche
% TICE. C'est la commande \partie un peu revue. Je crée les mêmes
% paramètres de contrôle que pour TPPartie en mettant TPSousPartie à
% la place.

\colorlet{TPSousPartieColor}{J1}
\colorlet{TPSousPartieBkgColor}{C2}
\colorlet{TPSousPartieNumColor}{Blanc}
\newcommand*\TPSousPartieFont{\fontsize{10}{12}\sffamily\bfseries}
\def\BeforeTPSousPartieVSpace{3mm plus1mm minus1mm}
\def\AfterTPSousPartieVSpace{0mm plus1mm}
\edef\TPSousPartieHSep{\the\dimexpr\ItemRuleWidth+1.5mm}

\newcommand*\souspartie[1]{%
  \colorlet{smc@curr@partiecolor}{TPSousPartieNumColor}%
  \colorlet{smc@curr@partiebkgcolor}{TPSousPartieBkgColor}%
  \let\smc@curr@partiefont\TPSousPartieFont
  \par\addvspace{\BeforeTPSousPartieVSpace}
  \leavevmode
  \psframe*[linecolor=smc@curr@partiebkgcolor]
           (0,\ItemRuleDepth)(\ItemRuleWidth,\ItemRuleHeight)
  \hspace*{\TPSousPartieHSep}%
  \textcolor{TPSousPartieColor}{\TPSousPartieFont #1}
  \par\nobreak\addvspace{\AfterTPSousPartieVSpace}
}%

\newcommand\RoseItalTice[1]{\emph{\textcolor{C2}{#1}}}

%%% La présentation d'un texte en vis à vis d'une image n'est pas tout
%%% à fait un habillage et la répétition de tels éléments risque de
%%% poser des problèmes. Il vaut mieux se faire son propre « habillage
%%% »
\newcommand\ImageDroite[2]{%
  % #1 = texte
  % #2 = image (ou autre)
  \setbox4=\hbox{#2}%
  \dimen4=\dimexpr\ht4+\dp4-0.7\baselineskip
  \par
  \begin{tabularx}{\linewidth}{@{}Xc@{}}
    #1 & \raisebox{-\dimen4}{#2} %\box4
  \end{tabularx}
  \par
}
\newcommand{\commandetice}[1]{%
  \bgroup
  \shorthandoff{;:!?}%
  \texttt{#1}%
  \egroup
}

\newcommand\touchecalc[1]{%
  \tikz[baseline=-0.5ex]{\node at (0,0) [rounded corners = 2pt, draw, line width=.25pt]
    {\footnotesize\textsf{#1}}}%
}

\DeclareFontFamily{U}{tipa}{}
\DeclareFontShape{U}{tipa}{m}{n}{<->tipa10}{}
\newcommand{\arc@char}{{\usefont{U}{tipa}{m}{n}\symbol{62}}}%

\newcommand{\overarc}[1]{\mathpalette\arc@arc{#1}}

\newcommand{\arc@arc}[2]{%
  \sbox0{$\m@th#1#2$}%
  \vbox{
    \hbox{\resizebox{\wd0}{\height}{\arc@char}}
    \nointerlineskip
    \box0
  }%
}

% Pour A2

\def\psThomae{\pst@object{psThomae}}
\def\psThomae@i(#1,#2){%
   \begin@ClosedObj
   \addto@pscode{
     \psk@dotsize
     1 1 500 {
       dup
       /ipSave ED
       /ip ED
       1 1 500 {
         dup
         /iqSave ED
         /iq ED
         {
           iq 0 le { exit } if
           ip iq mod
           /ip iq def
           /iq ED
         } loop
         ip 1 eq {
           \psk@dotsize
           \@nameuse{psds@\psk@dotstyle}
           \pst@usecolor\pslinecolor ipSave iqSave div 1 iqSave div 
\tx@ScreenCoor
           2 copy moveto Dot
         } if
       } for
     } for
   }%
   \end@ClosedObj%
}

\renewcommand\smc@AfficheListeMethodesTheme[2]{%
  \expandafter\ifx\csname ifsmc@lom#1\endcsname\iftrue
    \csname smc@thema#2Color\endcsname
    \expandafter\smc@bandeaulistemethodes
      \expandafter{\csname StringListeMethode#2\endcsname}
    \ifnum \smc@NombreColonnesListeMethodes=\@ne
      \@starttoc{lom#1}
    \else
      \begin{multicols}{\smc@NombreColonnesListeMethodes}
        \@starttoc{lom#1}
      \end{multicols}
    \fi
  \fi
  \newpage
}
\fancypagestyle{empty}{%
  \fancyhead{}
  \fancyfoot{}
} 

\renewcommand*\AfficheCorriges[1][\NombreColonnesCorriges]{%
  \clearpage
  \label{toutes-solutions}
  \pagestyle{corrige}
  \thispagestyle{firstcorrige}
  \rput[Bl](0,9mm){\CorrigeTitleFont \MakeUppercase{\StringCorriges}}
  \vspace*{-5mm}
  \begingroup
  \columnsep \dimexpr \SquareWidth*2
  \columnseprule \CorrigeRuleWidth
  \def\columnseprulecolor{\color{ExerciceColumnRuleColor}}%
  \xdef\smc@NbColonneCorrige{#1}%
  \begin{multicols*}{#1}
  \raggedcolumns
    \@starttoc{cor}%
  \end{multicols*}
  \endgroup
}


\renewcommand\smc@preinsertlexiquefinal[3][]{%
  \@ifmtarg{#1}%
    {\smc@sansdiacritique{#2}}%
    {\smc@sansdiacritique{#1}}%
  \ifcsname affiche-\smc@tri\endcsname
  \else
    \expandafter\gdef\csname affiche-\smc@tri\endcsname{true}%
    \@ifmtarg{#1}%
      {\smc@sansdiacritique{#2}}%
      {\smc@sansdiacritique{#1}}%
    \global\advance\smc@numlexique \@ne
    \@ifmtarg{#1}%
      {\smc@@preFirstUppercase#2\@nil#3\@nil}%
      {\expandafter\protected@xdef\csname
lexique\the\smc@numlexique\endcsname
        {%
          \protect\textcolor{LexiqueEntreeColor}{%
            \protect\LexiqueEntreeFont #2%
          }%
%%%       \space\hbox to4.4em{\rdotfill}\kern0em\penalty0
          ~%%%
          \hspace*{\LexiquePageWidth}\penalty0
          \hspace{-\LexiquePageWidth}\dotfill
          \ifnum\csname nb-\smc@tri\endcsname>\@ne
            \protect\emph{ Pages~\csname pages-\smc@tri\endcsname}%
          \else
            \protect\emph{ Page~\csname pages-\smc@tri\endcsname}%
          \fi
        }%
      }%
    \expandafter\xdef\csname tri\the\smc@numlexique\endcsname
      {\smc@tri}%
  \fi
}
\long\def\smc@@preFirstUppercase#1#2#3\@nil#4\@nil{%
  \def\smc@arg{#1}%
  \ifx\smc@arg\smc@IeC
    \expandafter\protected@xdef\csname lexique\the\smc@numlexique\endcsname
      {%
        \protect\textcolor{LexiqueEntreeColor}
        {%
          \protect\LexiqueEntreeFont
          \MakeUppercase{#1#2}%
          \MakeLowercase{#3}%
        }%
        %%% \space\hbox to4.4em{\rdotfill}\kern-0.44em\penalty0
        ~%%%
        \hspace*{\LexiquePageWidth}\penalty0
        \hspace{-\LexiquePageWidth}\rdotfill
        \ifnum\csname nb-\smc@tri\endcsname>\@ne
          \protect\emph{ Pages~\csname pages-\smc@tri\endcsname}%
        \else
          \protect\emph{ Page~\csname pages-\smc@tri\endcsname}%
        \fi
      }%
  \else
    \expandafter\protected@xdef\csname lexique\the\smc@numlexique\endcsname
      {%
        \protect\textcolor{LexiqueEntreeColor}
        {%
          \protect\LexiqueEntreeFont
          \MakeUppercase{#1}%
          \MakeLowercase{#2#3}%
        }%
%%%     \space\hbox to4.4em{\rdotfill}\kern-0.44em\penalty0
        ~%%%
        \hspace*{\LexiquePageWidth}\penalty0
        \hspace{-\LexiquePageWidth}\rdotfill
        \ifnum\csname nb-\smc@tri\endcsname>\@ne
          \protect\emph{ Pages~\csname pages-\smc@tri\endcsname}%
        \else
          \protect\emph{ Page~\csname pages-\smc@tri\endcsname}%
        \fi
      }%
  \fi
} 

\makeatother

%\input{G1/biton} % pour l'instant on garde sinon les corrigés ne sont
						% pas dans classés dans un dossier "correction"

\usepackage{esvect,cancel} 
\newcommand{\chapeaumelon}[1]{\stackrel{\Large \frown}{#1}}

%%%%%%%% pour les figures en tikz
\usepackage{tikz}
\usepackage{tkz-tab,tkz-euclide}
\usetkzobj{all}
\usepackage{pgf}
\usetikzlibrary{arrows}
\usetikzlibrary{patterns}  
\definecolor{CyanTikz40}{cmyk}{.4,0,0,0}
\definecolor{CyanTikz20}{cmyk}{.2,0,0,0}

\definecolor{B1prime}      {cmyk}{0.00, 1.00, 0.00, 0.50}
\definecolor{H1prime}      {cmyk}{0.50, 0.00, 1.00, 0.00}

\tikzstyle{general}         =[font=\fontsize{7.5}{9}\selectfont,line width=0.3mm, >=stealth, x=1cm, y=1cm,line cap=round, line join=round]
\tikzstyle{quadrillage}     =[line width=0.3mm, color=CyanTikz40]
\tikzstyle{quadrillageNIV2} =[line width=0.3mm, color=CyanTikz20]
\tikzstyle{quadrillage55}   =[line width=0.3mm, color=CyanTikz40, xstep=0.5, ystep=0.5]
\tikzstyle{cote}            =[line width=0.3mm, <->]
\tikzstyle{epais}           =[line width=0.5mm, line cap=butt]
\tikzstyle{tres epais}      =[line width=0.8mm, line cap=butt]
\tikzstyle{axe}             =[line width=0.3mm, ->, color=Noir, line cap=rect]
\newcommand{\quadrillageSeyes}[2]{%
  \draw[line width=0.3mm, color=A1!10, ystep=0.2, xstep=0.8] #1 grid #2;
  \draw[line width=0.3mm, color=A1!30, xstep=0.8, ystep=0.8] #1 grid #2;
}

% ajouter pour manuel Flo
\newcommand*\circled[1]{\tikz[baseline=(char.base)]{
	\node[shape=circle,draw,inner sep=1pt] (char) {#1};}}

\newcommand{\axeX}[4][0]{%
  \draw[axe] (#2,#1)--(#3,#1);
  \foreach \x in {#4} {\draw (\x,#1) node {\small $+$};
    \draw (\x,#1) node[below] {\small $\numprint{\x}$};
  }%
}
\newcommand{\axeY}[4][0]{%
  \draw[axe] (#1,#2)--(#1,#3);
  \foreach \y in {#4} {\draw (#1, \y) node {\small $+$};
    \draw (#1, \y) node[left] {\small $\numprint{\y}$};
  }%
}
\newcommand{\axeOI}[3][0]{%
  \draw[axe] (#2,#1)--(#3,#1);
  \draw (1,#1) node {\small $+$};
  \draw (1,#1) node[below] {\small $I$};
}
\newcommand{\axeOJ}[3][0]{%
  \draw[axe] (#1,#2)--(#1,#3);
  \draw (#1, 1) node {\small $+$};
  \draw (#1, 1) node[left] {\small $J$};
}
\newcommand{\axeXgraduation}[2][0]{%
  \foreach \x in {#2} {\draw (\x,#1) node {\small $+$};}%
}
\newcommand{\axeYgraduation}[2][0]{%
  \foreach \y in {#2} {\draw (#1, \y) node {\small $+$};}%
}
\newcommand{\origine}{%
  \draw (0,0) node[below left] {\small $0$};
}
\newcommand{\origineO}{%
  \draw (0,0) node[below left] {$O$};
}
\newcommand{\point}[4]{%
  \draw (#1,#2) node[#4] {$#3$};
}
\newcommand{\pointGraphique}[4]{%
  \draw (#1,#2) node[#4] {$#3$};
  \draw (#1,#2) node {$+$};
}
\newcommand{\pointFigure}[4]{
  \draw (#1,#2) node[#4] {$#3$};
  \draw (#1,#2) node {$\times$};
}
\newcommand{\pointC}[3]{
  \draw (#1) node[#3] {$#2$};
}
\newcommand{\pointCGraphique}[3]{
  \draw (#1) node[#3] {$#2$};
  \draw (#1) node {$+$};
}
\newcommand{\pointCFigure}[3]{
  \draw (#1) node[#3] {$#2$};
  \draw (#1) node {$\times$};
}



\graphicspath{%
  {ex1/figures/}%
  {OpererAvecRelatifs/figures/}%
}


% création d'un nouveau thème "calcul" pour le document
\NewThema{C}{c}{calcul}{Calcul}{CALCUL}{PartieFonction}{A3}

\renewcommand\ListeMethodesThemes{{c}{C},{g}{G}}
\renewcommand*\StringListeMethode{M\'ethodes du livret 2}

\begin{document}

\themaC
\themaG


\setcounter{page}{6}

% ci-dessous c'est le chapitre d'exemples
\themaC
\chapter{Exemples d'usage}

\activites

\begin{activite}[Notion de limite]

Une première activité !

$2+2=4$

Sur la \textbf{demi-droite graduée} ci-dessous, quel est le nombre associé au point B ? Qu'est-ce qui te permet de l'affirmer ?

Ce nombre est associé à un événement historique important. Lequel ?
Décalque cette demi-droite et place le point N associé au nombre qui correspond à l'année de la chute du mur de Berlin.
Le nombre associé à un point sur une demi-droite graduée est l'\textbf{abscisse} de ce point.


\begin{partie}[une partie de l'activité...]
\vspace{-1.5em}
\begin{enumerate}
 \item Calculer $f(x)$ pour $x = 10$ ; 100 ; 1~000 ; $10^{4}$ ; $10^{5}$ ; etc.
 \item Que peut-on conjecturer quant à $f(x)$ lorsque $x \to +\infty$ ?
 \end{enumerate}
\end{partie}

\begin{partie}[... et une autre partie]
On vient de remarquer la propriété suivante, que l'on va par la suite chercher à démontrer (ah bon).
\end{partie}
\vspace{-4em}
\end{activite}



%%%%%%%%%%%%%%%%%%%%%%%%%%%%%%%%%%%%%%%%%%%%%%%%%%%%%%%%%%%%%%%%%%%%%%%%


\begin{activite}[Une autre activité]

\begin{partie}[partie 1]
Aux XVII\up{e} et XVIII\up{e} siècles, la notion de fonction. 
\end{partie}

\begin{partie}[Partie 2]
Au début du XIX\up{e} siècle, Bolzano et Cauchy.\end{partie}
\vspace{-4em}
\end{activite}


%%%%%%%%%%%%%%%%%%%%%%%%%%%%%%%%%%%%%%%%%%%%%%%%%%%%%%%%%%%%%%%%%%%%%%%%
%%%%%%%%%%%%%%%%%%%%%%%%%%%%%%%%%%%%%%%%%%%%%%%%%%%%%%%%%%%%%%%%%%%%%%%%
\pagebreak
\vspace*{-1cm}

\begin{activite}[Encore une activité (avec saut de page)]

\begin{partie}[Première partie]
On considère les deux fonctions $u$ et $v$ suivantes
\end{partie}

\begin{partie}[...Seconde partie]
Si $u$ et $v$ sont deux fonctions.

On donne les fonctions de référence $a$, $b$, $c$ et $d$ définies par :
\end{partie}

\begin{partie}[Partie 3]
Rien dans cette partie :)
\end{partie}
\end{activite}

\begin{debat}
Et là on peut mettre un petit débat.
\end{debat}



\cours
\section{Nombres entiers et décimaux}

% remarque : pour qu'un mot se retrouve dans le lexique : \MotDefinition{asymptote horizontale}{} 

Dans toute cette partie, $\mathscr{C}_f$ désigne la courbe représentative de la fonction $f$ dans un repère quelconque du plan.

\subsection{Limite finie en l’infini}

\begin{definition}
Soit $f$ une fonction définie au moins sur un intervalle de $\mathbb{R}$ du type $]a~;~+\infty[$.\\
La fonction $f$ a pour limite $\ell$ en $+\infty$ si tout intervalle ouvert contenant $\ell$
contient toutes les valeurs de $f(x)$ pour $x$ assez grand. On note alors : $\displaystyle \lim_{x\to+\infty}f(x)=\ell$.\\
\end{definition}

\begin{exemple*1}
Soit $f$ la fonction définie sur $]0~;~+\infty[$ par $f(x)=\dfrac{1}{x}+1$. On a $\displaystyle\lim_{x\to+\infty}\left(\dfrac{1}{x}+1\right)=1$.\\
En effet, l'inverse de $x$ se rapproche de $0$ à mesure que $x$ augmente.\\
Soit un intervalle ouvert $I$ tel que $1\in I$. Alors, $f(x)$ sera toujours dans $I$ pour $x$ assez grand. Graphiquement, aussi étroite que soit une bande parallèle à la droite d'équation $y=1$ et qui la contient, il existe toujours une valeur de $x$ au delà de laquelle $\mathscr{C}_f$ ne sort plus de cette bande.
\end{exemple*1}

\begin{definition}[Asymptote horizontale]
La droite d'équation $y=\ell$ est \MotDefinition{asymptote horizontale}{} à $\boldsymbol{\mathscr{C}_f}$ \textbf{en} $\boldsymbol{+\infty}$   si $\displaystyle\lim_{x\to+\infty} f(x)=\ell$.
\end{definition}

  \begin{remarque}
  On définit de façon analogue $\displaystyle\lim_{x\to-\infty} f(x)=\ell$
  qui caractérise une asymptote horizontale à $\mathscr{C}_f$ en $-\infty$ d'équation $y=\ell$.
  \end{remarque}

  \begin{exemple*1}
On a vu précédemment que $\displaystyle\lim_{x\to+\infty}\left(\dfrac{1}{x}+1\right)=1$. On a aussi $\displaystyle\lim_{x\to-\infty}\left(\dfrac{1}{x}+1\right)=1$.\\
Donc, la droite d'équation $y=1$  est asymptote horizontale à la courbe $\mathscr{C}_f$  en $+\infty$ et en $-\infty$ .
  \end{exemple*1}






\subsection{Limite infinie en l'infini}

\begin{definition}
La fonction $f$ a  pour limite $+\infty$ en $+\infty$ si tout intervalle de $\mathbb{R}$  du type  $]a~;~+\infty[$
contient toutes les valeurs de $f(x)$ pour $x$ assez grand. On note alors : $\displaystyle \lim_{x\to+\infty}f(x)=+\infty$.\\
\end{definition}

\begin{exemple*1}
Soit $f$ la fonction racine carrée. On a $\displaystyle\lim_{x\to+\infty}\sqrt{x}=+\infty$.\\
En effet, $\sqrt{x}$ devient aussi grand que l'on veut à mesure que $x$ augmente.\\
Soit un intervalle ouvert $I=]a~;~+\infty[$. Alors, $f(x)$ sera toujours dans $I$ pour $x$ assez grand.\\
 Graphiquement, si on considère le demi-plan supérieur de frontière une droite d'équation \mbox{$y=a$}, il existe toujours une valeur de $a$ au delà de laquelle $\mathscr{C}_f$ ne sort plus de ce demi-plan.
\begin{center}
\psset{xunit=0.5cm,yunit=1cm,algebraic=true}
\begin{pspicture*}(-2.5,-0.6)(20.5,4.8)
\psframe*[linecolor=H4](0,3.75)(20.5,4.8)
\psgrid[yunit=0.5cm,subgriddiv=1,linewidth=0.5pt,gridcolor=A3,gridlabels=0pt](0,0)(21,10)
\psaxes[linewidth=0.8pt,Dx=1,Dy=1,ticksize=-2pt]{->}(0,0)(-0.5,-0.25)(20.5,4.8)
\uput[r](1.2,2.7){\textcolor{B2}{$\boldsymbol{\mathscr{C}_{f}:y=\sqrt{x}}$}}
\psplot[linecolor=B2,plotpoints=5000,linewidth=1pt]{0.01}{20.5}{sqrt(x)}
\psline[linewidth=0.8pt,linestyle=dashed,linecolor=B2](0,3.75)(20.5,3.75)
\uput[l](0,3.75){\textcolor{H1}{$a$}}
\end{pspicture*}
\end{center}\vspace{-5mm}
\end{exemple*1}

\begin{remarque}
\begin{itemize}
\item On définit de façon analogue :
$\displaystyle\lim_{x\to +\infty}f(x)=-\infty$,
$\displaystyle\lim_{x\to -\infty}f(x)=+\infty$ et
$\displaystyle\lim_{x\to -\infty}f(x)=-\infty$.
\item Il existe des fonctions qui n'admettent pas de limite en
  l'infini. Par exemple, les fonctions sinus et cosinus n'admettent
  de limite ni en $+\infty$, ni en $-\infty$.
\item Une fonction qui tend vers $+\infty$ lorsque $x$ tend vers $+\infty$ n'est pas forcément croissante.
\end{itemize}
\begin{center}
\psset{xunit=1cm,yunit=.5cm,algebraic=true}
\begin{pspicture*}(-0.7,-1.5)(10.7,5.9)
\psgrid[yunit=0.5cm,subgriddiv=1,linewidth=0.5pt,gridcolor=A3,gridlabels=0pt](0,-2)(11,6)
\psaxes[linewidth=0.8pt,Dx=1,Dy=1,ticksize=-2pt]{->}(0,0)(-0.5,-1.5)(10.7,5.9)
\rput[0](9,1.5){\textcolor{B2}{$\boldsymbol{y=\sin 4x}$}}
\rput[0](3.5,3.5){\textcolor{H2!60!black}{$\boldsymbol{y=\cos 4x+\dfrac{x}{2}}$}}
\psplot[linecolor=B2,plotpoints=5000,linewidth=1pt]{0}{10.7}{sin(4*x)}
\psplot[linecolor=H2!60!black,plotpoints=5000,linewidth=1pt]{0}{10.7}{cos(4*x)+x/2}
\end{pspicture*}
\end{center}
\end{remarque}




% ci-dessous une boîte de méthode avec renvoi vers un des exercices
% la commande MethodeRefExercice contient la référence de l'exercice qui correspond à la fiche méthode
% la commande label contient la référence de la fiche méthode

\begin{methode*1}[Interpréter graphiquement les limites d'une fonction
  \MethodeRefExercice*{exo_test_boite_methode}]
L'aperçu de la courbe représentative d'une fonction avec une  calculatrice ou un logiciel peut aider à conjecturer une limite (et donc éventuellement une asymptote  à la courbe) mais il faut paramétrer correctement la fenêtre d'affichage pour limiter les erreurs de jugement.

  \exercice \label{test_boite_methode}
% \definecolor{fondTI}{HTML}{869286}
Soit $f$ une fonction dont on a un aperçu du graphe $\mathscr{C}$. Déterminer son ensemble de définition $\mathcal{D}$, puis conjecturer les limites aux bornes de $\mathcal{D}$ et les  asymptotes à $\mathscr{C}$.
\begin{colenumerate}{2}
\item $f:x\mapsto \dfrac{x^3-1}{x^3+1}$\par\vspace{1mm}
% \fcolorbox{fondTI}{fondTI}{
  
\item $f:x\mapsto 2x-\sqrt{4x^2-1}$ \par\vspace{1mm}
% \fcolorbox{fondTI}{fondTI}{
 \end{colenumerate}

  \correction

  \vspace{-3mm}

  \begin{enumerate}
  \item  $\mathcal{D}=\mathbb{R}\setminus\{-1\}$. A priori, on aurait : $\displaystyle\lim_{x\to\pm+\infty}f(x)=1$ ; $\displaystyle\lim_{\substack{x\to -1\\ x<-1}}f(x)=+\infty$ et $\displaystyle\lim_{\substack{x\to -1\\ x>-1}}f(x)=-\infty$.\par
        $\mathscr{C}$ aurait alors une asymptote  horizontale d'équation $y=1$ en $\pm\infty$ et une asymptote verticale d'équation $x=-1$.
  \item  $\mathcal{D}=]-\infty~;~-\tfrac{1}{2}[\,\cup\,]\tfrac{1}{2}~;~+\infty[$. On a : $\displaystyle\lim_{x\to -1/2}f(x)=-1$ et $\displaystyle\lim_{x\to 1/2}f(x)=1$ et, il semblerait que $\displaystyle\lim_{x\to-\infty}f(x)=-\infty$ et $\displaystyle\lim_{x\to +\infty}f(x)=0$.\par
        $\mathscr{C}$ aurait alors une asymptote horizontale d'équation $y=0$ (l'axe des abscisses) en $+\infty$.\par
         La vérification des conjectures est l'objet de l'exercice \RefExercice{suite_methode1} page \pageref{suite_methode1}.
  \end{enumerate}
  \vspace{-10mm}
\end{methode*1}

\vspace{-5mm}


\exercicesbase
\begin{colonne*exercice}
 \definecolor{fondTI}{HTML}{869286}


\serie{Limites : interprétation graphique}

% cet exercice possède à droite du titre un renvoi vers une fiche méthode du cours (elle doit exister!!)
% la commande ExerciceRefMethode contient la référence de la boite méthode
% la commande label contient la référence de l'exercice
% par convention on met le même nom avec "exo" en plus pour la référence de l'exercice
\begin{exercice*}[~\hfill\ExerciceRefMethode{test_boite_methode}]\label{exo_test_boite_methode}
Exercice avec renvoi à une fiche méthode et corrigé.
Soit la fonction $f$ définie sur $\mathbb{R}$  par :
\[f(x)= x^2\left(1 - \dfrac{x^2}{9}\right).\]
  \begin{enumerate}
  \item Conjecturer les limites de $f$ en $+\infty$ et en
    $-\infty$ à partir de la représentation graphique ci-dessous obtenue à l'aide d'un logiciel.
  \item Étudier les limites de $f$ en $+\infty$ et en $-\infty$.
  \item Expliquer pourquoi la conjecture était erronée.
  \end{enumerate}
 \begin{corrige}
  Ici on range le corrigé de l'exercice
  
  Test sur le corrigé
\end{corrige}
\end{exercice*}


% cet exercice contient un logo "INFO" à droite du titre
\begin{exercice}[~\hfill\tice]
  Soit $g$ la fonction définie par :
  \[g(x) = \dfrac{x}{\sqrt{3x^2 + x + 7}}\] représentée par
  $\mathscr{C}$ dans un repère.
  \begin{enumerate}
  \item Donner l'ensemble de définition de la fonction $g$.
  \item À l'aide d'un logiciel de géométrie dynamique :
    \begin{enumerate}
    \item Tracer la courbe $\mathscr{C}$.
    \item Conjecturer une valeur approchée de la limite en $+\infty$
      de la fonction $g$.
    \end{enumerate}
    \item Déterminer par calcul la valeur exacte de
      la limite de $g$ en $+\infty$.
  \end{enumerate}
\end{exercice}

% un exercice simple

\begin{exercice}
Soit $f$ la fonction définie sur $\mathbb{R}\setminus\{-3~;~3\}$ par :
\[f(x)=\dfrac{1-3x}{x^2-9}.\]
\begin{enumerate}
\item Déterminer la limite de $f$ en $-\infty$ et $+\infty$.
\begin{enumerate}
\item Sur une calculatrice, on a tracé le graphe de $f$ ce qui a donné l'écran suivant :
\item Expliquer pourquoi il semble apparaître une contradiction.
\end{enumerate}
\end{enumerate}
\begin{corrige}
  Et hop, encore un autre corrigé !
\end{corrige}
\end{exercice}

\serie{Limites : opérations}

% un exercice avec un titre en plus
% en plus cet exercice propose des questions sur deux colonnes "colenumerate"
\begin{exercice}[En $\boldsymbol{-2}$, c'est rationnel !]
 Étudier la limite de la fonction $f$ en  $-2$.
   \begin{colenumerate}{2}
    \item $f(x)=\dfrac{x-4}{x^2+3x+2}$
    \item $f(x)=\dfrac{-x^2+x+6}{2x^2+5x+2}$
    \item $f(x)=\dfrac{x^2-4}{\left(x+2\right)^2}$
    \item $f(x)=\dfrac{x^3+8}{x^2-x-6}$
    \end{colenumerate}
\end{exercice}


\begin{exercice}[En $\boldsymbol{0}$, c'est radical !]
 Étudier la limite de la fonction $f$ en  $0$.
   \begin{colenumerate}{2}
    \item $f(x)=\dfrac{\sqrt{x+1}}{x}$
    \item $f(x)=\dfrac{\sqrt{x+1}-1}{x}$
    \item $f(x)=\dfrac{\sqrt{x+4}-2}{x}$
    \item $f(x)=\dfrac{\sqrt{1-x}-1}{x^2-2x}$
    \end{colenumerate}
\end{exercice}






\begin{exercice}
Déterminer les limites suivantes.
\begin{colenumerate}{2}
\item $\displaystyle \lim_{x\to+\infty} \dfrac{2x+3}{3x-2}$
\item $\displaystyle \lim_{\substack{x\to 1\\ x<1}} \dfrac{x-1}{x^2+x-2}$
\item $\displaystyle \lim_{x\to-\infty}\sqrt{\dfrac{2x-1}{x-2}}$
\item $\displaystyle \lim_{\substack{x\to 1\\ x>1}} \dfrac{x-1}{x^2+x-2}$
\end{colenumerate}
\end{exercice}

\begin{exercice}
Déterminer les limites suivantes.
\begin{colenumerate}{2}
\item $\displaystyle \lim_{x\to+\infty} \sqrt{5-\dfrac{4}{x^2}}$
\item $\displaystyle \lim_{x\to-\infty} \left(2-\dfrac{1}{x}\right)^3$
\item $\displaystyle \lim_{x\to+\infty} \left(x-\sqrt{x}\right)$
\item $\displaystyle \lim_{\substack{x\to 0\\ x>0}} \sqrt{\dfrac{2-x}{x}}$
\end{colenumerate}
\end{exercice}


\serie{Limites : comparaison/encadrement}


% un exercice avec qcm 

\begin{exercice}
Soit une fonction $f$ telle que $f(x)$ vérifie une \mbox{inégalité} ou un encadrement sur un ensemble donné.\\
Indiquer les limites qu'on peut en déduire parmi les deux proposées.
\begin{enumerate}
\item Pour tout réel $x\neq0$, on a $\dfrac{1}{x}\leqslant f(x)$.
    \begin{ChoixQCM}{2}
\item $\displaystyle\lim_{\substack{x\to 0\\ x<0}}f(x)$
\item $\displaystyle\lim_{\substack{x\to 0\\ x>0}}f(x)$
\end{ChoixQCM}
\item Pour tout réel $x\neq0$, on a $f(x)\leqslant \dfrac{1}{x}$.
    \begin{ChoixQCM}{2}
\item $\displaystyle\lim_{\substack{x\to 0\\ x<0}}f(x)$
\item $\displaystyle\lim_{\substack{x\to 0\\ x>0}}f(x)$
\end{ChoixQCM}
\item Pour tout réel $x>1$, on a $x+\dfrac{1}{x}\leqslant f(x)\leqslant x+1$.
    \begin{ChoixQCM}{2}
\item $\displaystyle\lim_{\substack{x\to 1\\ x>1}}f(x)$
\item $\displaystyle\lim_{x\to+\infty}f(x)$
\end{ChoixQCM}
\item Pour tout réel $x>0$, on a $-\dfrac{1}{x}\leqslant f(x)\leqslant \dfrac{1}{x}$.
    \begin{ChoixQCM}{2}
\item $\displaystyle\lim_{\substack{x\to 0\\ x>0}}f(x)$
\item $\displaystyle\lim_{x\to+\infty}f(x)$
\end{ChoixQCM}
\item Pour tout réel $x\in]0~;~1[$, on a $|f(x)-1|\leqslant x$.
    \begin{ChoixQCM}{2}
\item $\displaystyle\lim_{\substack{x\to 0\\ x>0}}f(x)$
\item $\displaystyle\lim_{\substack{x\to 1\\ x<1}}f(x)$
\end{ChoixQCM}
\end{enumerate}
\end{exercice}


\end{colonne*exercice}


\exercicesappr
\begin{colonne*exercice}
Pour les exercices \RefExercice{continuite_first} à \RefExercice{continuite_last}, on donne ci-dessous la\linebreak définition de continuité en un réel.

\begin{cadre}[A1][A4]
Soit $f$ une fonction définie sur un intervalle $I$ de $\mathbb{R}$ et $x_0\in I$.
$f$ est \textbf{continue en} $\boldsymbol{x_0}$ si $\displaystyle\lim_{x\rightarrow x_0}f(x)=f(x_0)$.
 \end{cadre}



% 41
  \begin{exercice}\label{continuite_first}
La fonction $f$ définie sur $\mathbb{R}$ par :
    \[f (x)= \left\{\begin{array}{@{}cl}
    \dfrac{2 - \sqrt{x + 3}}{x - 1} & \text{si~} x\neq1\\
    -\dfrac{1}{4} & \text{si~} x=1
    \end{array}\right. \]
est-elle continue en $1$ ?
\end{exercice}

\begin{exercice}
 La fonction $f$ définie sur $[-1~;~+\infty[$ par :
    \[f(x)=\left\{
      \begin{array}{ll}
       \dfrac{x + 1}{\sqrt{x + 1}} & \text{si } x>-1\\
       1 & \text{si } x=-1.
      \end{array}\right.\]
est-elle continue en $-1$ ?
\end{exercice}
%
% 10
%
\begin{exercice}
 Soit $k$ un entier et $f$ une fonction définie sur $\mathbb{R}$.\\
 Déterminer $k$ pour que $f$  soit continue sur $\mathbb{R}$.
\begin{enumerate}
\item $f(x) =\left\{\begin{array}{ll}
      x^2 - 5 & \text{si } x < 1 \\
       k & \text{si } x \geqslant 1
    \end{array}\right.$.
\item $f(x) =\left\{\begin{array}{ll}
    k & \text{si } x= -1 \\
    \dfrac{2x + \sqrt{x + 5}}{x + 1} & \text{si } x>-1
  \end{array}\right.$.
  \end{enumerate}
\end{exercice}


  \begin{exercice}\label{continuite_last}
   Soit $a$ un réel et $g$ la fonction définie sur
        $\mathbb{R}$ par :
      \[g(x)=\left\{\begin{array}{@{}ll}
        x^2+1    & \text{si~} x \leqslant 1 \\
        x^2+ax+a & \text{si~} x > 1
        \end{array}\right..\]
      Peut-on déterminer $a$ pour que $g$ soit continue sur $\mathbb{R}$ ?
\begin{corrige}
Corrigé d'un exercice de la partie approfondissement.
\end{corrige}
  \end{exercice}
  
  
  \begin{exercice}[« La science est l'asymptote de la vérité »\footnote{« La science est l’asymptote de la vérité. Elle approche sans cesse et ne touche jamais. » d'après Hugo, Victor, \emph{William Shakspeare}.}]
Rudy a remarqué qu'« \emph{une asymptote, c'est comme une tangente à l'infini} ».
Son professeur  digresse alors.
\begin{enumerate}
\item Soit $f$ la fonction homographique propre :
\[f(x)=\dfrac{ax+b}{cx+d}\]
définie sur $\mathcal{D}=\mathbb{R}\setminus\left\{-\dfrac{d}{c}\right\}$ avec $c\neq0$ et $ad-bc\neq0$.\par
« \emph{Monsieur, pourquoi "homographique \textbf{propre}" ?} ».\par
De quel type serait la fonction $f$ :
\begin{colitemize}{2}
\item pour $c=0$ ? \item pour $ad-bc=0$ ?
\end{colitemize}
\item Montrez que :
\begin{enumerate}
\item $f(x)=\dfrac{a}{c}-\dfrac{ad-bc}{c(cx+d)}$ pour $x\in\mathcal{D}$. \label{homo2a}
\item $f(x)=\left(\dfrac{a+bx^{-1}}{c+dx^{-1}}\right)$ pour $x\in\mathcal{D}^*$. \label{homo2b}
\item $f'(x)=\dfrac{ad-bc}{\left(cx+d\right)^2}$ pour $x\in\mathcal{D}$.
\end{enumerate}
\item Déduisez de \RefItem{homo2a} et \RefItem{homo2b} les équations des asymptotes à la courbe représentative de $f$ aux bornes de $\mathcal{D}$. \label{homo3}
\item Calculez les limites suivantes :\label{homo4}
\begin{colenumerate}{2}
\item $\displaystyle\lim_{x\to\pm\infty}f'(x)$  \item $\displaystyle\lim_{x\to -d/c}f'(x)$
\end{colenumerate}
« \emph{Plus ou moins l'infini, vous n'en êtes pas sûr ?} ».\par
Le professeur précise qu'il veut les limites de $f'(x)$ en $+\infty$ et $-\infty$.
\item Rapprochez les résultats du  \RefItem{homo4} de celui du \RefItem{homo3}.\par
Concluez à propos de la remarque de Rudy.
\end{enumerate}

\begin{corrige}
Corrigé d'un autre exercice de la partie approfondissement !!
\end{corrige}
\end{exercice}



\end{colonne*exercice}

\connaissances
\begin{acquis}
\begin{itemize}
\item Déterminer la limite d’une somme,
d’un produit, d’un \mbox{quotient} ou d’une
composée de deux fonctions
\item Déterminer des limites par
comparaison et encadrement
\item Faire le lien entre limites et comportement asymptotique
\item Appréhender la notion de continuité d'une fonction
\item Exploiter le théorème des valeurs
intermédiaires (cas d'une
fonction strictement monotone) pour
résoudre un problème
\item Approcher une solution d'équation par l'algorithmique
\end{itemize}
\end{acquis}

\QCMautoevaluation{Pour chaque question, plusieurs réponses sont
  proposées.  Déterminer celles qui sont correctes.}

% 53
\begin{QCM}
  \begin{GroupeQCM}
    \begin{exercice}
      La limite en $+\infty$ de la fonction $f$ définie sur $]-\infty~;~-1[$ par $f(x)=\dfrac{1+x^2+x^3}{x\left(1-x^2\right)}$ est :
      \begin{ChoixQCM}{4}
      \item $0$
      \item $1$
      \item $-1$
      \item $-\infty$
      \end{ChoixQCM}
\begin{corrige}
     \reponseQCM{c}
   \end{corrige}
    \end{exercice}

    \begin{exercice}
      La limite à gauche  en $0$  de la fonction $f$ définie sur $[-1~;~0[$ par $f(x)=\sqrt{-\dfrac{x+1}{x}}$ est :
      \begin{ChoixQCM}{4}
      \item $0$
      \item $1$
      \item $-\infty$
      \item $+\infty$
      \end{ChoixQCM}
      \begin{corrige}
     \reponseQCM{d}
   \end{corrige}
    \end{exercice}

    % 54
    \begin{exercice}
      La limite en $+\infty$ de la fonction $f$ définie sur $\mathbb{R}\setminus\{-1~;~1\}$ par $f(x)=\dfrac{(2x-3)(x^2+1)}{\left(1-x^2\right)^2}$ est :
      \begin{ChoixQCM}{4}
      \item $-2$
      \item $0$
      \item $+\infty$
      \item $-\infty$
      \end{ChoixQCM}
      \begin{corrige}
     \reponseQCM{b}
   \end{corrige}
    \end{exercice}
  \end{GroupeQCM}
\end{QCM}


\begin{QCM}
  \begin{GroupeQCM}
    \begin{exercice}
      Soit $f$ une fonction définie sur $[2~;~+\infty[$. Si pour tout $x\geqslant 2$, on a $x^2 \leqslant f(x)$ alors :
      \begin{ChoixQCM}{4}
      \item $\displaystyle\lim_{x\to +\infty} f(x)=+\infty$
      \item $\displaystyle\lim_{x\to +\infty} \dfrac{f(x)}{x}=0$
      \item $\displaystyle\lim_{x\to +\infty} \dfrac{f(x)}{x}=+\infty$
      \item $\displaystyle\lim_{x\to +\infty} \dfrac{f(x)}{x^2}=1$
      \end{ChoixQCM}
      \begin{corrige}
     \reponseQCM{ac}
   \end{corrige}
    \end{exercice}


    \begin{exercice}
      La courbe représentative de la fonction
      $h:x\mapsto \dfrac{\left(2x - 1\right)^2}{2(4-x^2)}$ admet une asymptote d'équation :
      \begin{ChoixQCM}{4}
      \item $x = -2$
      \item $y = -2$
      \item $x = 2$
      \item $y = 2$
      \end{ChoixQCM}
      \begin{corrige}
     \reponseQCM{abc}
   \end{corrige}
    \end{exercice}

    \begin{exercice}
    Soit ci-dessous la courbe représentative d'une fonction $f$.
\begin{center}
\psset{xunit=1.cm,yunit=1.cm,algebraic=true}
\begin{pspicture*}(-5.25,-1.2)(5.25,1.4)
\psgrid[subgriddiv=1,linewidth=0.5pt,gridcolor=A3,subgridcolor=A3,gridlabels=0pt](-6,-2)(6,3)
\psaxes[linewidth=0.8pt,Dx=1,Dy=1,ticksize=-2pt]{->}(0,0)(-5.25,-1.2)(5.25,1.4)
%\psaxes[linewidth=0.5pt,Dx=10,Dy=10]{->}(0,0)(1,1)
  \rput[0](1.9,2.5){\textcolor{B2}{$\mathscr{C}$}}
\psplot[linecolor=B2,plotpoints=5000,linewidth=0.8pt]{-5.25}{-1.05}{((x^2+x)/(x^2-1)-(x^2-x)/(-x^2+1))/8}
\psplot[linecolor=B2,plotpoints=5000,linewidth=0.8pt]{-0.95}{0}{((x^2+x)/(x^2-1)-(x^2-x)/(-x^2+1))/8}
\psplot[linecolor=B2,plotpoints=5000,linewidth=0.8pt]{0}{0.95}{((x^2+x)/(x^2-1)-(x^2-x)/(-x^2+1))/8+0.13}
\psplot[linecolor=B2,plotpoints=5000,linewidth=0.8pt]{2.05}{5.25}{(((x-1)^2+x-1)/((x-1)^2-1)-((x-1)^2-x+1)/(-(x-1)^2+1))/8}
\psline[linecolor=B2,linewidth=0.8pt](1.2,-1.2)(1.8,1.4)
\rput(0,0){\textcolor{Blanc}{$\bullet$}} \rput(0,0){\textcolor{B2}{$\circ$}}
\rput(0,0.13){\textcolor{B2}{$\bullet$}}
\end{pspicture*}
\end{center}
Il est certain que la fonction $f$ n'est pas continue :
      \begin{ChoixQCM}{4}
      \item en $-1$
      \item en $0$
      \item en $2$
      \item en $6$
      \end{ChoixQCM}
      \begin{corrige}
     \reponseQCM{ab}
   \end{corrige}
    \end{exercice}




\end{GroupeQCM}
\end{QCM}

  

\TravauxPratiques % pour nous "travailler en groupe"

\begin{TP}[Un premier TP avec un logo à droite\hfill \algo]

\partie{Le principe et l'algorithme}

La \textbf{méthode de dichotomie} ou \textbf{méthode de la bissection} est un algorithme (voir ci-dessous) de recherche d'un zéro d'une fonction qui consiste à réitérer des partages d’un intervalle en deux  moitiés puis à sélectionner celui dans lequel se trouve le zéro de la fonction.\\
Si cela est possible, on dégrossit le plus souvent la recherche en se plaçant initialement sur un intervalle $[a~;~b]$ où la fonction est continue, strictement monotone
et telle que $f(a)f(b)<0$ afin d'appliquer le théorème des valeurs intermédiaires et assurer ainsi l'unicité de la solution.\\

\begin{minipage}{0.57\linewidth}
\begin{enumerate}
\item Que représente la variable $\varepsilon$ ?
\item Expliquer le premier pas de l'algorithme\par
dans les quatre cas de figures suivants :
\begin{ChoixQCM}{2}
\item \begin{minipage}{\linewidth}
\psset{xunit=0.8cm,yunit=0.8cm,algebraic=true}
\begin{pspicture*}(-0.2,-1.1)(3.2,1.1)
\psaxes[yAxis=false,linewidth=0.8pt,Dx=3,ticksize=-2pt]{->}(0,0)(-0.2,-1.2)(3.2,1.2)
\psplot[linecolor=B2,plotpoints=5000,linewidth=0.8pt]{0}{3}{x^2/4.5-1}
\psline[linestyle=dashed,linewidth=0.5pt](0,0)(0,-1)
\psline[linestyle=dashed,linewidth=0.5pt](3,0)(3,1)
\uput[u](0,0){$a$} \uput[d](3,0){$b$}
\end{pspicture*}
\end{minipage}
\item \begin{minipage}{\linewidth}
\psset{xunit=0.8cm,yunit=0.8cm,algebraic=true}
\begin{pspicture*}(-0.2,-1.1)(3.2,1.1)
\psaxes[yAxis=false,linewidth=0.8pt,Dx=3,ticksize=-2pt]{->}(0,0)(-0.2,-1.2)(3.2,1.2)
\psplot[linecolor=B2,plotpoints=5000,linewidth=0.8pt]{0}{3}{-(x-3)^2/4.5+1}
\psline[linestyle=dashed,linewidth=0.5pt](0,0)(0,-1)
\psline[linestyle=dashed,linewidth=0.5pt](3,0)(3,1)
\uput[u](0,0){$a$} \uput[d](3,0){$b$}
\end{pspicture*}
\end{minipage}
\item \begin{minipage}{\linewidth}
\psset{xunit=0.8cm,yunit=0.8cm,algebraic=true}
\begin{pspicture*}(-0.2,-1.1)(3.2,1.1)
\psaxes[yAxis=false,linewidth=0.8pt,Dx=3,ticksize=-2pt]{->}(0,0)(-0.2,-1.2)(3.2,1.2)
\psplot[linecolor=B2,plotpoints=5000,linewidth=0.8pt]{0}{3}{-x^2/4.5+1}
\psline[linestyle=dashed,linewidth=0.5pt](0,0)(0,1)
\psline[linestyle=dashed,linewidth=0.5pt](3,0)(3,-1)
\uput[d](0,0){$a$} \uput[u](3,0){$b$}
\end{pspicture*}
\end{minipage}
\item \begin{minipage}{\linewidth}
\psset{xunit=0.8cm,yunit=0.8cm,algebraic=true}
\begin{pspicture*}(-0.2,-1.1)(3.2,1.1)
\psaxes[yAxis=false,linewidth=0.8pt,Dx=3,ticksize=-2pt]{->}(0,0)(-0.2,-1.2)(3.2,1.2)
\psplot[linecolor=B2,plotpoints=5000,linewidth=0.8pt]{0}{3}{(x-3)^2/4.5-1}
\psline[linestyle=dashed,linewidth=0.5pt](0,0)(0,1)
\psline[linestyle=dashed,linewidth=0.5pt](3,0)(3,-1)
\uput[d](0,0){$a$} \uput[u](3,0){$b$}
\end{pspicture*}
\end{minipage}
\end{ChoixQCM}
\end{enumerate}
\end{minipage}
\begin{minipage}{0.42\linewidth}
\begin{oldalgorithme}
Lire a, b, $\varepsilon$
Tant que (b-a)>$\varepsilon$
  c prend la valeur (a+b)/2
  Si f(a)*f(c)>0 alors
    a prend la valeur c
  Sinon
    b prend la valeur c
  Fin Si
Fin Tant Que
Afficher c
\end{oldalgorithme}
\end{minipage}

\partie{Application : approcher le nombre d'or}

Intéressons-nous au nombre d'or, solution positive de l'équation :
\[\text{(E)} \quad x^2-x-1=0\]
\begin{enumerate}
\item Soit la fonction $f:x\mapsto x^2-x-1$ qu'on étudie sur $[1~;~2]$.
\begin{enumerate}
\item Justifier que la fonction $f$ est continue sur $[1~;~2]$.
\item Dresser le tableau de variation complet de $f$ sur $[1~;~2]$.
\item Montrer qu'il existe une solution unique $\varphi$ à l'équation $f(x)=0$.
\end{enumerate}
\item On applique l'algorithme de dichotomie à $f$ avec $a=1$, $b=2$ et $\varepsilon =10^{-5}$.
\begin{enumerate}
\item Justifier qu'après le premier pas,  $\varphi\in[1,5~;~2]$ et, qu'après le second, $\varphi\in[1,5~;~1,75]$.\par
\begin{minipage}{0.6\linewidth}
\psset{xunit=6cm,yunit=1.5cm,algebraic=true}
\begin{pspicture*}(0.8,-1.2)(2.2,1.1)
\psgrid[subgriddiv=10,linewidth=0.5pt,gridcolor=A3,subgridcolor=A3,gridlabels=0pt](1,-1)(2,1)
\psaxes[comma,Ox=1,linewidth=0.8pt,Dx=0.1,Dy=1,ticksize=-2pt]{->}(1,0)(1,-1)(2.05,1.1)
\psplot[linecolor=B2,plotpoints=5000,linewidth=0.8pt]{1}{1.5}{x^2-x-1}
\psplot[linecolor=H1!60!black,plotpoints=5000,linewidth=0.8pt]{1.5}{2}{x^2-x-1}
\end{pspicture*}
\end{minipage}\hspace{1cm}
\begin{minipage}{0.35\linewidth}
\psset{xunit=6cm,yunit=1.5cm,algebraic=true}
\begin{pspicture*}(1.3,-1.2)(2.2,1.1)
%\psgrid[subgriddiv=20,linewidth=0.5pt,gridcolor=A3,subgridcolor=A3,gridlabels=0pt](1.5,-1)(2,1)
\multido{\dx=1.50\psxunit+0.05\psxunit}{11}{%
  \psline[linewidth=0.5pt,linecolor=A3](\dx,-1)(\dx,1)
}
\multido{\dy=-1.00\psyunit+0.05\psyunit}{41}{%
  \psline[linewidth=0.5pt,linecolor=A3](1.5,\dy)(2.0,\dy)
}
\psaxes[comma,Ox=1.5,linewidth=0.8pt,Dx=0.1,Dy=1,ticksize=-2pt]{->}(1.5,0)(1.5,-1)(2.05,1.1)
\psplot[linecolor=H1!60!black,plotpoints=5000,linewidth=0.8pt]{1.5}{1.75}{x^2-x-1}
\psplot[linecolor=B2,plotpoints=5000,linewidth=0.8pt]{1.75}{2}{x^2-x-1}
\end{pspicture*} 
\end{minipage}
\item À l'aide d'AlgoBox ou d'un autre logiciel, programmer l'algorithme de dichotomie pour qu'il affiche les encadrements successifs de $\varphi$ et leurs précisions.
\begin{center}
\begin{tabular}{c@{\,}c@{\,}c|c}
1,5 & $<\varphi<$ &   2  &  0,5 \\
1,5 & $<\varphi<$ &   1,75  &  0,25 \\
 & \vdots  &     & \vdots \\
\end{tabular}
\end{center}
\end{enumerate}
\item On définit  la suite $(p_n)_{n\geqslant0}$ par $p_0=1$ et $p_{n+1}=\dfrac{p_n}{2}$.
\begin{enumerate}
\item Que représente $(p_n)$ ? Justifier qu'elle est décroissante et exprimer $p_n$ en fonction de $n$.
\item Écrire puis programmer un algorithme qui prend en entrée $\varepsilon$ et qui retourne le plus petit entier $n$ tel que $p_n<\varepsilon$ ?
\item À l'aide du programme, déterminer le plus petit entier $n$ tel que  $p_n$ soit inférieur à :
\begin{colitemize}{5}
\item $0,1$ \item $0,01$ \item $0,001$ \item $0,000\,1$ \item $0,000\,01$
\end{colitemize}
Commenter l'efficacité de l'algorithme de dichotomie à partir des résultats obtenus.  
\end{enumerate}
\end{enumerate}
\end{TP} 

\begin{TP}[Et un autre TP avec deux logos à droite\hfill\tice \algo]

La \textbf{méthode de Newton} est une autre méthode destinée à déterminer une valeur approchée du zéro d'une fonction, sous condition de sa dérivabilité sur un intervalle réel.\\
Partant d'un réel $x_0$ de préférence proche du zéro à trouver, on approche la fonction $f$ au premier ordre en la considérant à peu près égale à la fonction affine donnée par l'équation de la tangente à sa courbe représentative au point d'abscisse $x_0$ :
\[f(x)\simeq  f'(x_0)(x-x_0)+f(x_0).\]
On  résout alors l'équation $f'(x_0)(x-x_0)+f(x_0)=0$ pour obtenir $x_1$ qui, en général, est plus proche du zéro de $f$ que $x_0$. 
On réitère ensuite le processus.\\[2mm]
Le but de ce TP est de déterminer une valeur approchée du nombre d'or $\varphi$ comme dans le TP précédent et de comparer l'efficacité de la méthode de Newton à celle de dichotomie.
\partie{Approche graphique}\vspace{-5mm}
\begin{enumerate}
\item Avec un logiciel de géométrie dynamique, tracer le graphe $\mathscr{C}$  de $f:x\mapsto x^2-x-1$.
\item Tracer la tangente à $\mathscr{C}$ au point d'abscisse $x_0=1$. Elle coupe l'axe des abscisses en $A_1(x_1~;~0)$.
\item Réitérer le processus pour obtenir $x_1$ puis $x_2$. Est-on proche de $\varphi$ ?
\end{enumerate}

\partie{Avec l'algorithmique}
La construction devient vite compliquée avec l'agglomérat des tangentes successives.\\
On souhaite ainsi s'orienter vers l'élaboration et la programmation d'un algorithme.
\begin{enumerate}
\item Justifier qu'on peut définir la suite $(x_n)$ telle que $x_{n+1}=x_n-\dfrac{f(x_n)}{f'(x_n)}$.
\item Écrire et programmer l'algorithme en considérant la condition d'arrêt $|x_{n+1}-x_n|<\varepsilon$. 
\item Faire tourner l'algorithme pour $\varepsilon$ égal à $10^{-1}$, $10^{-2}$, \ldots, $10^{-5}$.
\item Rajouter un compteur d'itérations pour estimer l'efficacité de la méthode. Conclure.
\end{enumerate}
\end{TP}


\pagebreak

\recreation
\begin{enigme}[Des discontinuités... en continu !]


Soit $x$ et $y$ deux réels tels que $x<y$.\\
Définissons la suite  $(d_n)_{n\geqslant0}$ telle que $d_n=\dfrac{\lfloor10^n y\rfloor}{10^n}$ où $\lfloor a\rfloor$ désigne la partie entière de $a$.\\
\begin{enumerate}
\item À quel ensemble les nombres $d_n$  appartiennent-ils ? \hspace{1cm}
\begin{minipage}{0.5\linewidth}
\begin{colitemize}{5}
\item $\mathbb{N}$ ? \item $\mathbb{Z}$ ? \item $\mathbb{D}$ ? \item $\mathbb{Q}$ ? \item $\mathbb{R}$ ?
\end{colitemize}
\end{minipage}
\item
\begin{enumerate}
\item Montrer  que pour tout $n\in\mathbb{N}$, on a l'encadrement $\dfrac{10^ny-1}{10^n}< d_n\leqslant y$.
\item En déduire $\displaystyle\lim_{n\to+\infty}d_n$.
\end{enumerate}
\item \begin{enumerate}
\item Montrer que, quel que soit $\varepsilon>0$, il existe un entier naturel $N$ tel que pour tout $n\geqslant N$, $|d_n-y|< \varepsilon$.
\item En posant $\varepsilon=y-x$, en déduire que $x\leqslant d_N \leqslant y$.
\end{enumerate}
\end{enumerate}

\begin{center}
\begin{minipage}{1\linewidth}
\begin{cadre}[A1][A4]
On vient de montrer qu'entre deux réels, il existe toujours un décimal et donc toujours un rationnel.\\
On dit que l'ensemble des rationnels $\mathbb{Q}$ est \textbf{dense} dans l'ensemble des réels $\mathbb{R}$.
 \end{cadre}
\end{minipage}
\end{center}

La \textbf{fonction de Dirichlet} D et la \textbf{fonction de Thomae} T sont deux fonctions définies sur $\mathbb{R}$ par  :
\[\text{D}(x)=\left\{\begin{matrix} 1  &  \text{si~} x\in\mathbb{Q} \\ 0  & \text{si~} x\notin\mathbb{Q} \end{matrix}\right.
\text{~~et~~} \text{T}(x)=\left\{\begin{array}{@{}ll} 0  & \text{si~} x\notin\mathbb{Q} \\ 1 & \text{si~} x=0 \\ \dfrac{1}{q} & \text{si~} x=\dfrac{p}{q} \text{~est une fraction irréductible}\end{array}\right.\]
Introduite par Dirichlet\footnote{Johann Peter Gustav Lejeune Dirichlet (1805–1859), mathématicien allemand}  en 1829, la fonction D est discontinue partout ce que le résultat établi précédemment montre. Cette fonction est appelée aussi \textbf{fonction indicatrice des rationnels}.\\
Introduite par Thomae\footnote{Carl Johannes Thomae (1840–1921), mathématicien allemand} en 1875, la fonction T est continue en tout nombre irrationnel mais  discontinue en tout nombre rationnel. Cette fonction est appelée aussi la \textbf{fonction popcorn} (voir sa représentation ci-dessous !).
\begin{center}

\end{center}




\end{enigme} 




% et la, premier chapitre du livre
%\themaC
\chapter{Opérer avec les relatifs}

\activites

\begin{activite}[Une activité]

\begin{partie}[Une partie]
Blabla

\end{partie}

\begin{partie}[Une partie]
Bla bla
\end{partie}

\end{activite}




\cours
%\section{Une section}

% remarque : pour qu'un mot se retrouve dans le lexique : \MotDefinition{asymptote horizontale}{} 

\begin{methode*1}[Additionner deux nombres relatifs]

\begin{aconnaitre}
Pour \MotDefinition{additionner deux nombres relatifs de même signe}{}, on additionne leurs valeurs absolues et on garde le signe commun.

Pour \MotDefinition{additionner deux nombres relatifs de signes contraires}{}, on soustrait leurs valeurs absolues et on prend le signe de celui qui a la plus grande distance à zéro.
\end{aconnaitre}

\begin{exemple*1}
Effectue l'addition suivante : $A = (-2) + (-3)$.
\begin{tabular}{ll} 
$A = (-2) + (-3)$ & $\rightarrow$ On veut additionner deux nombres négatifs. \\
$A = -(2 + 3)$ & $\rightarrow$ On additionne les valeurs absolues \\
 & \phantom{$\rightarrow$} et on garde le signe commun : $-$. \\
$A = -5$ & $\rightarrow$ On calcule. \\
\end{tabular}
 \end{exemple*1}
 
 \begin{exemple*1}
Effectue l'addition suivante : $B = (-5) + (+7)$.
\begin{tabular}{ll} 
$B = (-5) + (+7)$ & $\rightarrow$ On veut additionner deux nombres de signes différents. \\
$B = +(7 - 5)$ & $\rightarrow$ On soustrait leurs valeurs absolues et on écrit \\
 & \phantom{$\rightarrow$} le signe du nombre qui a la plus grande valeur absolue. \\
$B = +2$ & $\rightarrow$ On calcule. \\
\end{tabular}
 \end{exemple*1}
 
 \exercice  
Effectue les additions suivantes :
\begin{colenumerate}{3}
 \item (+7) + (+4 ) ;
 \item (+12) + (-15) ;
 \item (-7) + (+19) ;
 \item (-11) + (-9) ;
 \item (+1) + (+3) + (-2) ;
 \item (-2) + (-6) + (+7) ;
 \item (-10,8) + (+2,5) ;
 \item (+25,2) + (-15,3) ;
 \item (-21,15) + (+21,15).
 \end{colenumerate}
%\correction

 \end{methode*1}

%%%%%%%%%%%%%%%%%%%%%%%%%%%%%%%%%%%%%%%%%%%%%%%%%%%%%%%%%%%%%%%%%%

\begin{methode*1}[Soustraire deux nombres relatifs]

\begin{aconnaitre}
\MotDefinition{Soustraire un nombre relatif}{} revient à additionner son opposé.
\end{aconnaitre}

 \begin{exemple*1}
Effectue la soustraction suivante : $C = (-2) - (-3)$.
\begin{tabular}{ll} 
$C = (-2) - (-3)$ & $\rightarrow$ On veut soustraire le nombre $-3$. \\
$C = (-2) + (+3)$ & $\rightarrow$ On additionne l'opposé de $-3$. \\
$C = + (3 - 2)$ & $\rightarrow$ On additionne deux nombres de signes différents donc \\
 & \phantom{$\rightarrow$} on soustrait leurs valeurs absolues et on écrit \\
 & \phantom{$\rightarrow$} le signe du nombre qui a la plus grande valeur absolue. \\
$C = +1$ & $\rightarrow$ On calcule. \\
\end{tabular}
 \end{exemple*1}

\exercice
Transforme les soustractions en additions et effectue :
\begin{colenumerate}{2}
 \item $(+5) - (-6)$ ;
 \item $(-3) - (+2)$ ;
 \item $(+4) - (+8)$ ;
 \item $(-7) - (-3,8)$ ;
 \item $(-2,3) - (+7)$ ;
 \item $(+6,1) - (-2)$.
 \end{colenumerate}
%\correction

\exercice
Effectue les soustractions suivantes :
\begin{colenumerate}{2}
 \item $(+3) - (-6)$ ;
 \item $(-3) - (-3)$ ;
 \item $(+7) - (+3)$ ;
 \item $(-5) - (+12)$ ;
 \item $(+2,1) - (+4)$ ;
 \item $(-7) - (+8,25)$.
 \end{colenumerate}
%\correction

 \end{methode*1}
 
 %%%%%%%%%%%%%%%%%%%%%%%%%%%%%%%%%%%%%%%%%%%%%%%%%%%%%%%%%%%%%%%%%%

\begin{methode*1}[Simplifier l'écriture d'un calcul]

\begin{aconnaitre}
Dans une suite d'additions de nombres relatifs, on peut supprimer les signes d'additions et les parenthèses autour d'un nombre.

Un nombre positif écrit en début de calcul peut s'écrire sans son signe.
\end{aconnaitre}

\begin{remarque}
Dans le cas d'une expression avec des soustractions, on peut se ramener à une suite d'additions.
 \end{remarque}

 \begin{exemple*1}
Simplifie l'expression $D = (+4) + (-11) - (+3)$ :
\begin{tabular}{ll} 
$D = (+4) + (-11) + (-3)$ & $\rightarrow$ On transforme les soustractions en additions \\
 & \phantom{$\rightarrow$} des opposés. \\
$D = +4 - 11 - 3$ & $\rightarrow$ On supprime les signes d'additions et \\
 & \phantom{$\rightarrow$} les parenthèses autour des nombres. \\
$D = 4 - 11 - 3$ & $\rightarrow$ On supprime le signe $+$ en début de calcul. \\
\end{tabular}
 \end{exemple*1}

\exercice
Simplifie les écritures suivantes :
\begin{enumerate}
 \item $(-5) - (-135) + (+3,41) + (-2,65)$ ;
 \item $(+18) - (+15) + (+6) - (-17)$.
 \end{enumerate}
%\correction

\end{methode*1}


\exercicesbase
\begin{colonne*exercice}

\serie{Sommes de relatifs}

\begin{exercice}
Recopie dans ton cahier, effectue les additions puis relie chaque calcul à son résultat :
\begin{center}
 \begin{tabularx}{0.95\linewidth}{|cc|X|cc|}
  \cline{1-2}\cline{4-5}
  $(- 12) + (- 4)$ & $\cdot$ & & $\cdot$ & $+ 4$ \\ \cline{1-2}\cline{4-5}
  $(+ 12) + (- 4)$ & $\cdot$ & & $\cdot$ & $- 20$ \\ \cline{1-2}\cline{4-5}
  $(- 12) + (- 8)$ & $\cdot$ & & $\cdot$ & $- 16$ \\ \cline{1-2}\cline{4-5}
  $(- 8) + (+ 12)$ & $\cdot$ & & $\cdot$ & $+ 12$ \\ \cline{1-2}\cline{4-5}
  $(+ 8) + (+ 4)$ & $\cdot$ & & $\cdot$ & $+ 8$ \\ \cline{1-2}\cline{4-5}
  \end{tabularx}
  \end{center}
\end{exercice}

\begin{exercice}
Effectue les additions suivantes :
\begin{colenumerate}{2}
 \item $(+ 2) + (+ 7)$ ;
 \item $(- 4) + (+ 5)$ ;
 \item $(- 8) + (- 14)$ ;
 \item $(+ 9) + (- 9)$ ;
 \item $(- 20) + (- 12)$ ;
 \item $(+ 40) + (- 60)$ ;
 \item $(- 36) + (+ 18)$ ;
 \item $(- 25) + (+ 0)$.
 \end{colenumerate}
\end{exercice}


\begin{exercice}
Effectue les additions suivantes :
\begin{colenumerate}{2}
 \item $(- 8) + (- 16)$ ;
 \item $(+ 24) + (- 4)$ ;
 \item $(- 14) + (- 3)$ ;
 \item $(- 7) + (+ 7)$ ;
 \item $(+ 14) + (+ 8)$ ;
 \item $(+ 11) + (+ 33)$ ;
 \item $(+ 30) + (- 47)$ ;
 \item $(+ 19) + (+ 1)$ ;
 \item $(- 11) + (- 13)$ ;
 \item $(+ 63) + (- 63)$.
 \end{colenumerate}
\end{exercice}


\begin{exercice}
Effectue les additions suivantes :
\begin{colenumerate}{2}
 \item $(- 2,3) + (- 4,7)$ ;
 \item $(+ 6,8) + (- 9,9)$ ;
 \item $(- 3,5) + (+ 1,8)$ ;
 \item $(- 2,51) + (- 0)$ ;
 \item $(- 7,8) + (- 2,1)$ ;
 \item $(+ 13,4) + (- 20,7)$ ;
 \item $(- 10,8) + (+ 11,2)$ ;
 \item $(+ 17) + (+ 5,47)$.
 \end{colenumerate}
\end{exercice}


\begin{exercice}[La pyramide]
Recopie puis complète les pyramides suivantes sachant que le nombre contenu dans une case est la somme des nombres contenus dans les deux cases situées en dessous de lui :

\begin{minipage}[c]{0.48\linewidth}
\begin{center} \boxed{\phantom{hello}} \end{center}
\vspace{-0.69cm}
\begin{center} \boxed{\phantom{hello}} \negthinspace \boxed{\phantom{hello}} \end{center}
\vspace{-0.71cm}
\begin{center} \boxed{\phantom{hello}} \negthinspace \boxed{\phantom{hello}} \negthinspace  \boxed{\phantom{hello}} \end{center}
\vspace{-0.71cm}
\begin{center} \negthinspace \boxed{\phantom{!}$- 21$\phantom{!}} \negthinspace \boxed{$+ 12$} \negthinspace \boxed{\phantom{!}$- 4$\phantom{!}} \negthinspace \boxed{$- 9$\phantom{.}} \end{center}
 \end{minipage} \hfill%
 \begin{minipage}[c]{0.48\linewidth}
\begin{center} \boxed{\phantom{hello}} \end{center}
\vspace{-0.69cm}
\begin{center} \boxed{\phantom{hello}} \negthinspace \boxed{\phantom{hello}} \end{center}
\vspace{-0.71cm}
\begin{center} \boxed{\phantom{hello}} \negthinspace \boxed{\phantom{hello}} \negthinspace \boxed{\phantom{hello}} \end{center}
\vspace{-0.71cm}
\begin{center} \boxed{$- 1,2$} \negthinspace \boxed{$+ 3,3$} \negthinspace \boxed{$+ 4,1$} \negthinspace \boxed{$- 9,3$} \end{center}
  \end{minipage} \\
\end{exercice}


\begin{exercice}[La pyramide (bis)]
\begin{minipage}[c]{0.48\linewidth}
\begin{center} \boxed{\phantom{hello}} \end{center}
\vspace{-0.69cm}
\begin{center} \boxed{\phantom{.}$- 14$\phantom{.}} \negthinspace \boxed{\phantom{hello}} \end{center}
\vspace{-0.71cm}
\begin{center} \boxed{\phantom{hello}} \negthinspace \boxed{\phantom{.}$+ 2$\phantom{.}} \negthinspace  \boxed{\phantom{hello}} \end{center}
\vspace{-0.71cm}
\begin{center} \negthinspace \boxed{\phantom{hello}} \negthinspace \boxed{\phantom{..}$- 7$\phantom{..}} \negthinspace \boxed{\phantom{hello}} \negthinspace \boxed{\phantom{.}$+ 3$\phantom{.}} \end{center}
 \end{minipage} \hfill%
 \begin{minipage}[c]{0.48\linewidth}
\begin{center} \boxed{$- 1,7$} \end{center}
\vspace{-0.69cm}
\begin{center} \boxed{$- 4,5$} \negthinspace \boxed{\phantom{hello}} \end{center}
\vspace{-0.71cm}
\begin{center} \boxed{\phantom{hello}} \negthinspace \boxed{$+ 2,1$} \negthinspace \boxed{\phantom{hello}} \end{center}
\vspace{-0.71cm}
\begin{center} \boxed{\phantom{hello}} \negthinspace \boxed{\phantom{hello}} \negthinspace \boxed{\phantom{hello}} \negthinspace \boxed{$+ 1,2$} \end{center}
  \end{minipage} \\
\end{exercice}


\begin{exercice}
Effectue les additions suivantes en détail :
\begin{enumerate}
 \item $(+ 3) + (- 7) + (- 8) + (+ 2)$ ;
 \item $(- 9) + (- 14) + (+ 25) + (- 3)$ ;
 \item $(- 2,3) + (- 12,7) + (+ 24,7) + (- 1,01)$ ;
 \item $(+ 7,8) + (+ 2,35) + (- 9,55) + (+ 4)$.
 \end{enumerate}
\end{exercice}


\begin{exercice}
Calcule les sommes suivantes en détail :
\begin{enumerate}
 \item $(+ 17) + (- 5) + (+ 4) + (+ 5) + (- 3)$ ;
 \item $(- 12) + (- 4) + (+ 7) + (+ 8) + (- 6)$ ;
 \item $(- 3) + (+ 5) + (- 4) + (+ 6) + (- 1)$ ;
 \item $(+ 1,2) + (- 4,2) + (+ 7,1) + (- 6,7)$.
 \end{enumerate}
\end{exercice}


\begin{exercice}[Durées de vie]
Remarque : pour cet exercice, n'oubliez pas que l'an 0 n'existe pas.
\begin{enumerate}
 \item Cicéron est né en l'an $- 23$ et est mort en l'an 38. Combien de temps a-t-il vécu ?
 \item Thalès de Milet est né en l'an $- 625$ et est mort à l'âge de 78 ans. En quelle année est-il mort ?
 \item L'Empire de Césarius a été créé en $- 330$ et s'est terminé en 213. Combien de temps a-t-il duré ?
 \item Ératosthène est mort en l'an $- 194$ à l'âge de 82 ans. En quelle année est-il né ?
 \item Thésée avait 11 ans à la mort de Claudius. Claudius est mort en l'an $- 18$. Thésée est mort en l'an 31. À quel âge est mort Thésée ?
 \end{enumerate}
\end{exercice}

%%%%%%%%%%%%%%%%%%%%%%%%%%%%%%%%%%%%%%%%%%%%%%%%%%%%%%%%%%%%%%%%%%%


\serie{Différences de relatifs}

\begin{exercice}
Recopie puis complète afin de transformer les soustractions suivantes en additions :
\begin{enumerate}
 \item $(+ 2) - (+ 7) = (+ 2) + (\ldots \ldots)$ ;
 \item $(- 4) - (+ 5) = (- 4) + (\ldots \ldots)$ ;
 \item $(- 8) - (- 14) =  (\ldots \ldots) + (\ldots \ldots)$ ;
 \item $(+ 9) - (- 9) =  (\ldots \ldots) + (\ldots \ldots)$.
 \end{enumerate}
\end{exercice}

\begin{exercice}
Transforme les soustractions suivantes en additions puis effectue-les :
\begin{colenumerate}{2}
 \item $(+ 4) - (+ 15)$ ;
 \item $(- 12) - (+ 5)$ ;
 \item $(- 10) - (- 7)$ ;
 \item $(+ 14) - (- 4)$ ;
 \item $(+ 6) - (+ 6)$ ;
 \item $(- 20) - (+ 7)$.
 \end{colenumerate}
\end{exercice}


\begin{exercice}
Effectue les soustractions suivantes :
\begin{colenumerate}{2}
 \item $(- 2,6) - (+ 7,8)$ ;
 \item $(+ 6,4) - (+ 23,4)$ ;
 \item $(+ 4,5) - (- 12,8)$ ;
 \item $(- 2,7) - (- 9,9)$ ;
 \item $(- 12,8) - (+ 9,5)$ ;
 \item $(+ 6,7) - (+ 2,4)$ ;
 \item $(+ 8,1) - (- 13,6)$ ;
 \item $(- 12,7) - (- 9,8)$.
 \end{colenumerate}
\end{exercice}


\begin{exercice}
Pour chaque expression, transforme les soustractions en additions puis effectue les calculs :
\begin{enumerate}
 \item $(+ 4) - (- 2) + (- 8) - (+ 7)$ ;
 \item $(- 27) - (- 35) - (- 20) + (+ 17)$ ;
 \item $(+ 3,1) + (- 3,5) - (+ 7,8) - (+ 1,6)$ ;
 \item $(- 16,1) - (+ 4,25) + (+ 7,85) - (+ 1,66)$.
 \end{enumerate}
\end{exercice}


\begin{exercice}
Jean et Saïd vont à la fête foraine. Ils misent la même somme d'argent au départ. Jean perd 2,30 CHF puis gagne 7,10 CHF. Saïd gagne 6 CHF puis perd 1,30 CHF. Lequel des deux amis a remporté le plus d'argent à la fin du jeu ?
\end{exercice}


\begin{exercice}
Pour chaque expression, transforme les soustractions en additions puis calcule les sommes :
\begin{enumerate}
 \item $(+ 12) - (- 6) + (- 2) + (+ 7) - (+ 8)$ ;
 \item $(- 20) - (+ 14) + (+ 40) + (- 12) - (- 10)$ ;
 \item $(- 2,4) + (- 7,1) - (- 3,2) - (+ 1,5) + (+ 8,4)$ ;
 \item $(+ 1,9) - (- 6,8) + (- 10,4) + (+ 7,7) - (+ 2)$.
 \end{enumerate}
\end{exercice}


\begin{exercice}
Le professeur Sésamatheux donne à ses élèves un questionnaire à choix multiples (Q.C.M) comportant huit questions. Il note de la façon suivante :
\begin{itemize}
 \item Réponse fausse ($F$) : $- 3$
 \item Sans réponse ($S$) : $- 1$
 \item Réponse bonne ($B$) : $+ 4$
 \end{itemize}
 \begin{enumerate}
 \item Calcule la note de Wenda dont les résultats aux questions sont : $F$ ; $B$ ; $S$ ; $F$ ; $F$ ; $B$ ; $B$ ; $S$. 
 \item Quelle est la note la plus basse qu'un élève peut obtenir ? Et la plus haute ?
 \item Quels sont les résultats possibles pour Emeline qui a obtenu une note $+ 4$ ?
 \end{enumerate}
\end{exercice}


\begin{exercice}
Calcule astucieusement les expressions suivantes :
\begin{enumerate}
 \item $(+ 14) + (- 45) + (- 14) + (+ 15)$ ;
 \item $(- 1,4) + (- 1,2) + (+ 1,6) - (+ 1,6)$ ;
 \item $(+ 1,35) + (- 2,7) - (- 0,65) + (- 1,3)$ ;
 \item $(- 5,45) - (- 0,45) + (+ 1,3) - (- 1) - (+ 1,3)$.
 \end{enumerate}
\end{exercice}


\begin{exercice}
Remplace les pointillés par le nombre qui convient :
\begin{enumerate}
 \item $(- 10) - \ldots \ldots  = 25$ ;
 \item $(+ 16) - \ldots \ldots  = 42$ ;
 \item $(+ 25) - (- 13) + (- 5) + \ldots \ldots = 26$ ;
 \item $(- 63) + (- 8) - \ldots \ldots + (+ 18) = 21$.
 \end{enumerate}
\end{exercice}


\begin{exercice}
Pour chaque cas, calcule en détail $x + y - z$ et $x - (y + z)$ :
\begin{center}
\begin{tabularx}{0.4\linewidth}{|X|c|c|c|}
\hline
 & x & y & z \\ \hline
\textbf{a.} & 10 & $- 3$ & 8 \\ \hline
\textbf{b.} & $- 6$ & $- 5$ & 2 \\ \hline
\textbf{c.} & 3 & $- 8$ & $- 2$ \\ \hline
\textbf{d.} & 7 & $- 2$ & $- 5$ \\ \hline
 \end{tabularx}
 \end{center}
\end{exercice}

%%%%%%%%%%%%%%%%%%%%%%%%%%%%%%%%%%%%%%%%%%%%%%%%%%%%%%%%%%%%%%%%%%%


\serie{Écriture simplifiée}

\begin{exercice}
Relie chaque expression à son écriture simplifiée :
\begin{center}
 \begin{tabularx}{0.8\linewidth}{|cc|X|cc|}
  \cline{1-2}\cline{4-5}
  $(- 8) + (- 16)$ & $\cdot$ & & $\cdot$ & $- 14 - 3$ \\ \cline{1-2}\cline{4-5}
  $(+ 24) - (- 4)$ & $\cdot$ & & $\cdot$ & $- 8 - 16$ \\ \cline{1-2}\cline{4-5}
  $(- 14) + (- 3)$ & $\cdot$ & & $\cdot$ & $14 + 8$ \\ \cline{1-2}\cline{4-5}
  $(- 7) - (+ 7)$ & $\cdot$ & & $\cdot$ & $- 7 - 7$ \\ \cline{1-2}\cline{4-5}
  $(+ 14) + (+ 8)$ & $\cdot$ & & $\cdot$ & $24 + 4$ \\ \cline{1-2}\cline{4-5}
  \end{tabularx}
  \end{center}
\end{exercice}


\begin{exercice}
Recopie et complète le tableau :
\begin{center}
\begin{tabularx}{1.04\linewidth}{|c|X|c|}
\hline
 & Écriture avec parenthèses & Écriture simplifiée \\ \hline
\textbf{a.} & \small{$(- 9) - (+ 13) + (- 15)$} & \\ \hline
\textbf{b.} & \small{$(- 10) + (+ 7) - (- 3) - (- 3)$} & \\ \hline
\textbf{c.} & \small{$(+ 5) - (- 2) + (+ 3) - (+ 2)$} & \\ \hline
\textbf{d.} & & \small{$- 6 - 8 + 5 - 3$} \\ \hline
\textbf{e.} & & \small{$15 - 13 - 8 - 7$} \\ \hline
\textbf{f.} & & \small{$- 13 - 5 - 9 + 1$} \\ \hline
 \end{tabularx}
 \end{center}
\end{exercice}


\begin{exercice}
Donne une écriture simplifiée des expressions suivantes en supprimant les parenthèses et les signes qui ne sont pas nécessaires:
\begin{enumerate}
 \item $(- 5) + (- 3)$ ;
 \item $(- 4) - (+ 6)$ ;
 \item $(+ 9) - (- 3)$ ;
 \item $(+ 4) + (+ 7)$ ;
 \item $(+ 17) - (- 5) + (+ 4) - (+ 5) - (- 3)$ ;
 \item $(- 15) + (+ 3,5) - (- 7,9) + (- 13,6)$.
 \end{enumerate}
\end{exercice}


\begin{exercice}
Effectue les calculs suivants :
\begin{colenumerate}{2}
 \item $5 - 14$ ;
 \item $8 - 13$ ;
 \item $- 6 - 6$ ;
 \item $- 13 + 9$ ;
 \item $- 53 - 48$ ;
 \item $- 2,8 - 4,7$ ;
 \item $- 5,7 + 4,4$ ;
 \item $3,2 - 8,9$.
 \end{colenumerate}
\end{exercice}


\begin{exercice}
Effectue en détail :
\begin{colenumerate}{2}
 \item $24 - 36 + 18$ ;
 \item $- 13 - 28 + 35$ ;
 \item $- 3,8 - 4,4 + 8,2$ ;
 \item $18 - 8 + 4 - 14$ ;
 \item $- 1,3 + 4,4 - 21$ ;
 \item $14 - 23 + 56 - 33$.
 \end{colenumerate}
\end{exercice}


\begin{exercice}
Effectue en détail :
\begin{enumerate}
 \item $5 + 13 - 4 + 3 - 6$ ;
 \item $- 7 + 5 - 4 - 8 + 13$ ;
 \item $3,5 - 4,2 + 6,5 - 3,5 + 5$ ;
 \item $25,2 + 12 - 4,8 + 24 - 3,4$.
 \end{enumerate}
\end{exercice}


\begin{exercice}
Regroupe les termes astucieusement puis effectue en détail :
\begin{enumerate}
 \item $13 + 15 + 7 - 15$ ;
 \item $- 8 + 4 + 18 - 2 + 12 + 6$ ;
 \item $4,3 - 7,4 + 4 - 2,25 + 6,7 + 3,4 - 2,75$ ;
 \item $- 2,5 + 4,8 - 3,6 + 0,2 + 2,5$.
 \end{enumerate}
\end{exercice}


\begin{exercice}
Calcule les expressions suivantes, en détail :
\begin{enumerate}
 \item $(- 3 + 9) - (4 - 11) - (- 5 - 6)$ ;
 \item $- 3 + 12 - (13 - 8) - (3 + 8)$ ;
 \item $- 3 - [4 - (3 - 9)]$.
 \end{enumerate}
\end{exercice}


\begin{exercice}[Températures]
Pour mesurer la température, il existe plusieurs unités. Celle que nous utilisons en Suisse est le degré Celsius ($^{\circ}$C). Cette unité est faite de façon à ce que la température à laquelle l'eau se transforme en glace est $0^{\circ}$C et celle à laquelle l'eau se transforme en vapeur est $100^{\circ}$C. Dans cette échelle, il existe des températures négatives. \\[0.5em]
Il existe une autre unité, le Kelvin (K), dans laquelle les températures négatives n'existent pas. Pour passer de l'une à l'autre, on utilise la formule :
\begin{center} $T_{Kelvin} = T_{degresCelsius} + 273,15$ \end{center}
Ainsi, $10^{\circ}$C correspondent à 283,15 K.
\begin{enumerate}
 \item Convertis en Kelvin les températures suivantes : $24^{\circ}$C ; $- 3^{\circ}$C et $- 22,7^{\circ}$C ;
 \item Convertis en degré Celsius les températures suivantes : 127,7 K ; 276,83 K ; 204 K et 500 K ;
 \item Quelle est en Kelvin la plus petite température possible ? À quelle température en degré Celsius correspond-elle ? Cette température est appelée le zéro absolu.
 \end{enumerate}
\end{exercice}
\end{colonne*exercice}


\exercicesappr
\begin{colonne*exercice}
\begin{exercice}[Sur un axe gradué]
\begin{enumerate}
 \item Soit $A$ le point d'abscisse 4. Quelle peut-être l'abscisse du point $B$ sachant que la longueur du segment $[AB] = 8$ ?
 \item Soit $C$ le point d'abscisse $-3$. Quelle peut-être l'abscisse du point $D$ sachant que la longueur du segment $[CD] = 2$ ?
 \item Soit $E$ le point d'abscisse $-5$. Détermine l'abscisse de $F$ sachant que la longueur du segment $[EF] = 9$ et que l'abscisse de $F$ est inférieure à celle de $E$.
 \end{enumerate}
\end{exercice}


\begin{exercice}
Recopie en remplaçant les $\lozenge$ par le signe $-$ ou le signe $+$ de sorte que les égalités soient vraies :
\begin{enumerate}
 \item $\lozenge \quad 7 \quad \lozenge \quad 3 = -4$ ;
 \item $\lozenge \quad 13 \quad \lozenge \quad 8 = -21$ ;
 \item $\lozenge \quad 3,7 \quad \lozenge \quad 8,4 = 4,7$ ;
 \item $\lozenge \quad 45 \quad \lozenge \quad 72 = -27$ ;
 \item $\lozenge \quad 2 \quad \lozenge \quad 7 \quad \lozenge \quad 13 = -8$ ;
 \item $\lozenge \quad 1,5 \quad \lozenge \quad 2,3 \quad \lozenge \quad 4,9 = -5,7$ ;
 \item $\lozenge \quad 8 \quad \lozenge \quad 5 \quad \lozenge \quad 12 \quad \lozenge \quad 2 = 13$ ;
 \item $\lozenge \quad 7 \quad \lozenge \quad 14 \quad \lozenge \quad 18 \quad \lozenge \quad 3 = -22$.
 \end{enumerate}
\end{exercice}


\begin{exercice}[Carré magique]
Recopie et complète ce carré magique sachant qu'il contient tous les entiers de $- 12$ à $12$ et que les sommes des nombres de chaque ligne, de chaque colonne et de chaque diagonale sont toutes nulles :
\begin{center}
\begin{tabular}{|*5{@{}>{\vrule width0pt height\dimexpr1cm/2+1ex-.2pt\relax depth\dimexpr1cm/2-1ex-.2pt\relax\centering\arraybackslash}p{\dimexpr1cm-.4pt\relax}@{}|}}\hline
    & & 0 & 8 & \\\hline
    & & & $-11$ & 2 \\\hline
   $-9$ & $-1$ & 12 & & 3 \\\hline
   $-3$ & & $-12$ & & 9 \\\hline
   $-2$ & 11 & $-6$ & 7 & \\\hline
\end{tabular}
 \end{center}
\end{exercice}


\begin{exercice}[Triangle magique]
La somme des nombres de chaque côté du triangle est 2. Remplis les cases vides avec les nombres relatifs $(-2)$ ; $(-1)$ ; 1 ; 2 et 3, qui doivent tous être utilisés.
\end{exercice}


\begin{exercice}[Coup de froid]
Chaque matin de la 1\up{re} semaine du mois de Février, Julie a relevé la température extérieure puis a construit le tableau suivant :
\begin{center}
\begin{tabularx}{1.03\linewidth}{|c|*{8}{>{\centering \arraybackslash}X|}}
\hline \cellcolor{H1} Jour & \cellcolor{H2} \small{Lu} & \cellcolor{H2} \small{Ma} & \cellcolor{H2} \small{Me} & \cellcolor{H2} \small{Je} & \cellcolor{H2} \small{Ve} & \cellcolor{H2} \small{Sa} & \cellcolor{H2} \small{Di} \\
\hline \cellcolor{U1} \small{Température (en $^{\circ}$C)} & \cellcolor{U2} \small{$-4$} & \cellcolor{U2} \small{$-2$} & \cellcolor{U2} \small{$-1$} & \cellcolor{U2} \small{$+1$} & \cellcolor{U2} \small{0} & \cellcolor{U2} \small{$+2$} & \cellcolor{U2} \small{$-3$} \\
\hline
\end{tabularx} \\
\end{center}
Calcule la moyenne des températures relevées par Julie.
\end{exercice}


\begin{exercice}
Recopie et complète les carrés magiques suivants :
\begin{enumerate}
 \item Pour l'addition : \\[0.5em]
\begin{center}
\begin{tabular}{|*3{@{}>{\vrule width0pt height\dimexpr1cm/2+1ex-.2pt\relax depth\dimexpr1cm/2-1ex-.2pt\relax\centering\arraybackslash}p{\dimexpr1cm-.4pt\relax}@{}|}}\hline
 &  $-9$ &  $-2$ \\ \hline
 &  $-4$ &  \\ \hline
 $-6$ &  &  \\ \hline
\end{tabular}
\end{center}
\vspace{0.5cm}
 \item Pour l'addition : \\[0.5em]
 \begin{center}
\begin{tabular}{|*3{@{}>{\vrule width0pt height\dimexpr1.4cm/2+1ex-.2pt\relax depth\dimexpr1.4cm/2-1ex-.2pt\relax\centering\arraybackslash}p{\dimexpr1.4cm-.4pt\relax}@{}|}}\hline
 1,6 &  &  \\ \hline
 &  $-5,4$ &  \\ \hline
 $-4,4$ &  &  $-12,4$\\ \hline
\end{tabular}
\end{center}
 \end{enumerate}
\end{exercice}


\begin{exercice}
La différence $a - b$ est égale à 12.

On augmente $a$ de 3 et on diminue $b$ de 4.

Combien vaut la différence entre ces deux nouveaux nombres? 
\end{exercice}


\begin{exercice}[Le nombre $-21$...]
\begin{enumerate}
 \item Écris le nombre $-21$ comme somme de deux nombres entiers relatifs consécutifs ;
 \item Écris le nombre $-21$ comme différence de deux carrés.
 \end{enumerate}
\end{exercice}



\end{colonne*exercice}

\connaissances
\begin{acquis}
\begin{itemize}
\item BlaBla1
\item BlaBla2
\item BlaBla3
\item BlaBla4
\item BlaBla5
\item BlaBla6
\end{itemize}
\end{acquis}

\QCMautoevaluation{Pour chaque question, plusieurs réponses sont
  proposées. Déterminer celles qui sont correctes.}

\begin{QCM}
  \begin{GroupeQCM}
    \begin{exercice}
      $(-10) + (+15) = \ldots$
      \begin{ChoixQCM}{4}
      \item $(-5)$
      \item $(-150)$
      \item $(+5)$
      \item $(-25)$
      \end{ChoixQCM}
\begin{corrige}
     \reponseQCM{a} % j'ai mis "a" partout
   \end{corrige}
    \end{exercice}
    
    
    \begin{exercice}
      $(+8) + \ldots = (-5)$
      \begin{ChoixQCM}{4}
      \item $(+3)$
      \item impossible
      \item $(+13)$
      \item $(-13)$
      \end{ChoixQCM}
\begin{corrige}
     \reponseQCM{a}
   \end{corrige}
    \end{exercice}


    \begin{exercice}
      $(+2,1) + (-3,9) = \ldots$
      \begin{ChoixQCM}{4}
      \item $6$
      \item $-6$
      \item $-1,8$
      \item $1,8$
      \end{ChoixQCM}
\begin{corrige}
     \reponseQCM{a}
   \end{corrige}
    \end{exercice}


    \begin{exercice}
      $(+7) - (-3) = \ldots$
      \begin{ChoixQCM}{4}
      \item $4$
      \item $10$
      \item $-4$
      \item $-10$
      \end{ChoixQCM}
\begin{corrige}
     \reponseQCM{a}
   \end{corrige}
    \end{exercice}


    \begin{exercice}
      $(-2) - \ldots = (-5)$
      \begin{ChoixQCM}{4}
      \item $(+3)$
      \item $(-7)$
      \item $(+7)$
      \item $(-3)$
      \end{ChoixQCM}
\begin{corrige}
     \reponseQCM{a}
   \end{corrige}
    \end{exercice}


    \begin{exercice}
      $1,3 - (-2,4) = \ldots$
      \begin{ChoixQCM}{4}
      \item $-1,1$
      \item $1,1$
      \item $3,7$
      \item $-3,7$
      \end{ChoixQCM}
\begin{corrige}
     \reponseQCM{a}
   \end{corrige}
    \end{exercice}


\end{GroupeQCM}
\end{QCM}

  

%\TravauxPratiques % pour nous "travailler en groupe"
%
\begin{TP}[]

Mon super TP

\end{TP}



\pagebreak

%\recreation
%\begin{enigme}[Des discontinuités... en continu !]


Soit $x$ et $y$ deux réels tels que $x<y$.\\
Définissons la suite  $(d_n)_{n\geqslant0}$ telle que $d_n=\dfrac{\lfloor10^n y\rfloor}{10^n}$ où $\lfloor a\rfloor$ désigne la partie entière de $a$.\\
\begin{enumerate}
\item À quel ensemble les nombres $d_n$  appartiennent-ils ? \hspace{1cm}
\begin{minipage}{0.5\linewidth}
\begin{colitemize}{5}
\item $\mathbb{N}$ ? \item $\mathbb{Z}$ ? \item $\mathbb{D}$ ? \item $\mathbb{Q}$ ? \item $\mathbb{R}$ ?
\end{colitemize}
\end{minipage}
\item
\begin{enumerate}
\item Montrer  que pour tout $n\in\mathbb{N}$, on a l'encadrement $\dfrac{10^ny-1}{10^n}< d_n\leqslant y$.
\item En déduire $\displaystyle\lim_{n\to+\infty}d_n$.
\end{enumerate}
\item \begin{enumerate}
\item Montrer que, quel que soit $\varepsilon>0$, il existe un entier naturel $N$ tel que pour tout $n\geqslant N$, $|d_n-y|< \varepsilon$.
\item En posant $\varepsilon=y-x$, en déduire que $x\leqslant d_N \leqslant y$.
\end{enumerate}
\end{enumerate}

\begin{center}
\begin{minipage}{1\linewidth}
\begin{cadre}[A1][A4]
On vient de montrer qu'entre deux réels, il existe toujours un décimal et donc toujours un rationnel.\\
On dit que l'ensemble des rationnels $\mathbb{Q}$ est \textbf{dense} dans l'ensemble des réels $\mathbb{R}$.
 \end{cadre}
\end{minipage}
\end{center}

La \textbf{fonction de Dirichlet} D et la \textbf{fonction de Thomae} T sont deux fonctions définies sur $\mathbb{R}$ par  :
\[\text{D}(x)=\left\{\begin{matrix} 1  &  \text{si~} x\in\mathbb{Q} \\ 0  & \text{si~} x\notin\mathbb{Q} \end{matrix}\right.
\text{~~et~~} \text{T}(x)=\left\{\begin{array}{@{}ll} 0  & \text{si~} x\notin\mathbb{Q} \\ 1 & \text{si~} x=0 \\ \dfrac{1}{q} & \text{si~} x=\dfrac{p}{q} \text{~est une fraction irréductible}\end{array}\right.\]
Introduite par Dirichlet\footnote{Johann Peter Gustav Lejeune Dirichlet (1805–1859), mathématicien allemand}  en 1829, la fonction D est discontinue partout ce que le résultat établi précédemment montre. Cette fonction est appelée aussi \textbf{fonction indicatrice des rationnels}.\\
Introduite par Thomae\footnote{Carl Johannes Thomae (1840–1921), mathématicien allemand} en 1875, la fonction T est continue en tout nombre irrationnel mais  discontinue en tout nombre rationnel. Cette fonction est appelée aussi la \textbf{fonction popcorn} (voir sa représentation ci-dessous !).
\begin{center}

\end{center}




\end{enigme} 




%\themaG
%\include{SymetrieAxialeCentrale/SymetrieAxialeCentrale}



\AfficheListeMethodes
\AfficheCorriges[2]
\AfficheLexique

\end{document}

%%% Local Variables: 
%%% mode: latex
%%% TeX-master: t
%%% End: 
