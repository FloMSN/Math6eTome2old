\begin{acquis}
\begin{itemize}
\item Déterminer la limite d’une somme,
d’un produit, d’un \mbox{quotient} ou d’une
composée de deux fonctions
\item Déterminer des limites par
comparaison et encadrement
\item Faire le lien entre limites et comportement asymptotique
\item Appréhender la notion de continuité d'une fonction
\item Exploiter le théorème des valeurs
intermédiaires (cas d'une
fonction strictement monotone) pour
résoudre un problème
\item Approcher une solution d'équation par l'algorithmique
\end{itemize}
\end{acquis}

\QCMautoevaluation{Pour chaque question, plusieurs réponses sont
  proposées.  Déterminer celles qui sont correctes.}

% 53
\begin{QCM}
  \begin{GroupeQCM}
    \begin{exercice}
      La limite en $+\infty$ de la fonction $f$ définie sur $]-\infty~;~-1[$ par $f(x)=\dfrac{1+x^2+x^3}{x\left(1-x^2\right)}$ est :
      \begin{ChoixQCM}{4}
      \item $0$
      \item $1$
      \item $-1$
      \item $-\infty$
      \end{ChoixQCM}
\begin{corrige}
     \reponseQCM{c}
   \end{corrige}
    \end{exercice}

    \begin{exercice}
      La limite à gauche  en $0$  de la fonction $f$ définie sur $[-1~;~0[$ par $f(x)=\sqrt{-\dfrac{x+1}{x}}$ est :
      \begin{ChoixQCM}{4}
      \item $0$
      \item $1$
      \item $-\infty$
      \item $+\infty$
      \end{ChoixQCM}
      \begin{corrige}
     \reponseQCM{d}
   \end{corrige}
    \end{exercice}

    % 54
    \begin{exercice}
      La limite en $+\infty$ de la fonction $f$ définie sur $\mathbb{R}\setminus\{-1~;~1\}$ par $f(x)=\dfrac{(2x-3)(x^2+1)}{\left(1-x^2\right)^2}$ est :
      \begin{ChoixQCM}{4}
      \item $-2$
      \item $0$
      \item $+\infty$
      \item $-\infty$
      \end{ChoixQCM}
      \begin{corrige}
     \reponseQCM{b}
   \end{corrige}
    \end{exercice}
  \end{GroupeQCM}
\end{QCM}


\begin{QCM}
  \begin{GroupeQCM}
    \begin{exercice}
      Soit $f$ une fonction définie sur $[2~;~+\infty[$. Si pour tout $x\geqslant 2$, on a $x^2 \leqslant f(x)$ alors :
      \begin{ChoixQCM}{4}
      \item $\displaystyle\lim_{x\to +\infty} f(x)=+\infty$
      \item $\displaystyle\lim_{x\to +\infty} \dfrac{f(x)}{x}=0$
      \item $\displaystyle\lim_{x\to +\infty} \dfrac{f(x)}{x}=+\infty$
      \item $\displaystyle\lim_{x\to +\infty} \dfrac{f(x)}{x^2}=1$
      \end{ChoixQCM}
      \begin{corrige}
     \reponseQCM{ac}
   \end{corrige}
    \end{exercice}


    \begin{exercice}
      La courbe représentative de la fonction
      $h:x\mapsto \dfrac{\left(2x - 1\right)^2}{2(4-x^2)}$ admet une asymptote d'équation :
      \begin{ChoixQCM}{4}
      \item $x = -2$
      \item $y = -2$
      \item $x = 2$
      \item $y = 2$
      \end{ChoixQCM}
      \begin{corrige}
     \reponseQCM{abc}
   \end{corrige}
    \end{exercice}

    \begin{exercice}
    Soit ci-dessous la courbe représentative d'une fonction $f$.
\begin{center}
\psset{xunit=1.cm,yunit=1.cm,algebraic=true}
\begin{pspicture*}(-5.25,-1.2)(5.25,1.4)
\psgrid[subgriddiv=1,linewidth=0.5pt,gridcolor=A3,subgridcolor=A3,gridlabels=0pt](-6,-2)(6,3)
\psaxes[linewidth=0.8pt,Dx=1,Dy=1,ticksize=-2pt]{->}(0,0)(-5.25,-1.2)(5.25,1.4)
%\psaxes[linewidth=0.5pt,Dx=10,Dy=10]{->}(0,0)(1,1)
  \rput[0](1.9,2.5){\textcolor{B2}{$\mathscr{C}$}}
\psplot[linecolor=B2,plotpoints=5000,linewidth=0.8pt]{-5.25}{-1.05}{((x^2+x)/(x^2-1)-(x^2-x)/(-x^2+1))/8}
\psplot[linecolor=B2,plotpoints=5000,linewidth=0.8pt]{-0.95}{0}{((x^2+x)/(x^2-1)-(x^2-x)/(-x^2+1))/8}
\psplot[linecolor=B2,plotpoints=5000,linewidth=0.8pt]{0}{0.95}{((x^2+x)/(x^2-1)-(x^2-x)/(-x^2+1))/8+0.13}
\psplot[linecolor=B2,plotpoints=5000,linewidth=0.8pt]{2.05}{5.25}{(((x-1)^2+x-1)/((x-1)^2-1)-((x-1)^2-x+1)/(-(x-1)^2+1))/8}
\psline[linecolor=B2,linewidth=0.8pt](1.2,-1.2)(1.8,1.4)
\rput(0,0){\textcolor{Blanc}{$\bullet$}} \rput(0,0){\textcolor{B2}{$\circ$}}
\rput(0,0.13){\textcolor{B2}{$\bullet$}}
\end{pspicture*}
\end{center}
Il est certain que la fonction $f$ n'est pas continue :
      \begin{ChoixQCM}{4}
      \item en $-1$
      \item en $0$
      \item en $2$
      \item en $6$
      \end{ChoixQCM}
      \begin{corrige}
     \reponseQCM{ab}
   \end{corrige}
    \end{exercice}




\end{GroupeQCM}
\end{QCM}

  