
\begin{TP}[Un premier TP avec un logo à droite\hfill \algo]

\partie{Le principe et l'algorithme}

La \textbf{méthode de dichotomie} ou \textbf{méthode de la bissection} est un algorithme (voir ci-dessous) de recherche d'un zéro d'une fonction qui consiste à réitérer des partages d’un intervalle en deux  moitiés puis à sélectionner celui dans lequel se trouve le zéro de la fonction.\\
Si cela est possible, on dégrossit le plus souvent la recherche en se plaçant initialement sur un intervalle $[a~;~b]$ où la fonction est continue, strictement monotone
et telle que $f(a)f(b)<0$ afin d'appliquer le théorème des valeurs intermédiaires et assurer ainsi l'unicité de la solution.\\

\begin{minipage}{0.57\linewidth}
\begin{enumerate}
\item Que représente la variable $\varepsilon$ ?
\item Expliquer le premier pas de l'algorithme\par
dans les quatre cas de figures suivants :
\begin{ChoixQCM}{2}
\item \begin{minipage}{\linewidth}
\psset{xunit=0.8cm,yunit=0.8cm,algebraic=true}
\begin{pspicture*}(-0.2,-1.1)(3.2,1.1)
\psaxes[yAxis=false,linewidth=0.8pt,Dx=3,ticksize=-2pt]{->}(0,0)(-0.2,-1.2)(3.2,1.2)
\psplot[linecolor=B2,plotpoints=5000,linewidth=0.8pt]{0}{3}{x^2/4.5-1}
\psline[linestyle=dashed,linewidth=0.5pt](0,0)(0,-1)
\psline[linestyle=dashed,linewidth=0.5pt](3,0)(3,1)
\uput[u](0,0){$a$} \uput[d](3,0){$b$}
\end{pspicture*}
\end{minipage}
\item \begin{minipage}{\linewidth}
\psset{xunit=0.8cm,yunit=0.8cm,algebraic=true}
\begin{pspicture*}(-0.2,-1.1)(3.2,1.1)
\psaxes[yAxis=false,linewidth=0.8pt,Dx=3,ticksize=-2pt]{->}(0,0)(-0.2,-1.2)(3.2,1.2)
\psplot[linecolor=B2,plotpoints=5000,linewidth=0.8pt]{0}{3}{-(x-3)^2/4.5+1}
\psline[linestyle=dashed,linewidth=0.5pt](0,0)(0,-1)
\psline[linestyle=dashed,linewidth=0.5pt](3,0)(3,1)
\uput[u](0,0){$a$} \uput[d](3,0){$b$}
\end{pspicture*}
\end{minipage}
\item \begin{minipage}{\linewidth}
\psset{xunit=0.8cm,yunit=0.8cm,algebraic=true}
\begin{pspicture*}(-0.2,-1.1)(3.2,1.1)
\psaxes[yAxis=false,linewidth=0.8pt,Dx=3,ticksize=-2pt]{->}(0,0)(-0.2,-1.2)(3.2,1.2)
\psplot[linecolor=B2,plotpoints=5000,linewidth=0.8pt]{0}{3}{-x^2/4.5+1}
\psline[linestyle=dashed,linewidth=0.5pt](0,0)(0,1)
\psline[linestyle=dashed,linewidth=0.5pt](3,0)(3,-1)
\uput[d](0,0){$a$} \uput[u](3,0){$b$}
\end{pspicture*}
\end{minipage}
\item \begin{minipage}{\linewidth}
\psset{xunit=0.8cm,yunit=0.8cm,algebraic=true}
\begin{pspicture*}(-0.2,-1.1)(3.2,1.1)
\psaxes[yAxis=false,linewidth=0.8pt,Dx=3,ticksize=-2pt]{->}(0,0)(-0.2,-1.2)(3.2,1.2)
\psplot[linecolor=B2,plotpoints=5000,linewidth=0.8pt]{0}{3}{(x-3)^2/4.5-1}
\psline[linestyle=dashed,linewidth=0.5pt](0,0)(0,1)
\psline[linestyle=dashed,linewidth=0.5pt](3,0)(3,-1)
\uput[d](0,0){$a$} \uput[u](3,0){$b$}
\end{pspicture*}
\end{minipage}
\end{ChoixQCM}
\end{enumerate}
\end{minipage}
\begin{minipage}{0.42\linewidth}
\begin{oldalgorithme}
Lire a, b, $\varepsilon$
Tant que (b-a)>$\varepsilon$
  c prend la valeur (a+b)/2
  Si f(a)*f(c)>0 alors
    a prend la valeur c
  Sinon
    b prend la valeur c
  Fin Si
Fin Tant Que
Afficher c
\end{oldalgorithme}
\end{minipage}

\partie{Application : approcher le nombre d'or}

Intéressons-nous au nombre d'or, solution positive de l'équation :
\[\text{(E)} \quad x^2-x-1=0\]
\begin{enumerate}
\item Soit la fonction $f:x\mapsto x^2-x-1$ qu'on étudie sur $[1~;~2]$.
\begin{enumerate}
\item Justifier que la fonction $f$ est continue sur $[1~;~2]$.
\item Dresser le tableau de variation complet de $f$ sur $[1~;~2]$.
\item Montrer qu'il existe une solution unique $\varphi$ à l'équation $f(x)=0$.
\end{enumerate}
\item On applique l'algorithme de dichotomie à $f$ avec $a=1$, $b=2$ et $\varepsilon =10^{-5}$.
\begin{enumerate}
\item Justifier qu'après le premier pas,  $\varphi\in[1,5~;~2]$ et, qu'après le second, $\varphi\in[1,5~;~1,75]$.\par
\begin{minipage}{0.6\linewidth}
\psset{xunit=6cm,yunit=1.5cm,algebraic=true}
\begin{pspicture*}(0.8,-1.2)(2.2,1.1)
\psgrid[subgriddiv=10,linewidth=0.5pt,gridcolor=A3,subgridcolor=A3,gridlabels=0pt](1,-1)(2,1)
\psaxes[comma,Ox=1,linewidth=0.8pt,Dx=0.1,Dy=1,ticksize=-2pt]{->}(1,0)(1,-1)(2.05,1.1)
\psplot[linecolor=B2,plotpoints=5000,linewidth=0.8pt]{1}{1.5}{x^2-x-1}
\psplot[linecolor=H1!60!black,plotpoints=5000,linewidth=0.8pt]{1.5}{2}{x^2-x-1}
\end{pspicture*}
\end{minipage}\hspace{1cm}
\begin{minipage}{0.35\linewidth}
\psset{xunit=6cm,yunit=1.5cm,algebraic=true}
\begin{pspicture*}(1.3,-1.2)(2.2,1.1)
%\psgrid[subgriddiv=20,linewidth=0.5pt,gridcolor=A3,subgridcolor=A3,gridlabels=0pt](1.5,-1)(2,1)
\multido{\dx=1.50\psxunit+0.05\psxunit}{11}{%
  \psline[linewidth=0.5pt,linecolor=A3](\dx,-1)(\dx,1)
}
\multido{\dy=-1.00\psyunit+0.05\psyunit}{41}{%
  \psline[linewidth=0.5pt,linecolor=A3](1.5,\dy)(2.0,\dy)
}
\psaxes[comma,Ox=1.5,linewidth=0.8pt,Dx=0.1,Dy=1,ticksize=-2pt]{->}(1.5,0)(1.5,-1)(2.05,1.1)
\psplot[linecolor=H1!60!black,plotpoints=5000,linewidth=0.8pt]{1.5}{1.75}{x^2-x-1}
\psplot[linecolor=B2,plotpoints=5000,linewidth=0.8pt]{1.75}{2}{x^2-x-1}
\end{pspicture*} 
\end{minipage}
\item À l'aide d'AlgoBox ou d'un autre logiciel, programmer l'algorithme de dichotomie pour qu'il affiche les encadrements successifs de $\varphi$ et leurs précisions.
\begin{center}
\begin{tabular}{c@{\,}c@{\,}c|c}
1,5 & $<\varphi<$ &   2  &  0,5 \\
1,5 & $<\varphi<$ &   1,75  &  0,25 \\
 & \vdots  &     & \vdots \\
\end{tabular}
\end{center}
\end{enumerate}
\item On définit  la suite $(p_n)_{n\geqslant0}$ par $p_0=1$ et $p_{n+1}=\dfrac{p_n}{2}$.
\begin{enumerate}
\item Que représente $(p_n)$ ? Justifier qu'elle est décroissante et exprimer $p_n$ en fonction de $n$.
\item Écrire puis programmer un algorithme qui prend en entrée $\varepsilon$ et qui retourne le plus petit entier $n$ tel que $p_n<\varepsilon$ ?
\item À l'aide du programme, déterminer le plus petit entier $n$ tel que  $p_n$ soit inférieur à :
\begin{colitemize}{5}
\item $0,1$ \item $0,01$ \item $0,001$ \item $0,000\,1$ \item $0,000\,01$
\end{colitemize}
Commenter l'efficacité de l'algorithme de dichotomie à partir des résultats obtenus.  
\end{enumerate}
\end{enumerate}
\end{TP} 

\begin{TP}[Et un autre TP avec deux logos à droite\hfill\tice \algo]

La \textbf{méthode de Newton} est une autre méthode destinée à déterminer une valeur approchée du zéro d'une fonction, sous condition de sa dérivabilité sur un intervalle réel.\\
Partant d'un réel $x_0$ de préférence proche du zéro à trouver, on approche la fonction $f$ au premier ordre en la considérant à peu près égale à la fonction affine donnée par l'équation de la tangente à sa courbe représentative au point d'abscisse $x_0$ :
\[f(x)\simeq  f'(x_0)(x-x_0)+f(x_0).\]
On  résout alors l'équation $f'(x_0)(x-x_0)+f(x_0)=0$ pour obtenir $x_1$ qui, en général, est plus proche du zéro de $f$ que $x_0$. 
On réitère ensuite le processus.\\[2mm]
Le but de ce TP est de déterminer une valeur approchée du nombre d'or $\varphi$ comme dans le TP précédent et de comparer l'efficacité de la méthode de Newton à celle de dichotomie.
\partie{Approche graphique}\vspace{-5mm}
\begin{enumerate}
\item Avec un logiciel de géométrie dynamique, tracer le graphe $\mathscr{C}$  de $f:x\mapsto x^2-x-1$.
\item Tracer la tangente à $\mathscr{C}$ au point d'abscisse $x_0=1$. Elle coupe l'axe des abscisses en $A_1(x_1~;~0)$.
\item Réitérer le processus pour obtenir $x_1$ puis $x_2$. Est-on proche de $\varphi$ ?
\end{enumerate}

\partie{Avec l'algorithmique}
La construction devient vite compliquée avec l'agglomérat des tangentes successives.\\
On souhaite ainsi s'orienter vers l'élaboration et la programmation d'un algorithme.
\begin{enumerate}
\item Justifier qu'on peut définir la suite $(x_n)$ telle que $x_{n+1}=x_n-\dfrac{f(x_n)}{f'(x_n)}$.
\item Écrire et programmer l'algorithme en considérant la condition d'arrêt $|x_{n+1}-x_n|<\varepsilon$. 
\item Faire tourner l'algorithme pour $\varepsilon$ égal à $10^{-1}$, $10^{-2}$, \ldots, $10^{-5}$.
\item Rajouter un compteur d'itérations pour estimer l'efficacité de la méthode. Conclure.
\end{enumerate}
\end{TP}
